\section{Transition properties modeling}
\label{sec_3_3_SVD_NMF}
In the subsection \ref{subsec_2_4_2_QT}, it has been explained that \gls{cnmc} has two built-in modal decomposition methods for the $\bm Q / \bm T$ tensors, i.e., \gls{svd} and NMF.
There are two main concerns for which performance measurements are needed.
First, in subsection \ref{subsec_3_3_1_SVD_Speed}, the computational costs of both methods are examined.
Then in subsection \ref{subsec_3_3_2_SVD_Quality}, the \gls{svd} and \gls{nmf} prediction quality will be presented and assessed.

\subsection{Computational cost}
\label{subsec_3_3_1_SVD_Speed}
In this subsection, the goal is to evaluate the computational cost of the two decomposition methods implemented in \gls{cnmc}.
\gls{nmf} was already used in \emph{first CNMc} and it was found to be one of the most computational expensive tasks.
With an increasing model order $L$ it became the most computational task by far, which is acknowledged by \cite{Max2021}. 
The run time was one of the main reasons why \gls{svd} should be implemented in \gls{cnmc}. 
To see if \gls{svd} can reduce run time, both methods shall be compared.\newline 

First, it is important to mention that \gls{nmf} is executed for one single predefined mode number $r$. 
It is possible that a selected $r$ is not optimal, since $r$ is a parameter that depends not only on the chosen dynamical system but also on other parameters, e.g., the number of centroids $K$ and training model parameter values $n_{\beta, tr}$, as well as \gls{nmf} specific attributes. 
These are the maximal number of iterations in which the optimizer can converge and tolerance convergence.
However, to find an appropriate $r$, \gls{nmf} can be executed multiple times with different values for $r$.
Comparing the execution time of \gls{nmf} with multiple invocations against \gls{svd} can be regarded as an unbalanced comparison.
Even though for a new dynamical system and its configuration the optimal $r_{opt}$ for \gls{nmf} is most likely to be found over a parameter study, for the upcoming comparison, the run time of one single \gls{nmf} solution is measured.\newline 

The model for this purpose is \emph{SLS}. Since \emph{SLS} is trained with the output of 7 pairwise different model parameter values $n_{\beta,tr} = 7$, the maximal rank in \gls{svd} is limited to 7.
Nevertheless, allowing \gls{nmf} to find a solution $r$ was defined as $r=9$, the maximal number of iterations in which the optimizer can converge is 10 million and the convergence tolerance is $1\mathrm{e}{-6}$.
Both methods can work with sparse matrices. 
However, the \gls{svd} solver is specifically designed to solve sparse matrices.
The measured times for decomposing the $\bm Q / \bm T$ tensors for 7 different $L$ are listed in table \ref{tab_6_NMF_SVD}.
It can be observed that for \gls{svd} up to $L=6$, the computational time for both $\bm Q / \bm T$ tensors is less than 1 second.
Such an outcome is efficient for science and industry applications.
With $L=7$ a big jump in time for both $\bm Q / \bm T$ is found.
However, even after this increase, the decomposition took around 5 seconds, which still is acceptable.\newline 

\begin{table}
    \centering
    \begin{tabular}{c| c c |c c  }
        \textbf{$L$} &\textbf{SVD} $\bm Q$ & \textbf{NMF} $\bm Q$
        &\textbf{SVD} $\bm T$ & \textbf{NMF} $\bm T$\\
        \hline \\
        [-0.8em]
         $1$  & $2 \,\mathrm{e}{-4}$ s  & $64$ s & $8 \, \mathrm{e}{-05}$ s & $3 \, \mathrm{e}{-2}$  s \\

         $2$  & $1 \, \mathrm{e}{-4}$ s  & $8 \, \mathrm{e}{-2}$ s  & $1 \, \mathrm{e}{-4}$ s & $1$ h   \\

         $3$  & $2 \, \mathrm{e}{-4}$ s  & $10$ s  & $2 \, \mathrm{e}{-4}$ s & $0.1$ s \\

         $4$  & $4 \, \mathrm{e}{-3}$ s  & $20$ s & $7 \, \mathrm{e}{-3}$ s & $1.5$ h \\

         $5$  & $6 \, \mathrm{e}{-2}$ s  & $> 3$ h   & $3 \, \mathrm{e}{-2}$ s & -   \\

         $6$  & $0.4$ s & -     & $0.4$ s  & -    \\

         $7$  & $5.17$ s & -      & $4.52$ s & -  
    \end{tabular}
    \caption{Execution time for \emph{SLS} of \gls{nmf} and \gls{svd} for different $L$ }
    \label{tab_6_NMF_SVD}
\end{table}

Calculating $\bm Q$ with \gls{nmf} for $L=1$ already takes 64 seconds. 
This is more than \gls{svd} demanded for $L=7$. 
The $\bm T$ tensor on the other is much faster and is below a second.
However, as soon as $L=2$ is selected, $\bm T$ takes 1 full hour, $L=4$ more than 1 hour.
The table for \gls{nmf} is not filled, since running $\bm Q$ for $L=5$ was taking more than 3 hours, but still did not finish. 
Therefore, the time measurement was aborted. 
This behavior was expected since it was already mentioned in \cite{Max2021}.
Overall, the execution time for \gls{nmf} is not following a trend, e.g., computing $\bm T$ for $L=3$ is faster than for $L=2$ and $\bm Q$ for $L=4$ is faster than for $L=1$.
In other words, there is no obvious rule, on whether even a small $L$ could lead to hours of run time.\newline 

It can be concluded that \gls{svd} is much faster than \gls{nmf} and it also shows a clear trend, i.e. the computation time is expected to increase with $L$.
\gls{nmf} on the other hand first requires an appropriate mode number $r$, which most likely demands a parameter study.
However, even for a single \gls{nmf} solution, it can take hours.
With increasing $L$ the amount of run time is generally expected to increase, even though no clear rule can be defined.
Furthermore, it needs to be highlighted that \gls{nmf} was tested on a small model, where $n_{\beta,tr} = 7$. The author of this thesis experienced an additional increase in run time when $n_{\beta,tr}$ is selected higher. 
Also, executing \gls{nmf} on multiple dynamical systems or model configurations might become infeasible in terms of time.
Finally, with the implementation of \gls{svd}, the bottleneck in modeling $\bm Q / \bm T$ could be eliminated.


\subsection{Prediction quality}
\label{subsec_3_3_2_SVD_Quality}
In this subsection, the quality of the \gls{svd} and \gls{nmf} $\bm Q / \bm T$ predictions are evaluated. 
The used model configuration for this aim is \emph{SLS}.
First, only the $\bm Q$ output with \gls{svd} followed by \gls{nmf} shall be analyzed and compared. Then, the same is done for the $\bm T$ output.\newline


In order to see how many modes $r$ were chosen for \gls{svd} the two figures \ref{fig_54} and \ref{fig_55} are shown. 
It can be derived that with $r = 4$, $99 \%$ of the information content could be captured. The presented results are obtained for $\bm Q$ and $L =1$.\newline

\begin{figure}[!h]
    %\vspace{0.5cm}
    \begin{minipage}[h]{0.47\textwidth}
        \centering
        \includegraphics[width =\textwidth]
        {2_Figures/3_Task/2_Mod_CPE/10_lb_Q_Cumlative_E.pdf}
        \caption{\emph{SLS}, \gls{svd}, cumulative energy of $\bm Q$ for $L=1$}
        \label{fig_54}    
    \end{minipage}
    \hfill
    \begin{minipage}{0.47\textwidth}
        \centering
        \includegraphics[width =\textwidth]
        {2_Figures/3_Task/2_Mod_CPE/11_lb_Q_Sing_Val.pdf}
        \caption{\emph{SLS}, \gls{svd}, singular values of $\bm Q$ for $L=1$}
        \label{fig_55}    
    \end{minipage}
\end{figure}

Figures \ref{fig_56} to \ref{fig_58} depict the original $\bm{Q}(\beta_{unseen} = 28.5)$, which is generated with CNM, the \gls{cnmc} predicted $\bm{\tilde{Q}}(\beta_{unseen} = 28.5)$ and their deviation $| \bm{Q}(\beta_{unseen} = 28.5) -  \bm{\tilde{Q}}(\beta_{unseen} = 28.5) |$, respectively.
In the graphs, the probabilities to move from centroid $c_p$ to $c_j$ are indicated. 
Contrasting figure \ref{fig_56} and \ref{fig_57} exhibits barely noticeable differences.
For highlighting present deviations, the direct comparison between the \gls{cnm} and \gls{cnmc} predicted $\bm Q$ tensors is given in figure \ref{fig_58}.
It can be observed that the highest value is $max( \bm{Q}(\beta_{unseen} = 28.5) -  \bm{\tilde{Q}}(\beta_{unseen} = 28.5) |) \approx 0.0697 \approx 0.07$. 
Note that all predicted $\bm Q$ and $\bm T$ tensors are obtained with \gls{rf} as the regression model.
\newline

\begin{figure}[!h]
    %\vspace{0.5cm}
    \begin{subfigure}[h]{0.5\textwidth}
        \centering
        \caption{Original $\bm{Q}(\beta_{unseen} = 28.5)$}
        \includegraphics[width =\textwidth]
        {2_Figures/3_Task/2_Mod_CPE/12_lb_0_Q_Orig_28.5.pdf}
        \label{fig_56}    
    \end{subfigure}
    \hfill
    \begin{subfigure}{0.5\textwidth}
        \centering
        \caption{\gls{cnmc} predicted $\bm{\tilde{Q}}(\beta_{unseen} = 28.5)$ }
        \includegraphics[width =\textwidth]
        {2_Figures/3_Task/2_Mod_CPE/13_lb_2_Q_Aprox_28.5.pdf}
        \label{fig_57}    
    \end{subfigure}

    \smallskip
    \centering
    \begin{subfigure}{0.7\textwidth}
        \caption{Deviation $| \bm{Q}(\beta_{unseen}) -  \bm{\tilde{Q}}(\beta_{unseen}) |$ }
        \includegraphics[width =\textwidth]
        {2_Figures/3_Task/2_Mod_CPE/14_lb_4_Delta_Q_28.5.pdf}
        \label{fig_58}       
    \end{subfigure}
    \vspace{-0.3cm}
    \caption{\emph{SLS}, \gls{svd}, original $\bm{Q}(\beta_{unseen} = 28.5)$ , \gls{cnmc} predicted $\bm{\tilde{Q}}(\beta_{unseen} = 28.5)$ and  deviation $| \bm{Q}(\beta_{unseen} = 28.5) -  \bm{\tilde{Q}}(\beta_{unseen} = 28.5) |$ for $L=1$}
    \label{fig_58_Full}       
\end{figure}

The same procedure shall now be performed with NMF. 
The results are depicted in figures \ref{fig_59} and \ref{fig_60}.
Note that the original \gls{cnm} $\bm{Q}(\beta_{unseen} = 28.5)$ does not change, thus figure \ref{fig_56} can be reused.  
By exploiting figure \ref{fig_61}, it can be observed that the highest deviation for the \gls{nmf} version is $max( \bm{Q}(\beta_{unseen} = 28.5) -  \bm{\tilde{Q}}(\beta_{unseen} = 28.5) |) \approx 0.0699  \approx 0.07$.
The maximal error of \gls{nmf} $(\approx 0.0699)$ is slightly higher than that of \gls{svd}  $(\approx 0.0697)$. 
Nevertheless, both methods have a very similar maximal error and seeing visually other significant differences is hard.
\newline

\begin{figure}[!h]
    %\vspace{0.5cm}
    \begin{subfigure}[h]{0.5\textwidth}
        \centering
        \caption{\gls{cnmc} predicted $\bm{\tilde{Q}}(\beta_{unseen} = 28.5)$ }
        \includegraphics[width =\textwidth]
        {2_Figures/3_Task/2_Mod_CPE/15_lb_2_Q_Aprox_28.5.pdf}
        \label{fig_59}    
    \end{subfigure}
    \hfill
    \begin{subfigure}{0.5\textwidth}
        \caption{Deviation $| \bm{Q}(\beta_{unseen} ) -  \bm{\tilde{Q}}(\beta_{unseen} ) |$}
        \includegraphics[width =\textwidth]
        {2_Figures/3_Task/2_Mod_CPE/16_lb_4_Delta_Q_28.5.pdf}
        \label{fig_60}    
    \end{subfigure}
    \vspace{-0.3cm}
    \caption{\emph{SLS}, \gls{nmf}, \gls{cnmc} predicted $\bm{\tilde{Q}}(\beta_{unseen} = 28.5)$ and deviation $| \bm{Q}(\beta_{unseen} = 28.5) -  \bm{\tilde{Q}}(\beta_{unseen} = 28.5) |$ for $L=1$}
\end{figure}

In order to have a quantifiable error value, the Mean absolute error (MAE) following equation \eqref{eq_23} is leveraged.
The MAE errors for \gls{svd} and \gls{nmf} are  $MAE_{SVD} = 0.002 580 628$ and $MAE_{NMF} = 0.002 490 048$, respectively.
\gls{nmf} is slightly better than \gls{svd} with $ MAE_{SVD}  -  MAE_{NMF} \approx 1 \mathrm{e}{-4}$, which can be considered to be negligibly small. 
Furthermore, it must be stated that \gls{svd} was only allowed to use $r_{SVD} = 4$ modes, due to the $99 \%$ energy demand, whereas \gls{nmf} used $r_{NMF} = 9$ modes.
Given that \gls{svd} is stable in computational time, i.e., it is not assumed that for low $L$, the computational cost scales up to hours, \gls{svd} is the clear winner for this single comparison. \newline


For the sake of completeness, the procedure shall be conducted once  as well for the $\bm T$ tensor.
For this purpose figures \ref{fig_61} to \ref{fig_65} shall be considered.
It can be inspected that the maximal errors for \gls{svd} and \gls{nmf} are $max( \bm{T}(\beta_{unseen} = 28.5) -  \bm{\tilde{T}}(\beta_{unseen} = 28.5) |) \approx 0.126 $ and
 $max( \bm{T}(\beta_{unseen} = 28.5) -  \bm{\tilde{T}}(\beta_{unseen} = 28.5) | ) \approx 0.115 $, respectively. 
The MAE errors are, $MAE_{SVD} = 0.002 275 379 $ and $MAE_{NMF} = 0.001 635 510$. 
\gls{nmf} is again slightly better than \gls{svd} with $ MAE_{SVD}  -  MAE_{NMF} \approx 6 \mathrm{e}{-4}$, which is a deviation of $\approx 0.06 \%$ and might also be considered as negligibly small. \newline


%------------------------------------- SVD T -----------------------------------
\begin{figure}[!h]
    \begin{subfigure}{0.5 \textwidth}
        \centering
        \caption{Original \gls{cnm} $\bm{T}(\beta_{unseen} = 28.5)$ }
        \includegraphics[width =\textwidth]
        {2_Figures/3_Task/2_Mod_CPE/17_lb_1_T_Orig_28.5.pdf}
        \label{fig_61}    
    \end{subfigure}
    \hfill
    \begin{subfigure}{.5 \textwidth}
        \centering
        \caption{\gls{cnmc} predicted $\bm{\tilde{T}}(\beta_{unseen} = 28.5)$}
        \includegraphics[width =\textwidth]
        {2_Figures/3_Task/2_Mod_CPE/18_lb_3_T_Aprox_28.5.pdf}
        \label{fig_62}    
    \end{subfigure}
    
    \smallskip
    \centering
    \begin{subfigure}{0.7\textwidth}
        \caption{Deviation $| \bm{T}(\beta_{unseen}) -  \bm{\tilde{T}}(\beta_{unseen}) |$}
        \includegraphics[width =\textwidth]
        {2_Figures/3_Task/2_Mod_CPE/19_lb_5_Delta_T_28.5.pdf}
        \label{fig_63}    
    \end{subfigure}
    \vspace{-0.3cm}
    \caption{\emph{SLS}, \gls{svd}, original $\bm{T}(\beta_{unseen} = 28.5)$, predicted $\bm{\tilde{T}}(\beta_{unseen} = 28.5)$ and deviation $| \bm{T}(\beta_{unseen} = 28.5) -  \bm{\tilde{T}}(\beta_{unseen} = 28.5) |$ for $L=1$}
\end{figure}


%------------------------------------- NMF T -----------------------------------
\begin{figure}[!h]
    %\vspace{0.5cm}
    \begin{subfigure}[h]{0.5\textwidth}
        \centering
        \caption{\gls{cnmc} predicted $\bm{\tilde{T}}(\beta_{unseen} = 28.5)$}
        \includegraphics[width =\textwidth]
        {2_Figures/3_Task/2_Mod_CPE/20_lb_3_T_Aprox_28.5.pdf}
        \label{fig_64}    
    \end{subfigure}
    \hfill
    \begin{subfigure}{0.5\textwidth}
        \centering
        \caption{Deviation $| \bm{T}(\beta_{unseen}) -  \bm{\tilde{T}}(\beta_{unseen}) |$}
        \includegraphics[width =\textwidth]
        {2_Figures/3_Task/2_Mod_CPE/21_lb_5_Delta_T_28.5.pdf}
        \label{fig_65}    
    \end{subfigure}
    \vspace{-0.3cm}
    \caption{\emph{SLS}, \gls{nmf}, \gls{cnmc} predicted $\bm{\tilde{T}}(\beta_{unseen} = 28.5)$ and deviation $| \bm{T}(\beta_{unseen} = 28.5) -  \bm{\tilde{T}}(\beta_{unseen} = 28.5) |$ for $L=1$}
\end{figure}

Additional MAE errors for different $L$ and $\beta{unseen}= 28.5,\, \beta{unseen}= 32.5$ are collected in table \ref{tab_7_NMF_SVD_QT}.
First, it can be outlined that regardless of the chosen method, \gls{svd} or \gls{nmf}, all encountered MAE errors are very small. 
Consequently, it can be recorded that \gls{cnmc} convinces with an overall well approximation of the $\bm Q / \bm T$ tensors.
Second, comparing \gls{svd} and \gls{nmf} through their respective MAE errors, it can be inspected that the deviation of both is mostly in the order of $ \mathcal{O} \approx 1 \mathrm{e}{-2}$. 
It is a difference in $\approx 0.1 \%$ and can again be considered to be insignificantly small.\newline 

Despite this, \gls{nmf} required the additional change given in equation \eqref{eq_33}, which did not apply to \gls{svd}. 
The transition time entries at the indexes where the probability is positive should be positive as well. Yet, this is not always the case when \gls{nmf} is executed. To correct that, these probability entries are manually set to zero.
This rule was also actively applied to the results presented above. 
Still, the outcome is very satisfactory, because the modeling errors are found to be small. 
\newline

\begin{table}[!h]
    \centering
    \begin{tabular}{c c| c c|  c c  }
        \textbf{$L$} &$\beta_{unseen}$
         & $\boldsymbol{MAE}_{\gls{svd}, \bm Q}$  
         &$\boldsymbol{MAE}_{\gls{nmf}, \bm Q}$ 
         & $\boldsymbol{MAE}_{\gls{svd}, \bm T}$  
         &$\boldsymbol{MAE}_{\gls{nmf}, \bm T}$ \\
        \hline \\
        [-0.8em]
         $1$  & $28.5$ 
         & $0.002580628 $ & $0.002490048$ 
         & $0.002275379 $ & $0.001635510$\\
         
         $1$  & $32.5$ 
         & $0.003544923$ & $0.003650155$ 
         & $0.011152145$ & $0.010690052$\\
         
         $2$  & $28.5$ 
         & $0.001823848$ & $0.001776276$ 
         & $0.000409955$ & $0.000371242$\\
         
         $2$  & $32.5$ 
         & $0.006381635$ & $0.006053059$ 
         & $0.002417142$ & $0.002368680$\\
         
         $3$  & $28.5$ 
         & $0.000369228$ & $0.000356817$ 
         & $0.000067680$ & $0.000062964$\\

         $3$  & $32.5$ 
         & $0.001462458$ & $0.001432738$ 
         & $0.000346298$ & $0.000343520$\\
         

         $4$  & $28.5$ 
         & $0.000055002$ & $0.000052682$ 
         & $0.000009420$ & $0.000008790$\\

         $4$  & $32.5$ 
         & $0.000215147$ & $0.000212329$ 
         & $0.000044509$ & $0.000044225$
    
    \end{tabular}
    \caption{\emph{SLS}, Mean absolute error for different $L$ and two $\beta_{unseen}$}
    \label{tab_7_NMF_SVD_QT}
\end{table}

\begin{equation}
    \begin{aligned}
        TGZ := \bm T ( \bm Q  > 0) \leq 0 \\
        \bm Q ( TGZ) := 0
    \end{aligned}
    \label{eq_33}
\end{equation}

In summary, both methods \gls{nmf} and \gls{svd} provide a good approximation of the $\bm Q / \bm T$ tensors.
The deviation between the prediction quality of both is negligibly small.
However, since \gls{svd} is much faster than \gls{nmf} and does not require an additional parameter study, the recommended decomposition method is \gls{svd}.
Furthermore, it shall be highlighted that \gls{svd} used only $r = 4$ modes for the $\bm Q$ case, whereas for \gls{nmf} $r=9$ were used. 
Finally, as a side remark, all the displayed figures and the MAE errors are generated and calculated with \gls{cnm}'s default implemented methods. 
\FloatBarrier