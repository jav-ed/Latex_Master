\section{CPE modeling results}
\label{sec_3_2_MOD_CPE}
In this section, results to the \gls{cpevol} modeling explained in subsection \ref{subsec_2_4_1_CPE}, shall be presented and assessed.
First, a selection of equations, which defines the \gls{cpevol} are given for one model configuration.
Next, representative plots of the \gls{cpevol} for different models are analyzed.
Finally, the predicted centroid position is compared with the actual clustered centroid position.\newline 


Modeling the \emph{CPE} returns, among other results, analytical equations. 
These equations describe the behavior of the centroid positions across the range $\vec{\beta}$ and can also be used for making predictions for $\vec{\beta}_{unseen}$.
The model configuration for which they are be presented is \emph{SLS}, i.e. Lorenz system \eqref{eq_6_Lorenz}, $K= 10,\, \vec{\beta }_{tr} = [\, \beta_0 = 28 ; \, \beta_{end} =33  \,], \, n_{\beta, tr} = 7$. 
The analytical \gls{cpevol} expressions are listed in \eqref{eq_27} to \eqref{eq_29} for the centroids  $[\,1,\, 2,\,7\,]$, respectively.
Recalling that the behavior across the 3 different axes (x, y, z) can vary greatly, each axis has its own regression model $(\tilde x,\, \tilde y,\, \tilde z)$.
Thus, for each label, 3 different analytical expressions are provided. \newline


\begin{figure}[!h]
    \begin{minipage}{.47\textwidth}
      \begin{equation}
        \begin{aligned}
            \tilde x &= -0.1661 \, cos(3  \, \beta) \\
            \tilde y &=  -0.1375 \, cos(3 \,  \beta) \\
            \tilde z &=  0.8326 \, \beta 
        \end{aligned}
        \label{eq_27}
      \end{equation}
    \end{minipage}%
    \hfill
    \begin{minipage}{.47\textwidth}
        \centering
        \includegraphics[width =\textwidth]{2_Figures/3_Task/2_Mod_CPE/1_lb_1.pdf}
        \caption{\emph{SLS}, \emph{CPE} model for centroid: 1 }
        \label{fig_45}    
    \end{minipage}
\end{figure}

\begin{figure}[!h]
    \begin{minipage}{.47\textwidth}
        \begin{equation}
            \begin{aligned}
            \tilde x &= 0.1543 \, sin(3 \, \beta) + 0.2446 \, cos(3 \, \beta) \\
            \tilde y &= 0.2638 \, sin(3 \, \beta) + 0.4225 \, cos(3 \, \beta) \\
            \tilde z &= 0.4877 \, \beta
        \end{aligned}
        \label{eq_28}
    \end{equation}
\end{minipage}%
\hfill
\begin{minipage}{.47\textwidth}
    \centering
    \includegraphics[width =\textwidth]{2_Figures/3_Task/2_Mod_CPE/2_lb_2.pdf}
    \caption{\emph{SLS}, \emph{CPE} model for centroid: 2 }
    \label{fig_46}    
\end{minipage}
\end{figure}

\begin{figure}[!h]
    \begin{minipage}{.47\textwidth}
      \begin{equation}
        \begin{aligned}
            \tilde x &= -0.1866 \, \beta + 0.133 \, sin(3 \, \beta) \\
            & \quad + 0.1411 \, cos(3 \, \beta) \\
            \tilde y &= -0.3 \, \beta \\
            \tilde z &= -1.0483+ 0.6358 \,\beta
        \end{aligned}
        \label{eq_29}
      \end{equation}
    \end{minipage}%
    \hfill
    \begin{minipage}{.47\textwidth}
        \centering
        \includegraphics[width =\textwidth]{2_Figures/3_Task/2_Mod_CPE/3_lb_7.pdf}
        \caption{\emph{SLS}, \emph{CPE} model for centroid: 7 }
        \label{fig_47}    
    \end{minipage}
\end{figure}


Right to the equations the corresponding plots are depicted in figures \ref{fig_45} to \ref{fig_47}. 
Here, the blue and green curves indicate true and modeled CPE, respectively.
Each of the figures supports the choice of allowing each axis to be modeled separately.
The z-axis appears to undergo less alteration or to be more linear than the x- and y-axis.
If a model is supposed to be valid for all 3 axes, a more complex model, i.e., a higher of terms, is required.
Although more flexible models fit training data increasingly better, they tend to overfit. 
In other words, complex models capture the trained data well but could exhibit oscillations for $\vec{\beta}_{unseen}$.
The latter is even more severe when the model is employed for extrapolation.
The difference between interpolation and extrapolation is that for extrapolation the prediction is made with $\beta_{unseen}$ which are not in the range of the trained $\vec{\beta}_{tr}$.
Therefore, less complexity is preferred.\newline 

Next, the performance of predicting the centroid for $\vec{\beta}_{unseen}$ is elaborated.
For this purpose, figures \ref{fig_48} to \ref{fig_52} shall be examined.
All figures depict the original centroid positions, which are obtained through the clustering step in green and the predicted centroid positions in blue.
On closer inspection, orange lines connecting the true and predicted centroid positions can be identified. 
Note, that they will only be visible if the deviation between the true and predicted state is high enough.
Figures \ref{fig_48_0} an \ref{fig_48_1} show the outcome for \emph{SLS} with $\beta_{unseen} = 28.5$ and $\beta_{unseen} = 32.5$, respectively.
Visually, both predictions are very close to the true centroid positions.
Because of this high performance in showed in figures \ref{fig_49_0} and \ref{fig_49_1} two examples for extrapolation are given for $\beta_{unseen} = 26.5$ and $\beta_{unseen} = 37$, respectively. 
For the first one, the outcome is very applicable. 
In contrast, $\beta_{unseen} = 37$ returns some deviations which are notably high.
\newline
 


% ----------------- Interpolation ----------------------------------------------
\begin{figure}[!h]
    \begin{subfigure}{0.5\textwidth}
        \centering
        \caption{$\beta_{unseen} = 28.5$ }
        \includegraphics[width =\textwidth]{2_Figures/3_Task/2_Mod_CPE/4_lb_c_28.5.pdf}
        %MSE = 0.622
        \label{fig_48_0}    
    \end{subfigure}%
    \hfill
    \begin{subfigure}{0.5\textwidth}
        \centering
        \caption{$\beta_{unseen} = 32.5$ }
        \includegraphics[width =\textwidth]{2_Figures/3_Task/2_Mod_CPE/5_lb_c_32.5.pdf}
        %MSE = 0.677
        \label{fig_48_1}    
    \end{subfigure}
    \vspace{-0.3cm}
    \caption{\emph{SLS}, original vs. modeled centroid position, $\beta_{unseen} = 28.5$ and $\beta_{unseen} = 32.5$ }
    \label{fig_48}    
\end{figure}

% ----------------- EXTRAPOLATION ----------------------------------------------
\begin{figure}[!h]
    \begin{subfigure}{0.5\textwidth}
        \centering
        \caption{$\beta_{unseen} = 26.5$ }
        \includegraphics[width =\textwidth]{2_Figures/3_Task/2_Mod_CPE/22_lb_c_26.5.pdf}
        %MSE = 0.622
        \label{fig_49_0}    
    \end{subfigure}%
    \hfill
    \begin{subfigure}{0.5\textwidth}
        \centering
        \caption{$\beta_{unseen} = 37$ }
        \includegraphics[width =\textwidth]{2_Figures/3_Task/2_Mod_CPE/23_lb_c_37.0.pdf}
        %MSE = 0.677
        \label{fig_49_1}    
    \end{subfigure}
    \vspace{-0.3cm}
    \caption{\emph{SLS}, original vs. modeled centroid position, extrapolated $\beta_{unseen} = 26.5$ and $\beta_{unseen} = 37$ }
    \label{fig_49}    
\end{figure}


% --------- MODEL LOrenz K= 20
\begin{figure}[!h]
    \begin{subfigure}{.5\textwidth}
        \centering
        \caption{$\beta_{unseen} = 31.75$}
        \includegraphics[width =\textwidth]{2_Figures/3_Task/2_Mod_CPE/6_lb_c_31.75.pdf}
        %MSE = 1.857
    \end{subfigure}%
    \hfill
    \begin{subfigure}{.5\textwidth}
        \centering
        \caption{$\beta_{unseen} = 51.75$ }
        \includegraphics[width =\textwidth]{2_Figures/3_Task/2_Mod_CPE/7_lb_c_51.75.pdf}
        %MSE = 2.536
    \end{subfigure}
    \vspace{-0.3cm}
    \caption{\emph{LS20}, original vs. modeled centroid position, $\beta_{unseen} = 31.75$ and $\beta_{unseen} = 51.75$}
    \label{fig_50}    
\end{figure}

% --------- MODEL 25_Four_Wing_1_K_15 ---------
\begin{figure}[!h]
    \begin{subfigure}{.5\textwidth}
        \centering
        \caption{$\beta_{unseen} = 8.7$}
        \includegraphics[width =\textwidth]{2_Figures/3_Task/2_Mod_CPE/8_lb_c_8.7.pdf}
        %MSE = 1.617
    \end{subfigure}%
    \hfill
    \begin{subfigure}{.5\textwidth}
        \centering
        \caption{$\beta_{unseen} = 10.1$ }
        \includegraphics[width =\textwidth]{2_Figures/3_Task/2_Mod_CPE/9_lb_c_10.1.pdf}
        %MSE = 1.5
    \end{subfigure}
    \vspace{-0.3cm}
    \caption{\emph{FW15}, original vs. modeled centroid position, $\beta_{unseen} = 8.7$ and $\beta_{unseen} = 10.1$}
    \label{fig_52}    
    
\end{figure}

Quantitative measurements are performed by applying the Mean Square Error (MSE) following equation \eqref{eq_30_MSE}. 
The variables are denoted as the number of samples $n$, which in this case is equal to the number of centroids $n = K$, the known $f(x_k)$ and the predicted $y_k$ centroid position.\newline

\begin{equation}
        MSE = \frac{1}{n} \, \sum_{i=1}^n \left(f(x_k) - y_k\right)^2
        \label{eq_30_MSE}
\end{equation}

The measured MSE errors for all displayed results are summarized in table \ref{tab_5_MSE}. 
The MSE for results of $\beta_{unseen} = 28.5$ and $\beta_{unseen} = 32.5$ in figures \ref{fig_48} is $0.622$ and $0.677$, respectively. 
Consequently, the performance of \gls{cnmc} is also confirmed quantitatively.
Figures in \ref{fig_50} illustrate the outcome for \emph{LS20} for $\beta_{unseen} = 31.75$ and $\beta_{unseen} = 51.75$.
In section \ref{sec_3_1_Tracking_Results} it is explained that for  \emph{LS20} cluster network deformations appear. 
Nevertheless, the outcome visually and quantitatively endorses the \emph{CPE} modeling capabilities.
Figures in \ref{fig_52}  depict the outcome for \emph{FW15} for $\beta_{unseen} = 8.7$ and $\beta_{unseen} = 10.1$.
A few orange lines are visible, however overall the outcome is very satisfactory.\newline

\begin{table}
    \centering
    \begin{tabular}{c c c c }
        \textbf{Figure} &\textbf{Model} & $\boldsymbol{\beta_{unseen}}$ & \textbf{MSE} \\
        \hline \\
        [-0.8em]
        \ref{fig_48} & \emph{SLS}& $28.5$  & $0.622$   \\
        \ref{fig_48} & \emph{SLS}& $32.5$  & $0.677$   \\
        \ref{fig_49} & \emph{SLS}& $26.5$  & $1.193$   \\
        \ref{fig_49} & \emph{SLS}& $37$  & $5.452$   \\
        \ref{fig_50} & \emph{LS20}& $31.75$  & $1.857$ \\
        \ref{fig_50} & \emph{LS20}& $51.75$  & $2.536$ \\
        \ref{fig_52} & \emph{FW15}& $8.7$  & $1.617$   \\
        \ref{fig_52} & \emph{FW15}& $10.1$  & $1.5$    
    \end{tabular}
    \caption{MSE for different Model configurations and $\vec{\beta}_{unseen}$}
    \label{tab_5_MSE}
\end{table}

It can be concluded that the \emph{CPE} modeling performance is satisfying.
In the case of a few cluster network deformations, \gls{cnmc} is capable of providing acceptable results.
However, as shown with \emph{SLS}, if the model's training range $\vec{\beta}_{tr}$ and the number of $K$ was selected appropriately, the MSE can be minimized.