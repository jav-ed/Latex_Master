\section{CNMc predictions}
\label{sec_3_5_Pred}
In this section, some representative outputs for the \gls{cnmc} predicted trajectories shall be discussed.
For that, first, the quality measurement abilities implemented in \gls{cnmc} are elaborated.
Next, the model \emph{SLS} is analyzed and explained in detail in the subsection \ref{subsec_3_5_1_SLS}.
Finally, the outcome for other models shall be presented briefly in subsection \ref{subsec_3_5_2_Models}.\newline

There are several methods implemented in \gls{cnmc} to assess the quality of the predicted trajectories.
The first one is the autocorrelation, which will be calculated for all $\vec{\beta}_{unseen}$ and all provided $\vec{L}$, for the true, \gls{cnm} and \gls{cnmc} predicted trajectories.
As usual, the output is plotted and saved as HTML files for a feature-rich visual inspection.
For qualitative assessment, the MAE errors are calculated for all $\vec{\beta}_{unseen}$ and $\vec{L}$ for two sets.
The first set consists of the MAE errors between the true and the \gls{cnm} predicted trajectories.
The second set contains the MAE errors between the true and the \gls{cnmc} predicted trajectories.
Both sets are plotted as MAE errors over $L$ and stored as HTML files.
Furthermore, the one $L$ value which exhibits the least MAE error is printed in the terminal and can be found in the log file as well. \newline

The second technique is the \gls{cpd}, which will also be computed for all the 3 trajectories, i.e., true, \gls{cnm} and \gls{cnmc} predicted trajectories.
The \gls{cpd} depicts the probability of being at one centroid $c_i$.
For each $\vec{\beta}_{unseen}$ and all $L$ the \gls{cpd} is plotted and saved.
The third method displays all the 3 trajectories in the state space.
Moreover, the trajectories are plotted as 2-dimensional graphs, i.e., each axis as a subplot over the time $t$.
The final method calculates the MAE errors of the $\bm Q / \bm T$ tensors for all $L$.\newline 

The reason why more than one quality measurement method is integrated into \gls{cnmc} is that \gls{cnmc} should be able to be applied to, among other dynamical systems, chaotic systems.
The motion of the Lorenz system \eqref{eq_6_Lorenz} is not as complex as of the, e.g., the \emph{Four Wing} \eqref{eq_10_4_Wing}.
Nevertheless, the Lorenz system already contains quasi-random elements, i.e., the switching from one ear to the other cannot be captured exactly with a surrogate mode. However, the characteristic of the Lorenz system and other chaotic dynamical systems as well can be replicated.
In order to prove the latter, more than one method to measure the prediction quality is required.