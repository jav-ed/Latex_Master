\section{Tracking results}
\label{sec_3_1_Tracking_Results}
In this section, some outputs of tracking data and workflow, described in subsection \ref{subsec_2_3_1_Tracking_Workflow}, shall be presented. 
After that, in short, the current \gls{cnmc} shall be compared to \emph{first CNMc} \newline

First, two illustrative solutions for the assignment problem from the final path, as explained in subsection \ref{subsec_2_3_1_Tracking_Workflow}, are provided in figures \ref{fig_27} and \ref{fig_28}. 
The axes are denoted as $c_k$ and $c_p$ and represent the labels of the $\beta_j$ and $\beta_i$ centroids, respectively.
The color bar on the right side informs about the euclidean distance, which is equivalent to the cost.
Above the solution of the assignment problem in figures \ref{fig_27} and \ref{fig_28}, the corresponding $\beta_i$ and $\beta_j$ centroid labels are given in the respective two figures, i.e., \ref{fig_27_1}, \ref{fig_27_2} and \ref{fig_28_1}, \ref{fig_28_2}.

\begin{figure}[!h]
    \begin{subfigure}{0.5\textwidth}
        \centering
        \caption{Ordered state, $\beta_i =32.167$ }
        \includegraphics[width =\textwidth]
        {2_Figures/3_Task/1_Tracking/16_lb_32.167.pdf}
        \label{fig_27_1}
    \end{subfigure}
    \hfill
    \begin{subfigure}{0.5\textwidth}
        \centering
        \caption{Ordered state, $\beta_j = 33$}
        \includegraphics[width =\textwidth]
        {2_Figures/3_Task/1_Tracking/17_lb_33.000.pdf}
        \label{fig_27_2}
    \end{subfigure}
    
    \smallskip
    \centering
    \begin{subfigure}{\textwidth}
        \caption{Solution to the assignment problem}
        \includegraphics[width =\textwidth]{2_Figures/3_Task/1_Tracking/1_LSA.pdf}
        \label{fig_27}    
    \end{subfigure}
    \vspace{-0.3cm}
    \caption{Illustrative solution for the assignment problem, $\beta_i =32.167,\, \beta_j = 33 ,\, K =10$}
    \label{fig_27_All}    
\end{figure}
%
%
The centroid $c_{k=1} (\beta_j = 33)$ has its lowest cost to 
$c_{p=3} (\beta_i = 32.167)$. In this case, this is also the solution for the assignment problem, outlined by the blue cross. 
However, the solution to the linear sum assignment problem is not always to choose the minimal cost for one \emph{inter} $\beta$ match.
It could be that one centroid in $\beta_i$ is to found the closest centroid to multiple centroids in $\beta_j$. 
Matching only based on the minimal distance does not include the restriction that exactly one centroid from $\beta_i$ must be matched with exactly one centroid from $\beta_j$. 
The latter demand is incorporated in the solution of the linear sum assignment problem. \newline 


\begin{figure}[!h]
    \begin{subfigure}{0.5\textwidth}
        \centering
        \caption{Ordered state, $\beta_i =31.333$ }
        \includegraphics[width =\textwidth]
        {2_Figures/3_Task/1_Tracking/18_lb_31.333.pdf}
        \label{fig_28_1}
    \end{subfigure}
    \hfill
    \begin{subfigure}{0.5\textwidth}
        \centering
        \caption{Ordered state, $\beta_j = 32.167$}
        \includegraphics[width =\textwidth]
        {2_Figures/3_Task/1_Tracking/16_lb_32.167.pdf}
        \label{fig_28_2}
    \end{subfigure}
    
    \smallskip
    \centering
    \begin{subfigure}{\textwidth}
        \caption{Solution to the assignment problem}
        \includegraphics[width =\textwidth]{2_Figures/3_Task/1_Tracking/2_LSA.pdf}
        \label{fig_28}    
    \end{subfigure}
    \vspace{-0.3cm}
    \caption{Illustrative solution for the assignment problem, $\beta_i =31.333,\, \beta_j = 32.167, \,K =10 $}
    \label{fig_28_All}    
\end{figure}

Comparing figure \ref{fig_27} with the second example in figure \ref{fig_28}, it can be observed that the chosen \emph{inter} $\beta$ centroid matches can have very different shapes. 
This can be seen by looking at the blue crosses.
Furthermore, paying attention to the remaining possible \emph{inter} $\beta$ centroid matches, it can be stated that there is a clear trend, i.e., the next best \emph{inter} $\beta$ centroid match has a very high increase in its cost.
For example, considering the following \emph{inter} $\beta$ match. With $c_{k=1} (\beta_j = 32.167)$ and $c_{p=1} (\beta_i = 31.333)$, the minimal cost is around $cost_{min} \approx 0.84$. The next best option jumps to $cost_{second} = 13.823$. These jumps can be seen for each \emph{inter} $\beta$ match in figure in both depicted figures \ref{fig_27} and \ref{fig_28}.
The key essence behind this finding is that for the chosen number of centroids $K$ of this dynamical model (Lorenz system \eqref{eq_6_Lorenz}), no ambiguous regions, as explained at the beginning of this chapter, occur.\newline 

Next, the tracking result of 3 different systems shall be viewed.
The tracked state for \emph{SLS} is depicted in figures \ref{fig_29}.
In each of the figures, one centroid is colored blue that denotes 
the first centroid in the sequence of the underlying trajectory.
Within the depicted range $\vec{\beta}$, it can be observed, that each label across the $\vec{\beta}$ is labeled as expected.
No single ambiguity or mislabeling can be seen. 
In other words, it highlights the high performance of the tracking algorithm.
%
%
% ==============================================================================
% ======================= SLS =================================================
% ==============================================================================
\begin{figure}[!h]
    \begin{subfigure}{0.5\textwidth}
        \centering
        \caption{$\beta =28$ }
        \includegraphics[width =\textwidth]
        {2_Figures/3_Task/1_Tracking/3_lb_28.000.pdf}
    \end{subfigure}
    \hfill
    \begin{subfigure}{0.5\textwidth}
        \centering
        \caption{ $\beta = 28.833$}
        \includegraphics[width =\textwidth]
        {2_Figures/3_Task/1_Tracking/4_lb_28.833.pdf}
    \end{subfigure}

    \smallskip
    \begin{subfigure}{0.5\textwidth}
        \centering
        \caption{$\beta = 31.333$}
        \includegraphics[width =\textwidth]
        {2_Figures/3_Task/1_Tracking/15_lb_31.333.pdf}
    \end{subfigure}
    \hfill
    \begin{subfigure}{0.5\textwidth}
        \centering
        \caption{ $\beta = 33$}
        \includegraphics[width =\textwidth]
        {2_Figures/3_Task/1_Tracking/5_lb_33.000.pdf}
    \end{subfigure}
    \vspace{-0.3cm}
    \caption{Tracked states for \emph{SLS}, $K = 10,\, \vec{\beta} = [\, 28, \, 28.333, \, 31.333, \, 31.14, \, 33  \, ]$}
    \label{fig_29}
\end{figure}
% ==============================================================================
% ======================= SLS =================================================
% ==============================================================================
%
The second model is the \emph{LS20}, i.e, $K= 20,\, \vec{\beta }_{tr} = [\, \beta_0 = 24.75 ; \, \beta_{end} = 53.75  \,], \, n_{\beta,tr} = 60$.
The outcome is depicted in figures \ref{fig_32}.
It can be noted that $\beta = 24.75$ and $\beta = 30.648$ exhibit very similar results to the \emph{SLS} model. 
The same is true for intermediate $\beta$ values, i.e., $24.75 \leq \beta \lessapprox 30.648 $. 
However,  with $\beta \gtrapprox 30.64$ as depicted for $\beta =  31.14$, one centroid, i.e. the centroid with the label $20$ in the right ear appears unexpectedly. 
With this, a drastic change to the centroid placing network is imposed.
Looking at the upcoming $\beta$ these erratic changes are found again.\newline 


% ==============================================================================
% ======================= LS20 =================================================
% ==============================================================================
\begin{figure}[!h]
    \begin{subfigure}{0.5\textwidth}
        \centering
        \caption{$\beta =24.75$ }
        \includegraphics[width =\textwidth]
        {2_Figures/3_Task/1_Tracking/6_lb_24.750.pdf}
    \end{subfigure}
    \hfill
    \begin{subfigure}{0.5\textwidth}
        \centering
        \caption{ $\beta = 28.682$}
        \includegraphics[width =\textwidth]
        {2_Figures/3_Task/1_Tracking/7_lb_28.682.pdf}
    \end{subfigure}

    \smallskip
    \begin{subfigure}{0.5\textwidth}
        \centering
        \caption{$\beta = 30.648$}
        \includegraphics[width =\textwidth]
        {2_Figures/3_Task/1_Tracking/7_lb_30.648.pdf}
    \end{subfigure}
    \hfill
    \begin{subfigure}{0.5\textwidth}
        \centering
        \caption{ $\beta = 31.140$}
        \includegraphics[width =\textwidth]
        {2_Figures/3_Task/1_Tracking/8_lb_31.140.pdf}
    \end{subfigure}

    \smallskip
    \begin{subfigure}{0.5\textwidth}
        \centering
        \caption{$\beta = 42.936$}
        \includegraphics[width =\textwidth]
        {2_Figures/3_Task/1_Tracking/9_lb_42.936.pdf}
    \end{subfigure}
    \hfill
    \begin{subfigure}{0.5\textwidth}
        \centering
        \caption{ $\beta = 53.750$}
        \includegraphics[width =\textwidth]
        {2_Figures/3_Task/1_Tracking/10_lb_53.750.pdf}
    \end{subfigure}
    \vspace{-0.3cm}
    \caption{Tracked states for \emph{LS20}, $K = 20,\, \vec{\beta} = [\, 24.75, \, 28.682, \, 30.648, \, 31.14, \, 31.14,$ $42.936, \, 53.75  \, ]$ }
    \label{fig_32}
\end{figure}
% ==============================================================================
% ======================= LS20 =================================================
% ==============================================================================
Generating a tracked state with these discontinuous cluster network deformations even manually can be considered hard to impossible because tracking demands some kind of similarity. 
If two cluster networks differ too much from each other, then necessarily at least tracked label is going to be unsatisfying. 
Hence, it would be wrong to conclude that the tracking algorithm is not performing well, but rather the clustering algorithm itself or the range of $\vec{\beta} $ must be adapted. If the range of $\vec{\beta} $  is shortened, multiple models can be trained and tracked.\newline 

\FloatBarrier
The third model is referred to as \emph{FW15}. 
Figures in \ref{fig_38} show the tracked state for 4 different $\beta$ values. It can be observed that for $\beta = 11$ the centroid placing has changed notably to the other $\beta$ values, thus tracking the centroids in the center for $\beta = 11$ becomes unfavorable. 
Overall, however, the tracked state results advocate the performance of the tracking algorithm.\newline

% ==============================================================================
% ======================= FW15 =================================================
% ==============================================================================
\begin{figure}[!h]
    \begin{subfigure}{0.5\textwidth}
        \centering
        \caption{$\beta =8$ }
        \includegraphics[width =\textwidth]
        {2_Figures/3_Task/1_Tracking/11_lb_8.000.pdf}
    \end{subfigure}
    \hfill
    \begin{subfigure}{0.5\textwidth}
        \centering
        \caption{ $\beta = 8.25$}
        \includegraphics[width =\textwidth]
        {2_Figures/3_Task/1_Tracking/12_lb_8.250.pdf}
    \end{subfigure}

    \smallskip
    \begin{subfigure}{0.5\textwidth}
        \centering
        \caption{$\beta = 10$}
        \includegraphics[width =\textwidth]
        {2_Figures/3_Task/1_Tracking/13_lb_10.000.pdf}
    \end{subfigure}
    \hfill
    \begin{subfigure}{0.5\textwidth}
        \centering
        \caption{ $\beta = 11$}
        \includegraphics[width =\textwidth]
        {2_Figures/3_Task/1_Tracking/14_lb_11.000.pdf}
    \end{subfigure}
    \vspace{-0.3cm}
    \caption{Tracked states for \emph{FW15}, $K = 15,\, \vec{\beta} = [\, 8, \, 8.25, \, 10, \, 11 \, ]$}
    \label{fig_38}
\end{figure}
% ==============================================================================
% ======================= FW15 =================================================
% ==============================================================================

It can be concluded that the tracking algorithm performs remarkably well. However, when the cluster placing network is abruptly changed from one $\beta$ to the other $\beta$, the tracking outcome gets worse and generates sudden cluster network deformation. 
As a possible solution, splitting up the $\vec{\beta}_{tr}$ range into smaller $\vec{\beta}_{tr,i}$ ranges, can be named. This is not only seen for the \emph{LS20}, but also for other dynamical systems as illustratively shown with the center area of the \emph{FW15} system for $\beta= 11$.
\FloatBarrier   

