\subsection{Assessment of SLS}
\label{subsec_3_5_1_SLS}
In this subsection, the prediction capability for the \emph{SLS} will be analyzed in detail. All the presented output is generated with \gls{svd} as the decomposition method and \gls{rf} as the $\bm Q / \bm T$ regressor.\newline

The final objective of \gls{cnmc} is to capture the characteristics of the original trajectory.
However, it is important to outline that \gls{cnmc} is trained with the \gls{cnm} predicted trajectories. 
Thus, the outcome of \gls{cnmc} highly depends on the ability of \gls{cnm} to represent the original data. 
Consequently, \gls{cnmc} can only be as effective as \gls{cnm} is in the first place,  with the approximation of the true data.
Figures \ref{fig_72} and \ref{fig_73} show the true, \gls{cnm} and \gls{cnmc} predicted trajectories and a focused view on the \gls{cnm} and \gls{cnmc} trajectories, respectively.
The output was generated for $\beta_{unseen} = 28.5$ and $L =1$.
First, it can be observed that \gls{cnm} is not able to capture the full radius of the Lorenz attractor.
This is caused by the low chosen number of centroids $K=10$.
Furthermore, as mentioned at the beginning of this chapter, the goal is not to replicate the true data one-to-one, but rather to catch the significant behavior of any dynamic system.
With the low number of centroids $K$, \gls{cnm} extracts the characteristics of the Lorenz system well.
Second, the other aim for \gls{cnmc} is to match the \gls{cnm} data as closely as possible.
Both figures \ref{fig_72} and \ref{fig_73} prove that \gls{cnmc} has fulfilled its task very well. \newline

\begin{figure}[!h]
    \begin{subfigure}{0.5\textwidth}
        \centering
        \caption{True, \gls{cnm} and \gls{cnmc} predicted trajectories}
        \includegraphics[width =\textwidth]
        {2_Figures/3_Task/4_SLS/0_lb_28.5_All.pdf}
        \label{fig_72}
    \end{subfigure}
    \hfill
    \begin{subfigure}{0.5\textwidth}
        \centering
        \caption{\gls{cnm} and \gls{cnmc} predicted trajectories}
        \includegraphics[width =\textwidth]
        {2_Figures/3_Task/4_SLS/1_lb_28.5.pdf}
        \label{fig_73}
    \end{subfigure}
    \vspace{-0.3cm}
    \caption{\emph{SLS}, $\beta_{unseen}=28.5,\, L=1$, true, \gls{cnm} and \gls{cnmc} predicted trajectories} 
\end{figure}


A close-up of the movement of the different axes is shown in the picture \ref{fig_74}.
Here, as well, the same can be observed as described above. Namely, the predicted \gls{cnmc} trajectory is not a one-to-one reproduction of the original trajectory.
However, the characteristics, i.e., the magnitude of the motion in all 3 directions (x, y, z) and the shape of the oscillations, are very similar to the original trajectory.
Note that even though the true and predicted trajectories are utilized to assess, whether the characteristical behavior could be extracted, a single evaluation based on the trajectories is not sufficient and often not advised or even possible.
In complex systems, trajectories can change rapidly while dynamical features persist \cite{Fernex2021a}. 
In \gls{cnmc} the predicted trajectories are obtained through the \gls{cnm} propagation, which itself is based on a probabilistic model, i.e. the $\bm Q$ tensor. 
Thus, matching full trajectories becomes even more unrealistic. 
The latter two statements highlight yet again that more than one method of measuring quality is needed. 
To further support the generated outcome the autocorrelation and \gls{cpd} in figure \ref{fig_75} and \ref{fig_76}, respectively, shall be considered.
It can be inspected that the \gls{cnm} and \gls{cnmc} autocorrelations are matching the true autocorrelation in the shape favorably well.
Nonetheless, the degree of reflecting the magnitude fully decreases quite fast.
Considering the \gls{cpd}, it can be recorded that the true \gls{cpd} could overall be reproduced satisfactorily.\newline 

\begin{figure}[!h]
    \centering
    \includegraphics[width =0.75\textwidth]
    {2_Figures/3_Task/4_SLS/2_lb_28.5_3V_All.pdf}
    \caption{\emph{SLS}, $\beta_{unseen}=28.5, \, L=1$, true, \gls{cnm} and \gls{cnmc} predicted trajectories as 2d graphs } 
    \label{fig_74}
\end{figure}


\begin{figure}[!h]
    \begin{subfigure}{0.5\textwidth}
        \centering
        \caption{autocorrelation} 
        \includegraphics[width =\textwidth]
        {2_Figures/3_Task/4_SLS/3_lb_3_all_28.5.pdf}
        \label{fig_75}
    \end{subfigure}
    \hfill
    \begin{subfigure}{0.5\textwidth}
        \centering
        \caption{\gls{cpd}} 
        \includegraphics[width =\textwidth]
        {2_Figures/3_Task/4_SLS/4_lb_28.5.pdf}
        \label{fig_76}
    \end{subfigure}
    \vspace{-0.3cm}
    \caption{\emph{SLS}, $\beta_{unseen}= 28.5, \, L =1$, autocorrelation  and \gls{cpd} for true, \gls{cnm} and \gls{cnmc} predicted trajectories} 
\end{figure}


To illustrate the influence of $L$, figure \ref{fig_77} shall be viewed.
It depicts the MAE error for the true and \gls{cnmc} predicted trajectories for $\beta_{unseen}= [\, 28.5,\, 32.5 \, ]$ with $L$ up to 7.
It can be observed that the choice of $L$ has an impact on the prediction quality measured by autocorrelation.
For $\beta_{unseen}=28.5$ and $\beta_{unseen}=32.5$, the optimal $L$ values are $L = 2$ and $L = 7$, respectively. To emphasize it even more that with the choice of $L$ the prediction quality can be regulated, figure \ref{fig_78} shall be considered.
It displays the 3 autocorrelations for $L = 7$. 
Matching the shape of the true autocorrelation was already established with $L =1$ as shown in figure \ref{fig_75}. In addition to that, $L=7$ improves by matching the true magnitude.
Finally, it shall be mentioned that similar results have been accomplished with other $K$ tested values, where the highest value was $K =50$.

\begin{figure}[!h]
    \begin{minipage}{0.47\textwidth}
        \centering
        \includegraphics[width =\textwidth]
        {2_Figures/3_Task/4_SLS/5_lb_1_Orig_CNMc.pdf}
        \caption{\emph{SLS}, MAE error for true and \gls{cnmc} predicted autocorrelations for $\beta_{unseen}= [\, 28.5,$ $32.5 \, ]$ and different values of $L$} 
        \label{fig_77}
    \end{minipage}
        \hfill
        \begin{minipage}{0.47\textwidth}
            \centering
            \includegraphics[width =\textwidth]
            {2_Figures/3_Task/4_SLS/6_lb_3_all_32.5.pdf}
            \caption{\emph{SLS}, $\beta_{unseen}=32.5, \, L=7$, \gls{cnm} and \gls{cnmc} predicted autocorrelation } 
            \label{fig_78}
        \end{minipage}
    \end{figure}
\FloatBarrier
