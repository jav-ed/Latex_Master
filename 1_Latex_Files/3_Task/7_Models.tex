\subsection{Results of further dynamical systems}
\label{subsec_3_5_2_Models}
In this subsection, the \gls{cnmc} prediction results for other models will be displayed. 
The chosen dynamical systems with their configurations are the following.
% ==============================================================================
\begin{enumerate}
    \item  \emph{FW50}, based on the \emph{Four Wing} set of equations \eqref{eq_10_4_Wing} with $K=50, \, \vec{\beta }_{tr} = [\, \beta_0 = 8 ; \, \beta_{end} = 11 \,], \, n_{\beta, tr} = 13$.

    \item  \emph{Rössler15}, based on the \emph{Rössler} set of equations \eqref{eq_7_Ross} with $K=15, \, \vec{\beta }_{tr} = [\, \beta_0 = 6 ; \, \beta_{end} = 13 \,], \, n_{\beta, tr} = 15$.

    \item  \emph{TS15}, based on the \emph{Two Scroll} set of equations \eqref{eq_9_2_Scroll} with $K=15, \, \vec{\beta }_{tr} = [\, \beta_0 = 5 ; \, \beta_{end} = 12 \,], \, n_{\beta, tr} = 15$.    
\end{enumerate}
All the presented outputs were generated with \gls{svd} as the decomposition method and \gls{rf} as the $\bm Q / \bm T$ regressor.
Furthermore, the B-spline interpolation in the propagation step of \gls{cnm} was replaced with linear interpolation. 
The B-spline interpolation was originally utilized for smoothing the motion between two centroids. 
However, it was discovered for a high number of $K$, the B-spline interpolation is not able to reproduce the motion between two centroids accurately, but rather would impose unacceptable high deviations or oscillations into the predictions. 
This finding is also mentioned in \cite{Max2021} and addressed as one of \emph{ first CNMc's} limitations.  
Two illustrative examples of the unacceptable high deviations caused by the B-spline interpolation are given in figures \ref{fig_82_Traject} and \ref{fig_82_Autocorr}. 
The results are obtained for \emph{LS20} for $\beta = 31.75$ and $\beta = 51.75$ with $L=3$. 
In figures \ref{fig_82_Traj_B} and \ref{fig_83_Traj_B} it can be inspected that the B-spline interpolation has a highly undesired impact on the predicted trajectories.
In Contrast to that, in figures, \ref{fig_82_Traj_L} and \ref{fig_83_Traj_L}, where linear interpolation is utilized, no outliers are added to the predictions.
The impact of the embedded outliers, caused by the B-spline interpolation, on the autocorrelation is depicted in figures \ref{fig_82_Auto_B} and \ref{fig_83_Auto_B}.
The order of the deviation between the true and the \gls{cnmc} predicted autocorrelation can be grasped by inspecting the vertical axis scale.
Comparing it with the linear interpolated autocorrelations, shown in figures \ref{fig_82_Auto_L} and \ref{fig_83_Auto_L}, it can be recorded that the deviation between the true and predicted autocorrelations is significantly lower than in the B-spline interpolation case.
\newline 

Nevertheless, it is important to highlight that the B-spline interpolation is only a tool for smoothing the motion between two centroids. 
The quality of the modeled $\bm Q / \bm T$ cannot be assessed directly by comparing the trajectories and the autocorrelations.
To stress that the \gls{cpd} in figure \ref{fig_82_CPD_B} and \ref{fig_83_CPD_B} shall be considered.
It can be observed that \gls{cpd} does not represent the findings of the autocorrelations, i.e., the true and predicted behavior agree acceptably overall. 
This is because the type of interpolation has no influence on the modeling of the probability tensor $\bm Q$.
Thus, the outcome with the B-spline interpolation should not be regarded as an instrument that enables making assumptions about the entire prediction quality of \gls{cnmc}.  The presented points underline again the fact that more than one method should be considered to evaluate the prediction quality of \gls{cnmc}.
\newline


\begin{figure}[!h]
    \begin{subfigure}{0.5\textwidth}
        \centering
        \caption{Trajectories, B-spline, $\beta_{unseen} = 31.75$ }
        \includegraphics[width =\textwidth]
        {2_Figures/3_Task/5_Models/18_lb_31.75_All.pdf}
        \label{fig_82_Traj_B}
    \end{subfigure}
    \hfill
    \begin{subfigure}{0.5\textwidth}
        \centering
        \caption{Trajectories, B-spline, $\beta_{unseen} = 51.75$}
        \includegraphics[width =\textwidth]
        {2_Figures/3_Task/5_Models/19_lb_51.75_All.pdf}
        \label{fig_83_Traj_B}
    \end{subfigure}
    
    % ------------- Linear ----------------------
    \smallskip
    \begin{subfigure}{0.5\textwidth}
        \centering
        \caption{Trajectories, linear, $\beta_{unseen} = 31.75$ }
        \includegraphics[width =\textwidth]
        {2_Figures/3_Task/5_Models/24_lb_31.75_All.pdf}
        \label{fig_82_Traj_L}
    \end{subfigure}
    \hfill
    \begin{subfigure}{0.5\textwidth}
        \centering
        \caption{Trajectories, linear, $\beta_{unseen} = 51.75$}
        \includegraphics[width =\textwidth]
        {2_Figures/3_Task/5_Models/25_lb_51.75_All.pdf}
        \label{fig_83_Traj_L}
    \end{subfigure}
    \vspace{-0.3cm}
    \caption{Illustrative undesired oscillations cased by the B-spline interpolation and its impact on the predicted trajectory contrasted with linear interpolation, \emph{LS20}, $\beta = 31.75$ and $\beta =51.75$, $L=3$}
    \label{fig_82_Traject}
\end{figure}

%----------------------------------- AUTOCOR -----------------------------------

\begin{figure}[!h]
    \begin{subfigure}{0.5\textwidth}
        \centering
        \caption{Autocorrelations, B-spline, $\beta = 31.75$ }
        \includegraphics[width =\textwidth]
        {2_Figures/3_Task/5_Models/20_lb_3_all_31.75.pdf}
        \label{fig_82_Auto_B}
    \end{subfigure}
    \hfill
    \begin{subfigure}{0.5\textwidth}
        \centering
        \caption{Autocorrelations, B-spline, $\beta_{unseen} = 51.75$}
        \includegraphics[width =\textwidth]
        {2_Figures/3_Task/5_Models/21_lb_3_all_51.75.pdf}
        \label{fig_83_Auto_B}
    \end{subfigure}
    
    \smallskip
    % ------------- LINEAR ----------------------
    \begin{subfigure}{0.5\textwidth}
        \centering
        \caption{Autocorrelations, linear, $\beta = 31.75$ }
        \includegraphics[width =\textwidth]
        {2_Figures/3_Task/5_Models/26_lb_3_all_31.75.pdf}
        \label{fig_82_Auto_L}
    \end{subfigure}
    \hfill
    \begin{subfigure}{0.5\textwidth}
        \centering
        \caption{Autocorrelations, linear, $\beta_{unseen} = 51.75$}
        \includegraphics[width =\textwidth]
        {2_Figures/3_Task/5_Models/27_lb_3_all_51.75.pdf}
        \label{fig_83_Auto_L}
    \end{subfigure}
    \vspace{-0.3cm}
    \caption{Illustrative undesired oscillations cased by the B-spline interpolation and its impact on the predicted autocorrelations contrasted with linear interpolation, \emph{LS20}, $\beta = 31.75$ and $\beta =51.75$, $L=3$}
    \label{fig_82_Autocorr}
\end{figure}
    
\begin{figure}[!h]
    % ------------- CPD ----------------------
    \begin{subfigure}{0.5\textwidth}
        \centering
        \caption{\gls{cpd}, $\beta = 31.75$ }
        \includegraphics[width =\textwidth]
        {2_Figures/3_Task/5_Models/22_lb_31.75.pdf}
        \label{fig_82_CPD_B}
    \end{subfigure}
    \hfill
    \begin{subfigure}{0.5\textwidth}
        \centering
        \caption{\gls{cpd}, $\beta_{unseen} = 51.75$}
        \includegraphics[width =\textwidth]
        {2_Figures/3_Task/5_Models/23_lb_51.75.pdf}
        \label{fig_83_CPD_B}
    \end{subfigure}
    \vspace{-0.3cm}
    \caption{Illustrative the B-spline interpolation and its impact on the \glspl{cpd}, \emph{LS20}, $\beta = 31.75$ and $\beta =51.75$, $L=3$}
\end{figure}

\FloatBarrier
The results generated with the above mentioned linear interpolation for  \emph{FW50}, \emph{Rössler15} and \emph{TS15} are depicted in figures \ref{fig_79} to \ref{fig_81}, respectively. 
Each of them consists of an illustrative trajectory, 3D and 2D trajectories, the autocorrelations, the \gls{cpd} and the MAE error between the true and \gls{cnmc} predicted trajectories for a range of $\vec{L}$ and some $\vec{\beta}_{unseen}$.
The illustrative trajectory includes arrows, which provide additional information.
First, the direction of the motion, then the size of the arrows represents the velocity of the system. Furthermore, the change in the size of the arrows is equivalent to a change in the velocity, i.e., the acceleration.
Systems like the \emph{TS15} exhibit a fast change in the size of the arrows, i.e., the acceleration is nonlinear. 
The more complex the behavior of the acceleration is, the more complex the overall system becomes.
The latter statement serves to emphasize that \gls{cnmc} can be applied not only to rather simple systems such as the Lorenz attractor \cite{lorenz1963deterministic}, but also to more complex systems such as the \emph{FW50} and \emph{TS15}.\newline 

All in all, the provided results for the 3 systems are very similar to those already explained in the previous subsection \ref{subsec_3_5_1_SLS}.
Thus, the results presented are for demonstration purposes and will not be discussed further.
However, the 3 systems also have been calculated with different values for $K$. 
For \emph{FW50}, the range of $\vec{K}= [\, 15, \, 30, \, 50 \, ]$ was explored with the finding that the influence of $K$ remained quite small.
For \emph{Rössler15} and \emph{TS15}, the ranges were chosen as $\vec{K}= [\, 15, \, 30, \, 100\,]$ and $\vec{K}= [\, 15, \, 75 \,]$, respectively.
The influence of $K$ was found to be insignificant also for the latter two systems.
% ==============================================================================
% ======================= FW50 =================================================
% ==============================================================================
\begin{figure}[!h]
    \begin{subfigure}{0.5\textwidth}
        \centering
        \caption{Illustrative trajectory $\beta = 9$ }
        \includegraphics[width =\textwidth]
        {2_Figures/3_Task/5_Models/0_lb_9.000.pdf}
    \end{subfigure}
    \hfill
    \begin{subfigure}{0.5\textwidth}
        \centering
        \caption{Trajectories, $\beta_{unseen} = 8.1$}
        \includegraphics[width =\textwidth]
        {2_Figures/3_Task/5_Models/1_lb_8.1_All.pdf}
    \end{subfigure}

    \smallskip
    \begin{subfigure}{0.5\textwidth}
        \centering
        \caption{2D-trajectories, $\beta_{unseen} = 8.1$}
        \includegraphics[width =\textwidth]
        {2_Figures/3_Task/5_Models/2_lb_8.1_3V_All.pdf}
    \end{subfigure}
    \hfill
    \begin{subfigure}{0.5\textwidth}
        \centering
        \caption{Autocorrelations, $\beta_{unseen} = 8.1$}
        \includegraphics[width =\textwidth]
        {2_Figures/3_Task/5_Models/3_lb_3_all_8.1.pdf}
    \end{subfigure}
    
    
    \smallskip
    \begin{subfigure}{0.5\textwidth}
        \centering
        \caption{\gls{cpd}, $\beta_{unseen} = 8.1$}
        \includegraphics[width =\textwidth]
        {2_Figures/3_Task/5_Models/4_lb_8.1.pdf}
    \end{subfigure}
    \hfill
    \begin{subfigure}{0.5\textwidth}
        \centering
        \caption{Autocorrelations $MAE(L,\, \beta_{unseen})$}
        \includegraphics[width =\textwidth]
        {2_Figures/3_Task/5_Models/5_lb_1_Orig_CNMc.pdf}
    \end{subfigure}
    \vspace{-0.3cm}
    \caption{Results for \emph{FW50}, $\beta_{unseen} = 8.1, \, L= 2$}
    \label{fig_79}
\end{figure}
% ==============================================================================
% ======================= FW50 =================================================
% ==============================================================================

% ==============================================================================
% ======================= Rossler 15 ===========================================
% ==============================================================================
\begin{figure}[!h]
    \begin{subfigure}{0.5\textwidth}
        \centering
        \caption{Illustrative trajectory $\beta = 7.5$ }
        \includegraphics[width =\textwidth]
        {2_Figures/3_Task/5_Models/6_lb_7.500.pdf}
    \end{subfigure}
    \hfill
    \begin{subfigure}{0.5\textwidth}
        \centering
        \caption{Trajectories, $\beta_{unseen} = 9.6$}
        \includegraphics[width =\textwidth]
        {2_Figures/3_Task/5_Models/7_lb_9.6_All.pdf}
    \end{subfigure}

    \smallskip
    \begin{subfigure}{0.5\textwidth}
        \centering
        \caption{2D-trajectories, $\beta_{unseen} = 9.6$}
        \includegraphics[width =\textwidth]
        {2_Figures/3_Task/5_Models/8_lb_9.6_3V_All.pdf}
    \end{subfigure}
    \hfill
    \begin{subfigure}{0.5\textwidth}
        \centering
        \caption{Autocorrelations, $\beta_{unseen} = 9.6$}
        \includegraphics[width =\textwidth]
        {2_Figures/3_Task/5_Models/9_lb_3_all_9.6.pdf}
    \end{subfigure}
    
    
    \smallskip
    \begin{subfigure}{0.5\textwidth}
        \centering
        \caption{\gls{cpd}, $\beta_{unseen} = 9.6$}
        \includegraphics[width =\textwidth]
        {2_Figures/3_Task/5_Models/10_lb_9.6.pdf}
    \end{subfigure}
    \hfill
    \begin{subfigure}{0.5\textwidth}
        \centering
        \caption{Autocorrelations $MAE(L,\, \beta_{unseen})$}
        \includegraphics[width =\textwidth]
        {2_Figures/3_Task/5_Models/11_lb_1_Orig_CNMc.pdf}
    \end{subfigure}
    \vspace{-0.3cm}
    \caption{Results for \emph{Rössler15}, $\beta_{unseen} = 9.6,\, L =1$}
    \label{fig_80}
\end{figure}
% ==============================================================================
% ======================= Rossler 15 ===========================================
% ==============================================================================


% ==============================================================================
% ======================= TS 15 ===========================================
% ==============================================================================
\begin{figure}[!h]
    \begin{subfigure}{0.5\textwidth}
        \centering
        \caption{Illustrative trajectory $\beta = 11$ }
        \includegraphics[width =\textwidth]
        {2_Figures/3_Task/5_Models/12_lb_11.000.pdf}
    \end{subfigure}
    \hfill
    \begin{subfigure}{0.5\textwidth}
        \centering
        \caption{Trajectories, $\beta_{unseen} = 5.1$}
        \includegraphics[width =\textwidth]
        {2_Figures/3_Task/5_Models/13_lb_5.1_All.pdf}
    \end{subfigure}

    \smallskip
    \begin{subfigure}{0.5\textwidth}
        \centering
        \caption{2D-trajectories, $\beta_{unseen} = 5.1$}
        \includegraphics[width =\textwidth]
        {2_Figures/3_Task/5_Models/14_lb_5.1_3V_All.pdf}
    \end{subfigure}
    \hfill
    \begin{subfigure}{0.5\textwidth}
        \centering
        \caption{Autocorrelations, $\beta_{unseen} = 5.1$}
        \includegraphics[width =\textwidth]
        {2_Figures/3_Task/5_Models/15_lb_3_all_5.1.pdf}
    \end{subfigure}
    
    
    \smallskip
    \begin{subfigure}{0.5\textwidth}
        \centering
        \caption{\gls{cpd}, $\beta_{unseen} = 5.1$}
        \includegraphics[width =\textwidth]
        {2_Figures/3_Task/5_Models/16_lb_5.1.pdf}
    \end{subfigure}
    \hfill
    \begin{subfigure}{0.5\textwidth}
        \centering
        \caption{Autocorrelations $MAE(L,\, \beta_{unseen})$}
        \includegraphics[width =\textwidth]
        {2_Figures/3_Task/5_Models/17_lb_1_Orig_CNMc.pdf}
    \end{subfigure}
    \vspace{-0.3cm}
    \caption{Results for \emph{TS15}, $\beta_{unseen} = 5.1,\, L =2$}
    \label{fig_81}
\end{figure}
% ==============================================================================
% ======================= TS 15 ================================================
% ==============================================================================

