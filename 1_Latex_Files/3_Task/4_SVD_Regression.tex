\section{Transition property regression models}
\label{sec_3_4_SVD_Regression}
In this section, the results of the 3 different regression methods, \glsfirst{rf}, AdaBoost and Gaussian Process (GP) are compared.
All the 3 regressors are implemented in \gls{cnmc} and can be selected via \emph{settings.py}.
The utilized model configuration is \emph{SLS} and the decomposition method is \gls{svd}.\newline 


First, it shall be noted that \gls{cnmc} also offers the possibility to apply \emph{pySindy}. 
However, \emph{pySindy} has struggled to represent the training data in the first place, thus it cannot be employed for predicting $\beta_{unseen}$.
The latter does not mean that \emph{pySindy} is not applicable for the construction of a surrogate model for the decomposed $\bm Q / \bm T$ modes, but rather that the selected candidate library was not powerful enough. 
Nevertheless, only results for the 3 initially mentioned regressors will be discussed.\newline 

In figures \ref{fig_66} to \ref{fig_71} the true (dashed) and the approximation (solid) of the first 4 $\bm Q / \bm T$ modes are shown for the methods RF, AdaBoost and GP, respectively.
To begin with, it can be noted that the mode behavior over different model parameter values $mod(\beta)$ is discontinuous, i.e., it exhibits spikes or sudden changes.
In figures \ref{fig_66} and \ref{fig_67} it can be observed that \gls{rf} reflects the actual behavior of $mod(\beta)$ quite well.
However, it encounters difficulties in capturing some spikes. 
AdaBoost on the other hand proves in figures \ref{fig_68} and \ref{fig_69} to represent the spikes better. 
Overall, AdaBoost outperforms \gls{rf} in mirroring training data. \newline

\begin{figure}[!h]
    %\vspace{0.5cm}
    \begin{subfigure}[h]{0.5 \textwidth}
        \centering
        \caption{$\bm Q$}
        \includegraphics[width =\textwidth]
        {2_Figures/3_Task/3_SVD_QT/0_model_Decomp_Regr_RF_More_Q.pdf}
        \label{fig_66}    
    \end{subfigure}
    \hfill
    \begin{subfigure}{0.5 \textwidth}
        \centering
        \caption{$\bm T$}
        \includegraphics[width =\textwidth]
        {2_Figures/3_Task/3_SVD_QT/1_model_Decomp_Regr_RF_More_T.pdf}
        \label{fig_67}    
    \end{subfigure}
    \vspace{-0.3cm}
    \caption{\emph{SLS}, \gls{svd}, $\bm Q / \bm T$ modes approximation with \gls{rf} for $L=1$}
\end{figure}

\begin{figure}[!h]
    %\vspace{0.5cm}
    \begin{subfigure}[h]{0.5 \textwidth}
        \centering
        \caption{$\bm Q$}
        \includegraphics[width =\textwidth]
        {2_Figures/3_Task/3_SVD_QT/2_model_Decomp_Regr_ABoost_More_Q.pdf}
        \label{fig_68}    
    \end{subfigure}
    \hfill
    \begin{subfigure}{0.5 \textwidth}
        \centering
        \caption{$\bm T$}
        \includegraphics[width =\textwidth]
        {2_Figures/3_Task/3_SVD_QT/3_model_Decomp_Regr_ABoost_More_T.pdf}
        \label{fig_69}    
    \end{subfigure}
    \vspace{-0.3cm}
    \caption{\emph{SLS}, \gls{svd}, $\bm Q / \bm T$ mode approximation with AdaBoost for $L=1$}
\end{figure}

\begin{figure}[!h]
    %\vspace{0.5cm}
    \begin{subfigure}[h]{0.5 \textwidth}
        \centering
        \caption{$\bm Q$}
        \includegraphics[width =\textwidth]
        {2_Figures/3_Task/3_SVD_QT/4_model_Decomp_Regr_GP_More_Q.pdf}
        \label{fig_70}    
    \end{subfigure}
    \hfill
    \begin{subfigure}{0.5 \textwidth}
        \centering
        \caption{$\bm T$}
        \includegraphics[width =\textwidth]
        {2_Figures/3_Task/3_SVD_QT/5_model_Decomp_Regr_GP_More_T.pdf}
        \label{fig_71}    
    \end{subfigure}
    \vspace{-0.3cm}
    \caption{\emph{SLS}, \gls{svd}, $\bm Q / \bm T$ mode approximation with GP for $L=1$}
\end{figure}

Gaussian Process (GP) is a very powerful method for regression. 
Often this is also true when reproducing $mod(\beta)$. 
However, there are cases where the performance is insufficient, as shown in figures \ref{fig_70} and \ref{fig_71}. 
Applying GP results in absolutely incorrect predicted tensors $\bm \tilde{Q}(\beta_{unseen}),\, \bm \tilde{T}(\beta_{unseen})$, where too many tensors entries are wrongly forced to zero. 
Therefore, $\bm \tilde{Q}(\beta_{unseen}),\, \bm \tilde{T}(\beta_{unseen})$ will eventually lead to an unacceptably high deviation from the original trajectory.
Consequently, the GP regression is not applicable for the decomposed $\bm Q / \bm T$ modes without further modification.\newline 

The two remaining regressors are \glsfirst{rf} and AdaBoost.
Although AdaBoost is better at capturing the true modal behavior $mod(\beta)$, there is no guarantee that it will always be equally better at predicting the modal behavior for unseen model parameter values $mod(\beta_{unseen})$.
In table \ref{tab_8_RF_ABoost} the MAE errors for different $L$ and $\beta_{unseen} = [\, 28.5,\, 32.5\,]$ are provided. 
Since the table exhibits much information, the results can also be read qualitatively through the graphs \ref{fig_72_QT_28} and \ref{fig_72_QT_32} for $\beta_{unseen} = 28.5$ and $\beta_{unseen} = 32.5$, respectively.
For the visual inspection, it is important to observe the order of the vertical axis scaling.
It can be noted that the MAE errors themselves and the deviation between the \gls{rf} and AdaBoost MAE errors are very low.
Thus, it can be stated that \gls{rf} as well ad AdaBoost are both well-suited regressors.\newline 


\begin{table}[!h]
    \centering
    \begin{tabular}{c c| c c|  c c  }
        \textbf{$L$} &$\beta_{unseen}$
         & $\boldsymbol{MAE}_{RF, \bm Q}$  
         &$\boldsymbol{MAE}_{AdaBoost, \bm Q}$ 
         & $\boldsymbol{MAE}_{RF, \bm T}$  
         &$\boldsymbol{MAE}_{AdaBoost, \bm T}$ \\
        \hline \\
        [-0.8em]
         $1$  & $28.5$ 
         & $0.002580628 $ & $0.002351781$ 
         & $0.002275379 $ & $0.002814208$\\
         
         $1$  & $32.5$ 
         & $0.003544923$ & $0.004133114$ 
         & $0.011152145$ & $0.013054876$\\
         
         $2$  & $28.5$ 
         & $0.001823848$ & $0.001871858$ 
         & $0.000409955$ & $0.000503748$\\
         
         $2$  & $32.5$ 
         & $0.006381635$ & $0.007952153$ 
         & $0.002417142$ & $0.002660403$\\
         
         $3$  & $28.5$ 
         & $0.000369228$ & $0.000386292$ 
         & $0.000067680$ & $0.000082808$\\

         $3$  & $32.5$ 
         & $0.001462458$ & $0.001613434$ 
         & $0.000346298$ & $0.000360097$\\
         

         $4$  & $28.5$ 
         & $0.000055002$ & $0.000059688$ 
         & $0.000009420$ & $0.000011500$\\

         $4$  & $32.5$ 
         & $0.000215147$ & $0.000230404$ 
         & $0.000044509$ & $0.000046467$\\

         $5$  & $28.5$ 
         & $0.000007276$ & $0.000007712$ 
         & $0.000001312$ & $0.000001600$\\

         $5$  & $32.5$ 
         & $0.000028663$ & $0.000030371$ 
         & $0.000005306$ & $0.000005623$\\

         $6$  & $28.5$ 
         & $0.000000993$ & $0.000052682$ 
         & $0.000000171$ & $0.000000206$\\

         $6$  & $32.5$ 
         & $0.000003513$ & $0.000003740$ 
         & $0.000000629$ & $0.000000668$\\

         $7$  & $28.5$ 
         & $0.000000136$ & $0.000000149$ 
         & $0.000000023$ & $0.000000031$ \\

         $7$  & $32.5$ 
         & $0.000000422$ & $0.000000454$ 
         & $0.000000078$ & $0.000000082$

    
    \end{tabular}
    \caption{\emph{SLS}, Mean absolute error for comparing \gls{rf} and AdaBoost different $L$ and two $\beta_{unseen}$}
    \label{tab_8_RF_ABoost}
\end{table}

\begin{figure}[!h]
    %\vspace{0.5cm}
    \begin{subfigure}[h]{0.5 \textwidth}
        \centering
        \caption{$\bm Q$}
        \includegraphics[width =\textwidth]
        {2_Figures/3_Task/3_SVD_QT/6_Q_28_5.pdf}
        \label{fig_72_Q_28}    
    \end{subfigure}
    \hfill
    \begin{subfigure}{0.5 \textwidth}
        \centering
        \caption{$\bm T$}
        \includegraphics[width =\textwidth]
        {2_Figures/3_Task/3_SVD_QT/7_T_28_5.pdf}
        \label{fig_72_T_28}    
    \end{subfigure}
    \vspace{-0.3cm}
    \caption{\emph{SLS}, Mean absolute error for comparing \gls{rf} and AdaBoost different $L$ and $\beta_{unseen} = 28.5$}
    \label{fig_72_QT_28}    
\end{figure}

\begin{figure}[!h]
    %\vspace{0.5cm}
    \begin{subfigure}[h]{0.5 \textwidth}
        \centering
        \caption{$\bm Q$}
        \includegraphics[width =\textwidth]
        {2_Figures/3_Task/3_SVD_QT/8_Q_32_5.pdf}
        \label{fig_72_Q_32}    
    \end{subfigure}
    \hfill
    \begin{subfigure}{0.5 \textwidth}
        \centering
        \caption{$\bm T$}
        \includegraphics[width =\textwidth]
        {2_Figures/3_Task/3_SVD_QT/9_T_32_5.pdf}
        \label{fig_72_T_32}    
    \end{subfigure}
    \vspace{-0.3cm}
    \caption{\emph{SLS}, Mean absolute error for comparing \gls{rf} and AdaBoost different $L$ and $\beta_{unseen} = 32.5$}
    \label{fig_72_QT_32}    
\end{figure}


In summary, the following can be said, \gls{rf} and AdaBoost are both performing well in regression. Furthermore, no clear winner between the two regressors can be detected. 
The third option GP is dismissed as it sometimes has unacceptably low regression performance. 
Finally, there is the possibility to use \emph{pySindy}, however, for that, an appropriate candidate library must be defined.
\FloatBarrier