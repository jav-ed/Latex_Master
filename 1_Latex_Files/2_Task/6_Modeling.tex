\section{Modeling}
\label{sec_2_4_Modeling}
In this section, the fourth main step of \gls{cnmc}, i.e., modeling, is elaborated.
The data and workflow is described in figure \ref{fig_42}.
It comprises two main sub-tasks, which are modeling the \glsfirst{cpevol} and modeling the transition properties tensors $\bm Q / \bm T$. 
The settings are as usually defined in \emph{settings.py} and the extracted attributes are distributed to the sub-tasks. 
Modeling the \gls{cpevol} and the $\bm Q/ \bm T$ tensors can be executed separately from each other. 
If the output of one of the two modeling sub-steps is at hand, \gls{cnmc} is not forced to recalculate both modeling sub-steps.
Since the tracked states are used as training data to run the modeling they are prerequisites for both modeling parts.
The modeling of the centroid position shall be explained in the upcoming subsection \ref{subsec_2_4_1_CPE}, followed by the explanation of the transition properties in subsection \ref{subsec_2_4_2_QT}.
A comparison between this \gls{cnmc} and the \emph{first CNMc} version is provided at the end of the respective subsections.
The results of both modeling steps can be found in section
\ref{sec_3_2_MOD_CPE} and \ref{sec_3_3_SVD_NMF}

\begin{figure} [!h]
    \hspace*{-4cm} 
    \resizebox{1.2\textwidth}{!}{
    
%%% Preamble Requirements %%%
% \usepackage{geometry}
% \usepackage{amsfonts}
% \usepackage{amsmath}
% \usepackage{amssymb}
% \usepackage{tikz}

% Optional packages such as sfmath set through python interface
% \usepackage{sfmath}

% \usetikzlibrary{arrows,chains,positioning,scopes,shapes.geometric,shapes.misc,shadows}

%%% End Preamble Requirements %%%

\input{/home/jav/Progs/Virt_Env/writing/lib/python3.12/site-packages/pyxdsm/diagram_styles.tex}
\begin{tikzpicture}

\matrix[MatrixSetup]{
%Row 0
\node [DataIO] (output_data) {$\begin{array}{c}\text{modelling settings}\end{array}$};&
&
&
&
\\
%Row 1
\node [Function] (data) {$\begin{array}{c}\text{Modeling}\end{array}$};&
\node [DataInter] (data-ode) {$\begin{array}{c}\text{CPE settings,} \\ \text{tracked state}$$\,(\beta)$$\end{array}$};&
\node [DataInter] (data-qt) {$\begin{array}{c}$$\boldsymbol{Q} / \boldsymbol{T}$$\text{ settings,} \\ \text{tracked state}$$\,(\beta)$$\end{array}$};&
&
\\
%Row 2
&
\node [Function,stack] (ode) {$\begin{array}{c}\text{Model} \\ \text{CPE}$$\,(\beta)$$\end{array}$};&
&
&
\node [DataIO] (right_output_ode) {$\begin{array}{c}\text{save: CPE}$$\,(\beta)$$ \\ \text{plots}\end{array}$};\\
%Row 3
&
&
\node [Function,stack] (qt) {$\begin{array}{c}\text{Model} \\ $$\boldsymbol{Q} / \boldsymbol{T} \,(\beta)$$\end{array}$};&
&
\node [DataIO] (right_output_qt) {$\begin{array}{c}\text{save:}$$ \boldsymbol{Q} / \boldsymbol{T} \,(\beta)$$ \\ \text{plots}\end{array}$};\\
%Row 4
&
&
&
&
\\
};

% XDSM process chains
{ [start chain=process]
 \begin{pgfonlayer}{process} 
\chainin (data);
\chainin (ode) [join=by ProcessHVA];
\end{pgfonlayer}
}{ [start chain=process]
 \begin{pgfonlayer}{process} 
\chainin (data);
\chainin (qt) [join=by ProcessHVA];
\end{pgfonlayer}
}

\begin{pgfonlayer}{data}
\path
% Horizontal edges
(data) edge [DataLine] (data-ode)
(data) edge [DataLine] (data-qt)
(ode) edge [DataLine] (right_output_ode)
(qt) edge [DataLine] (right_output_qt)
% Vertical edges
(data-ode) edge [DataLine] (ode)
(data-qt) edge [DataLine] (qt)
(data) edge [DataLine] (output_data);
\end{pgfonlayer}

\end{tikzpicture}

    }
    \caption{Data and workflow of the fourth step: Modeling}
    \label{fig_42}
\end{figure}


\subsection{Modeling the centroid position evolution}
\label{subsec_2_4_1_CPE}
In this subsection, the modeling of the \gls{cpevol} is described. 
The objective is to find a surrogate model, which returns all $K$ centroid positions for an unseen $\beta_{unseen}$. 
The training data for this are the tracked centroids from the previous step, which is described in section \ref{sec_2_3_Tracking}. 
To explain the modeling of the \emph{CPE}, figure \ref{fig_43} shall be inspected.
The model parameter values which shall be used to train the model $\vec{\beta_{tr}}$ are used for generating a so-called candidate library matrix $\boldsymbol{\Theta}\,(\vec{\beta_{tr}})$. The candidate library matrix $\boldsymbol{\Theta}\,(\vec{\beta_{tr}})$ is obtained making use of a function of \emph{pySindy} \cite{Silva2020,Kaptanoglu2022,Brunton2016}.
In \cite{Brunton2016} the term $\boldsymbol{\Theta}\,(\vec{\beta_{tr}})$ is explained well. However, in brief terms, it allows the construction of a matrix, which comprises the output of defined functions. 
These functions could be, e.g., a linear, polynomial, trigonometrical or any other non-linear function. Made-up functions that include logical conditions can also be applied. \newline 

\begin{figure} [!h]
    \hspace*{-4cm} 
    \resizebox{1.2\textwidth}{!}{
    
%%% Preamble Requirements %%%
% \usepackage{geometry}
% \usepackage{amsfonts}
% \usepackage{amsmath}
% \usepackage{amssymb}
% \usepackage{tikz}

% Optional packages such as sfmath set through python interface
% \usepackage{sfmath}

% \usetikzlibrary{arrows,chains,positioning,scopes,shapes.geometric,shapes.misc,shadows}

%%% End Preamble Requirements %%%

\input{/home/jav/Progs/Virt_Env/writing/lib/python3.12/site-packages/pyxdsm/diagram_styles.tex}
\begin{tikzpicture}

\matrix[MatrixSetup]{
%Row 0
\node [DataIO] (output_theta) {$\begin{array}{c}\text{tracked state}$$\,(\vec{\beta})$$\end{array}$};&
&
&
&
&
&
\\
%Row 1
\node [Function] (theta) {$\begin{array}{c}$$ \boldsymbol \theta \, (\vec{\beta})  $$\end{array}$};&
\node [DataInter] (theta-linear) {$\begin{array}{c}$$ \boldsymbol \theta \, (\vec{\beta}) $$\end{array}$};&
\node [DataInter] (theta-pysindy) {$\begin{array}{c}$$ \boldsymbol \theta \, (\vec{\beta})  $$\end{array}$};&
\node [DataInter] (theta-elastic) {$\begin{array}{c}$$ \boldsymbol \theta \, (\vec{\beta}) $$\end{array}$};&
&
&
\\
%Row 2
&
\node [Function] (linear) {$\begin{array}{c}\text{Linear}\end{array}$};&
&
&
\node [DataInter] (linear-comp) {$\begin{array}{c}$$ md_{lin} $$\end{array}$};&
&
\\
%Row 3
&
&
\node [Function] (pysindy) {$\begin{array}{c}\text{pySindy}\end{array}$};&
&
\node [DataInter] (pysindy-comp) {$\begin{array}{c}$$ md_{pysind}  $$\end{array}$};&
&
\\
%Row 4
&
&
&
\node [Function] (elastic) {$\begin{array}{c}\text{Elastic net}\end{array}$};&
\node [DataInter] (elastic-comp) {$\begin{array}{c}$$ md_{elast} $$\end{array}$};&
&
\\
%Row 5
&
&
&
&
\node [Function] (comp) {$\begin{array}{c}\text{compare}\end{array}$};&
&
\node [DataIO] (right_output_comp) {$\begin{array}{c}\text{save: best modesl}\end{array}$};\\
%Row 6
&
&
&
&
&
&
\\
};

% XDSM process chains
{ [start chain=process]
 \begin{pgfonlayer}{process} 
\chainin (theta);
\chainin (linear) [join=by ProcessHVA];
\end{pgfonlayer}
}{ [start chain=process]
 \begin{pgfonlayer}{process} 
\chainin (theta);
\chainin (pysindy) [join=by ProcessHVA];
\end{pgfonlayer}
}{ [start chain=process]
 \begin{pgfonlayer}{process} 
\chainin (theta);
\chainin (elastic) [join=by ProcessHVA];
\end{pgfonlayer}
}{ [start chain=process]
 \begin{pgfonlayer}{process} 
\chainin (elastic);
\chainin (comp) [join=by ProcessHVA];
\end{pgfonlayer}
}{ [start chain=process]
 \begin{pgfonlayer}{process} 
\chainin (linear);
\chainin (comp) [join=by ProcessHVA];
\end{pgfonlayer}
}{ [start chain=process]
 \begin{pgfonlayer}{process} 
\chainin (pysindy);
\chainin (comp) [join=by ProcessHVA];
\end{pgfonlayer}
}

\begin{pgfonlayer}{data}
\path
% Horizontal edges
(theta) edge [DataLine] (theta-linear)
(theta) edge [DataLine] (theta-pysindy)
(theta) edge [DataLine] (theta-elastic)
(linear) edge [DataLine] (linear-comp)
(pysindy) edge [DataLine] (pysindy-comp)
(elastic) edge [DataLine] (elastic-comp)
(comp) edge [DataLine] (right_output_comp)
% Vertical edges
(theta-linear) edge [DataLine] (linear)
(theta-pysindy) edge [DataLine] (pysindy)
(theta-elastic) edge [DataLine] (elastic)
(linear-comp) edge [DataLine] (comp)
(pysindy-comp) edge [DataLine] (comp)
(elastic-comp) edge [DataLine] (comp)
(theta) edge [DataLine] (output_theta);
\end{pgfonlayer}

\end{tikzpicture}

    }
    \caption{Data and workflow of modeling the \glsfirst{cpevol}} 
    \label{fig_43}
\end{figure}

Since, the goal is not to explain, how to operate \emph{pySindy} \cite{Brunton2016}, the curious reader is referred to the \emph{pySindy} very extensive online documentation and \cite{Silva2020,Kaptanoglu2022}.
Nevertheless, to understand $\boldsymbol{\Theta}\,(\vec{\beta_{tr}})$ equation \eqref{eq_20} shall be considered. 
In this example, 3 different functions, denoted as $f_i$ in the first row, are employed.
The remaining rows are the output for the chosen $f_i$.
Furthermore, $n$ is the number of samples or the size of $\vec{\beta_{tr} }$, i.e., $n_{\beta,tr} $ and $m$ denotes the number of the features, i.e., the number of the functions $f_i$. \newline

\begin{equation}
    \boldsymbol{\Theta_{exampl(n \times m )}}(\,\vec{\beta_{tr}}) =
    % \renewcommand\arraystretch{3} 
    \renewcommand\arraycolsep{10pt} 
    \begin{bmatrix}
      f_1 = \beta & f_2 = \beta^2 & f_2 = cos(\beta)^2 - exp\,\left(\dfrac{\beta}{-0.856} \right) \\[1.5em]
      1  & 1^2 & cos(1)^2 - exp\,\left(\dfrac{1}{-0.856} \right) \\[1.5em]
      2 &  2^2  & cos(2)^2 - exp\,\left(\dfrac{2}{-0.856} \right) \\[1.5em]
    \end{bmatrix}
    \label{eq_20}
\end{equation}

The actual candidate library matrix $\boldsymbol{\Theta}\,(\vec{\beta_{tr}})$ incorporates  a quadratic polynomial, the inverse $ \frac{1}{\beta}$, the exponential $exp(\beta)$ and 3 frequencies of cos and sin, i.e., $cos(\vec{\beta}_{freq}), \ sin(\vec{\beta}_{freq})$, where $\vec{\beta}_{freq} = [1, \, 2,\, 3]$. 
There are much more complex $\boldsymbol{\Theta}\,(\vec{\beta_{tr}})$  available in \gls{cnmc}, which can be selected if desired. 
Nonetheless, the default $\boldsymbol{\Theta}\,(\vec{\beta_{tr}})$ is chosen as described above.
Once $\boldsymbol{\Theta}\,(\vec{\beta_{tr}})$  is the generated, the system of equations \eqref{eq_21} is solved. 
Note, this is very similar to solving the well-known $\bm A \, \vec{x} =  \vec{y}$ system of equations.
The difference is that the vectors $\vec{x}, \, \vec{y}$ can be vectors in the case of \eqref{eq_21} as well, but in general, they are the matrices $\bm{X} ,\, \bm Y$, respectively. The solution to the matrix $\bm{X}$ is the desired output.
It contains the coefficients which assign importance to the used functions $f_i$. 
The matrix $\bm Y$ contains the targets or the known output for the chosen functions $f_i$.
Comparing $\bm A$ and $ \boldsymbol{\Theta}\,(\vec{\beta_{tr}})$ mathematically, no difference exists.\newline
 
\begin{equation}
    \boldsymbol{\Theta}\,(\vec{\beta_{tr}}) \: \bm X = \bm Y
    \label{eq_21}
\end{equation}

With staying in the \emph{pySindy} environment, the system of equations \eqref{eq_21} is solved by means of the optimizer \emph{SR3}, which is implemented in  \emph{pySindy}.
Details and some advantages of the  \emph{SR3}  optimizer can be found in \cite{SR3}. Nevertheless, two main points shall be stated. 
It is highly unlikely that the $\boldsymbol{\Theta}\,(\vec{\beta_{tr}}),\: \bm X,\, \bm Y$ is going to lead to a well-posed problem, i.e., the number of equations are equal to the number of unknowns and having a unique solution.
In most cases the configuration will be ill-posed, i.e., the number of equations is not equal to the number of unknowns. 
In the latter case, two scenarios are possible, the configuration could result in an over-or under-determined system of equations.\newline 

For an over-determined system, there are more equations than unknowns. 
Thus,  generally, no outcome that satisfies equation \eqref{eq_21} exists.
In order to find a representation that comes close to a solution, an error metric is defined as the objective function for optimization.
There are a lot of error metrics or norms, however, some commonly used \cite{Brunton2019} are given in equations \eqref{eq_22} to \eqref{eq_24}, where $f(x_k)$ are true values of a function and $y_k$ are their corresponding predictions.
The under-determined system has more unknown variables than equations, thus infinitely many solutions exist. 
To find one prominent solution, again, optimization is performed.
Note, for practical application penalization or regularization parameter are exploited as additional constraints within the definition of the optimization problem. 
For more about over- and under-determined systems as well as for deploying optimization for finding a satisfying result the reader is referred to \cite{Brunton2019}.\newline

\begin{equation}
    E_{\infty} = \max_{1<k<n} |f(x_k) -y_k | \quad \text{Maximum Error} \;(l_{\infty})
    \label{eq_22}
\end{equation}

\vspace{0.1cm}
\begin{equation}
    E_{1} = \frac{1}{n} \sum_{k=1}^{n} |f(x_k) -y_k | \quad \text{Mean Absolute Error} \;(l_{1})
    \label{eq_23}
\end{equation}

\vspace{0.1cm}
\begin{equation}
    E_{2} =  \sqrt{\frac{1}{n} \sum_{k=1}^{n} |f(x_k) -y_k |^2 }  \quad \text{Least-squares Error} \;(l_{2})
    \label{eq_24}
\end{equation}
\vspace{0.1cm}

The aim for modeling \emph{CPE} is to receive a regression model, which is sparse, i.e., it is described through a small number of functions $f_i$.
For this to work, the coefficient matrix $\bm X$ must be sparse, i.e., most of its entries are zero. 
Consequently, most of the used functions $f_i$ would be inactive and only a few $f_i$ are actively applied to capture the \emph{CPE} behavior.
The $l_1$ norm as defined in \eqref{eq_23} and the $l_0$ are metrics which promotes sparsity.
In simpler terms, they are leveraged to find only a few important and active functions $f_i$. 
The $l_2$ norm as defined in \eqref{eq_24} is known for its opposite effect, i.e. to assign importance to a high number of $f_i$.
The \emph{SR3} optimizer is a sparsity promoting optimizer, which deploys $l_0$ and $l_1$ regularization.\newline

The second point which shall be mentioned about the \emph{SR3} optimizer is that it can cope with over-and under-determined systems and solves them without any additional input.
One important note regarding the use of \emph{pySindy} is that \emph{pySindy} in this thesis is not used as it is commonly. For modeling the \emph{CPE} only the modules for generating the candidate library matrix $\boldsymbol{\Theta}\,(\vec{\beta_{tr}})$ and the \emph{SR3} optimizer are utilized.\newline 

Going back to the data and workflow in figure \ref{fig_43}, the candidate library matrix $\boldsymbol{\Theta}\,(\vec{\beta_{tr}})$ is generated. 
Furthermore, it also has been explained how it is passed to \emph{pySindy} and how \emph{SR3} is used to find a solution. It can be observed that $\boldsymbol{\Theta}\,(\vec{\beta_{tr}})$ is also passed to a \emph{Linear} and \emph{Elastic net} block. The \emph{Linear} block is used to solve the system of equations \eqref{eq_21} through linear interpolation.
The \emph{Elastic net} solves the same system of equations with the elastic net approach. In this the optimization is penalized with an $l_1$ and $l_2$ norm.
In other words, it combines the Lasso \cite{Lasso, Brunton2019} and Ridge \cite{Brunton2019}, regression respectively. 
The linear and elastic net solvers are invoked from the \emph{Scikit-learn} \cite{scikit-learn} library.\newline 

The next step is not depicted in figure \ref{fig_43}. 
Namely, the linear regression model is built with the full data. For \emph{pySindy} and the elastic net, the models are trained with $90 \%$ of the training data and the remaining $10 \%$ are used to test or validate the model.
For \emph{pySindy} $20$ different models with the linear distributed thresholds starting from $0.1$ and ending at $2.0$ are generated. 
The model which has the least mean absolute error \eqref{eq_23} will be selected as the \emph{pySindy} model.
The mean absolute error of the linear, elastic net and the selected \emph{pySindy} will be compared against each other. 
The one regression model which has the lowest mean absolute error is selected as the final model.\newline

The described process is executed multiple times. 
In 3-dimensions the location of a centroid is given as the coordinates of the 3 axes. 
Since the \emph{CPE} across the 3 different axes can deviate significantly, capturing the entire behavior in one model would require a complex model. 
A complex model, however, is not sparse anymore. 
Thus, a regression model for each of the $K$ labels and for each of the 3 axes is required. 
In total $3 \, K$ regression models are generated. \newline

Finally, \emph{first CNMc} and \gls{cnmc} shall be compared.
First, in \emph{first CNMc} only \emph{pySindy} with a different built-in optimizer. 
Second, the modeling \emph{CPE} was specifically designed for the Lorenz system \eqref{eq_6_Lorenz}. 
Third, \emph{first CNMc} entirely relies on \emph{pySindy}, no linear and elastic models are calculated and used for comparison. 
Fourth, the way \emph{first CNMc} would perform prediction, was by transforming the active $f_i$ with their coefficients to equations such that \emph{SymPy} could be applied. 
The disadvantage is that if $\boldsymbol{\Theta}\,(\vec{\beta_{tr}})$ is changed, modifications for \emph{SymPy} are necessary.
Also, $\boldsymbol{\Theta}\,(\vec{\beta_{tr}})$ can be used for arbitrary defined functions $f_i$, \emph{SymPy} functions, however, are restricted to some predefined functions.
In \gls{cnmc} it is also possible to get the active $f_i$ as equations. 
However, the prediction is obtained with a regular matrix-matrix multiplication as given in equation \eqref{eq_25}. The variables are denoted as the predicted outcome $\bm{\tilde{Y}}$, the testing data for which the prediction is desired $\bm{\Theta_s}$ and the coefficient matrix $\bm X$ from equation \eqref{eq_21}.

\begin{equation}
    \bm{\tilde{Y}} = \bm{\Theta_s} \, \bm X 
    \label{eq_25}
\end{equation}


With leveraging equation \eqref{eq_25} the limitations imposed through \emph{SymPy} are removed.
