

\chapter{Introduction}
\label{chap_1_Intro}
In this work, a tool called \glsfirst{cnmc} is further developed.
The overall goal, in very brief terms, is to generate a model, which is able to 
predict the trajectories of general dynamical systems. The model
shall be capable of predicting the trajectories when a model parameter
value is changed. 
Some basics about dynamical systems are covered in 
subsection \ref{subsec_1_1_1_Principles} and in-depth explanations about \gls{cnmc} are given in
chapter \ref{chap_2_Methodlogy}.\newline 

However, for a short and broad introduction to \gls{cnmc} the workflow depicted in figure \ref{fig_1_CNMC_Workflow} shall be highlighted. 
The input it receives is data of a dynamical system or space state vectors for a range of model parameter values. The two main important outcomes are some accuracy measurements and the predicted trajectory for each desired model parameter value.
Any inexperienced user may only have a look at the predicted trajectories to 
quickly decide visually whether the prediction matches the trained data. Since \gls{cnmc} is written in a modular manner, meaning it can be regarded as 
a black-box function, it can easily be integrated into other existing codes or 
workflows. \newline

\begin{figure}[!h]
  \def\svgwidth{\linewidth}
  \input{2_Figures/1_Task/1_CNMc.pdf_tex}
  \caption{Broad overview: Workflow of \gls{cnmc}}
  \label{fig_1_CNMC_Workflow}
\end{figure}

% ==============================================================================
% ==================== Motivation ==============================================
% ==============================================================================
\section{Motivation}
\label{sec_Motivation}
\gls{cfd} is an 
indispensable technique, when aimed to obtain information about aerodynamic properties, such 
as drag and lift distributions. Modern \gls{cfd} solvers, such as \gls{dlr}'s \emph{TAU}
\cite{Langer2014} often solves 
the \gls{rans} equations to obtain one flow-field. Advanced solvers like \emph{TAU} apply advanced 
mathematical knowledge to speed up calculations and
heavily exploit multiple \glspl{cpu} in an optimized manner. Nevertheless,
depending on the size of the object and accuracy demands or in other terms mesh grid size, the computation often is not economically 
efficient enough. If the object for which a flow field is desired is a full aircraft, then even with a big cluster and making use of symmetry properties of the shape of the airplane, if such exists, the computation of one single
flow field can still easily cost one or even multiple months in computation time. \newline 

In modern science, there is a trend towards relying on \glspl{gpu} instead of \glspl{cpu}. Graphic cards possess much 
more cores than a CPU. However, even with the utilization of \glspl{gpu} and GPU-optimized \gls{cfd} solvers, the computation is still very expensive. Not only in time but also 
in electricity costs. 
Running calculations on a cluster for multiple months is such expensive that wind tunnel measurements can be considered to be the economically more
efficient choice to make.
Regarding accuracy, wind tunnel measurements and \gls{cfd} simulations with state-of-the-art solvers can be considered to be 
equally useful. When using \gls{cfd} solvers, there is one more thing to keep 
in mind. 
Each outcome is only valid for one single set of input parameters.
Within the set of input parameters, the user often is only interested 
in the impact of one parameter, e.g., the angle of attack. Consequently,
wanting to capture the effect of the change of the angle of attack on the flow field, 
multiple \gls{cfd} calculations need to be performed, i.e., for each desired 
angle of attack.
Based on the chosen angle of attack the solver might be able to converge faster to a solution. However, the calculation time 
needs to be added up for each desired angle of attack. 
In terms of time and energy costs, this could again be more expensive than wind-tunnel
measurements. Wind tunnel measurements are difficult to set up, but once a 
configuration is available, measuring flow field properties with it, in general, is known to be faster and easier than running \gls{cfd} simulations.\newline 

% ------------------------------------------------------------------------------
Within the scope of this work, a data-driven tool was developed that allows predictions for dynamic systems. 
In \cite{Max2021} the first version of it showed promising results. 
However, it was dedicated to the solution of one single dynamical system, i.e., the Lorenz system \cite{lorenz1963deterministic}.
Due to the focus on one singular dynamical system, the proposed \glsfirst{cnmc} was not verified for other dynamical systems.
Hence, one of the major goals of this thesis is to enable \gls{cnmc} to be applied to any general dynamical system. 
For this, it is important to state that because of two main reasons \gls{cnmc} was not built upon the first version of \gls{cnmc}, but written from scratch.
First, since the initial version of \gls{cnmc} was designed for only a single dynamic system, extending it to a general \gls{cnmc} was considered more time-consuming than starting fresh.
Second, not all parts of the initial version of \gls{cnmc} could be executed without errors.
The current \gls{cnmc} is therefore developed in a modular manner, i.e., on the one hand, the implementation of any other dynamical system is straightforward. 
To exemplify this, 10 different dynamic systems are available by default, so new dynamic systems can be added analogously.\newline 

The second important aspect for allowing \gls{cnmc} to be utilized in any general dynamical system is the removal of the two limitations.
In the first version of \gls{cnmc} the behavior of the dynamical systems had to be circular as, e.g., the ears of the Lorenz system \cite{lorenz1963deterministic} are.
Next, its dimensionality must be strictly 3-dimensional.
Neither is a general dynamical system is not bound to exhibit a circular motion nor to be 3-dimensional.
By removing these two limitations \gls{cnmc} can be leveraged on any dynamical system.
However, the first version of \gls{cnmc} employed \glsfirst{nmf} as the modal decomposition method. 
The exploited \gls{nmf} algorithm is highly computationally intensive, which makes a universal \gls{cnmc} application economically inefficient.
Therefore, the current \gls{cnmc} has been extended by the option to choose between the \gls{nmf} and the newly implemented \glsfirst{svd}.
The aim is not only that \gls{cnmc} is returning results within an acceptable timescale, but also to ensure that the quality of the modal decomposition remains at least at an equal level.
Proofs for the latter can be found in section \ref{sec_3_3_SVD_NMF}.\newline

With these modifications, the current \gls{cnmc} is now able to be used in any dynamical system within a feasible time frame.
The next addressed issue is the B-spline interpolation. 
It is used in the propagation step of \glsfirst{cnm} \cite{Fernex2021} to smooth the predicted trajectory. 
However, as already noted in \cite{Max2021}, when the number of the clustering centroids $K$ is $K \gtrapprox 15$, the B-spline interpolation embeds oscillations with unacceptable high deviations from the original trajectories.
To resolve this problem, the B-spline interpolation is replaced with linear interpolation. 
By preventing the occurrence of outliers caused by the B-spline interpolation, neither the autocorrelation defined in subsection \ref{subsec_1_1_3_first_CNMc} nor the predicted trajectories are made impractical.
Apart from the main ability of \gls{cnmc} a high number of additional features are available, e.g., the entire pipeline of \gls{cnmc} with all its parameters can be adjusted via one file (\emph{settings.py}), an incorporated log file, storing results at desired steps, the ability to execute multiple dynamical models consequentially and activating and disabling each step of \gls{cnmc}.
The latter is particularly designed for saving computational time.
Also, \gls{cnmc} comes with its own post-processor. 
It is optional to generate and save the plots. 
However, in the case of utilizing this feature, the plots are available as HTML files which, e.g., allow extracting further information about the outcome or rotating and zooming in 3d plots.