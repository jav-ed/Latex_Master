
% ===================================================
% ==================== STATE OF THE ART =============
% ===================================================
\section{State of the art}
\label{sec_1_1_State}

The desire to get fast \gls{cfd} output is not new and also
a data-driven approach is found in the literature. 
This section aims to describe some evolutionary steps of \glsfirst{cnmc}.  Given that this work is built upon the most recent advancements,
they will be explained in particular detail.
Whereas the remaining development stages are briefly 
summarized to mainly clarify the differences and  
mention the reasons why improvements were desired. Since, this topic 
demands some prior knowledge to follow \gls{cnmc}'s workflow and goal, some basic principles about important topics shall be given in their subsection.\newline

The first data-driven approach, which is known to the author, 
is by \cite[]{Kaiser2014} and shall be called \gls{cmm}.
\gls{cnmc} is not directly built upon \gls{cmm} but on the latest version 
of \gls{cnm} and is described in \cite[]{Fernex2021}.
\gls{cnmc} invokes \gls{cnm} many times in order to use 
its outcome for further progress. Therefore, it's evident that only if \gls{cnm} is understood, CNMc's
progress can be followed. \gls{cmm} on the other hand has only a historical link to \gls{cnmc}, but no line of code of \gls{cmm} is invoked in \gls{cnmc}'s workflow. Consequently, \gls{cnm} will be explained in more detail than \gls{cmm}.

\subsection{Principles}
\label{subsec_1_1_1_Principles}
CNM \cite[]{Fernex2021} is a method that uses some machine learning 
techniques, graphs, and probability theory to mirror the behavior of 
complex systems. These complex systems are described often by dynamical systems, which themselves are simply a set of 
differential equations. Differential equations are useful to 
capture motion. Thus, a dynamical system can be seen as a synonym for motion
over time. Some differential equations can be 
solved in closed form, meaning analytically. However, for most of them 
either it is too difficult to obtain an analytical solution or the 
analytical solution is very unhandy or unknown. Unhandy in terms of the solution
being expressed in too many terms. Therefore, in most 
cases, differential equations are solved numerically. Since 
the purpose of \gls{cnm} is not to be only used for analytically 
solvable equations, a numerical ordinary differential integrator 
is used. \newline 

The default solver is \emph{SciPy}'s \emph{RK45} solver.
It is a widely deployed solver and can also be applied to 
chaotic systems for integration 
over a certain amount of time.
Another option for solving chaotic \gls{ode}s is 
\emph{LSODA}. The developers of \emph{pySindy} \cite{Silva2020, Kaptanoglu2022} 
state on their homepage \cite{pysindy_Home} that 
\emph{LSODA} even outperforms the default \emph{RK45} when it comes to chaotic dynamical systems. The reasons why for \gls{cnmc} still \emph{RK45} was chosen will be given in
section 
\ref{sec_2_2_Data_Gen}. 
It is important to remember that turbulent flows are chaotic.
This is the main reason why in this work \gls{cnmc}, has been designed to handle not only general dynamical systems but also general chaotic attractors.
Other well-known instances where chaos is found are, e.g., the weather, the 
motion of planets and also the financial market is believed to be chaotic.
For more places, where chaos is found the reader is referred to \cite{Argyris2017}.\newline 

Note that \gls{cnmc} is designed for all kinds of dynamical systems, it is not restricted to linear, nonlinear or chaotic systems. 
Therefore, chaotic systems shall be recorded to be only one application example of \gls{cnmc}.
However, because chaotic attractors were primarily exploited in the context of the performed investigations in this work, a slightly lengthier introduction to chaotic systems is provided in the appendix \ref{ch_Ap_Chaotic}.
Two terms that will be used extensively over this entire thesis are called model parameter value $\beta$ and a range of model parameter values $\vec{\beta}$. A regular differential equation can be expressed as 
in equation \eqref{eq_1_0_DGL}, where $F$ is denoted as the function which describes the dynamical system. 
The vector $\vec{x}(t)$ is the state vector. 
The form in which differential equations are viewed in this work is given in equation \eqref{eq_1_1_MPV}.

\begin{equation}
    F = \dot{\vec{x}}(t) = \frac{\vec{x}(t)}{dt} = f(\vec{x}(t))
    \label{eq_1_0_DGL}
\end{equation}
\begin{equation}
    F_{\gls{cnmc}} = \left(\dot{\vec{x}}(t), \, \vec{\beta} \right) = 
    \left( \frac{\vec{x}(t)}{dt}, \, \vec{\beta} \right) =
     f(\vec{x}(t), \, \vec{\beta} )
    \label{eq_1_1_MPV}
\end{equation}

Note the vector $\vec{\beta}$ indicates a range of model parameter values,  i.e., the differential equation is solved for each model parameter value $\beta$ separately.
The model parameter value $\beta$ is a constant and does not depend on the time, but rather it is a user-defined value. 
In other terms, it remains unchanged over the entire timeline for which the dynamical system is solved.
The difference between $F$ and $F_{\gls{cnmc}}$ is that $F$ is the differential equation for only one $\beta$, while $F_{\gls{cnmc}}$ can be considered as the same differential equation, however, solved, for a range of individual $\beta$ values. 
The subscript \gls{cnmc} stresses that fact that \gls{cnmc} is performed for a range of model parameter values $\vec{\beta}$.
Some dynamical systems, which will be used for \gls{cnmc}'s validation can be found in section \ref{sec_2_2_Data_Gen}. They are written as a set of differential equations in the $\beta$ dependent form.
Even a tiny change in $\beta$ can result in the emergence of an entirely different trajectory. \newline
% The behavior could exhibit such strong alterations, such 
% that one might believe to require new underlying differential equations. 
% These heavy transitions are called bifurcations.\newline

% Although bifurcations 
% cause the trajectory to vary seemingly arbitrary, 
% there exist canonical bifurcation types. Explanations for  
% deriving their equations and visualization are well covered in literature
% and can be found, 
% e.g., in \cite{Argyris2017,Kutz2022,Strogatz2019}. Although a detailed coverage of bifurcations is not feasible within the scope of this thesis, the method of how the trajectory is changed in such a significant way shall be outlined.
% Namely, bifurcations can replace, remove and generate new attractors, e.g., the above introduced fix-point, limit cycle and torus attractor. 
% Bifurcations were mentioned here only for the sake of completeness. Indeed, one of the final goals for \gls{cnmc} is the extension to handle bifurcations. However, the latter is not part of this thesis.\newline 

In summary, the following key aspects can be concluded. The reason why \gls{cnmc} in future releases is believed to be able to manage real \gls{cfd} fluid flow data and make predictions for unknown model parameter values $\beta$ is that turbulent flows are chaotic. Thus, allowing \gls{cnmc} to work with chaotic attractors in the course of this thesis is considered to be the first step toward predicting entire flow fields. 
% The second point is that there is no real unified definition of chaos, but there are some aspects that are more prevalent in the literature.