\appendix
\chapter{Further implemented dynamical systems}
\label{ch_Ap_Dyna}
\begin{enumerate}
    \item \textbf{Chen} \cite{Chen1999}: 
    \begin{equation}
        \label{eq_8_Chen}
        \begin{aligned}
            \dot x &= a\, (y - x) \\
            \dot y &= x \,(\beta - a) - xz + \beta y \\
            \dot z &= x y -b z
        \end{aligned}
    \end{equation}

    \item \textbf{Lu} \cite{Lu2002}:
    \begin{equation}
        \label{eq_9_Lu}
        \begin{aligned}
            \dot x &= a \, (y -x) \\
            \dot y &= \beta y -x z  \\
            \dot z &= x y - b z
        \end{aligned}
    \end{equation}

    \item \textbf{Van der Pol} \cite{VanderPol}:
    \begin{equation}
        \label{eq_14_VDP}
        \begin{aligned}
            \dot x &= y \\
            \dot y &= y \beta\,(1-x^2) -x
        \end{aligned}
    \end{equation}

\end{enumerate}

\chapter{Some basics about chaotic systems}
\label{ch_Ap_Chaotic}
Since 
Chaotic systems are the height 
of intricacy when considering dynamical systems. 
The reason why the term intricacy was chosen
instead of complexity is that chaotic systems can be, but are not necessarily 
complex. For the relation between complex and 
chaotic the reader is referred to \cite{Rickles2007}. 
The mentioned intricacy of chaotic systems shall be explained by 
reviewing two reasons. First, 
chaotic systems are sensitive to their initial conditions. 
To understand this, imagine we want to solve an \gls{ode}. In order to solve any
differential 
equation, the initial condition or starting state must be known. Meaning, that the 
solution to the \gls{ode} at the very first initial step, from where the 
remaining interval is solved, must be identified beforehand. 
One might believe, a starting point, which is not guessed unreasonably off, 
should suffice to infer the system's future dynamics.\newline

This is 
an educated attempt, however, it is not true for systems that exhibit
sensitivity to initial conditions. These systems amplify any 
perturbation or deviation exponentially 
as time increases. From this it can be concluded
that even in case the initial value would be accurate to, e.g., 10 decimal places,
still after some time, the outcome can not be trusted anymore. 
Visually 
this can be comprehended by thinking of initial conditions
as locations in space. Let us picture two points with two initial conditions
that are selected to be next to each other. Only by zooming in multiple times, 
a small spatial deviation should be perceivable. 
As the time changes, the points will leave the location defined through the initial condition. \newline 


With 
chaotic systems in mind, both initially neighboring 
points will diverge exponentially fast from each other.
As a consequence of the initial condition not being
known with infinite precision, the initial microscopic
errors become macroscopic with increasing time. Microscopic mistakes 
might be considered to be imperceptible and thus have no impact 
on the outcome, which would be worth to be mentioned.
Macroscopic mistakes on the other hand are visible. Depending on 
accuracy demands solutions might be or might not be accepted.
However, as time continues further, the results eventually 
will become completely unusable and diverge from the actual output on a macroscopic scale.\newline 


The second reason, why chaotic systems are very difficult 
to cope with, is the lack of a clear definition. It can be 
argued that even visually, it is not always possible to
unambiguously identify a chaotic system. The idea 
is that at some time step, a chaotic system appears to 
be evolving randomly over time. The question then arises,
how is someone supposed to distinguish between something which 
is indeed evolving randomly and something which only appears 
to be random. The follow-up question most likely is going to be, 
what is the difference between chaos and randomness, or 
even if there is a difference. \newline 

Maybe randomness itself is only 
a lack of knowledge, e.g., the movement of gas particles 
can be considered to be chaotic or random. If the 
velocity and spatial position of each molecule are 
trackable, the concept of temperature is made 
redundant. Gibbs only invented the concept of temperature 
in order to be able to make some qualitative statements 
about a system \cite{Argyris2017}.
A system that can not be described microscopically.
Here the question arises if the movement of the molecules 
would be random, how is it possible that every time 
some amount of heat is introduced into a system, the temperature
changes in one direction. If a random microscale system 
always tends to go in one direction within a macroscale view,  
a clear definition of randomness is required. \newline

Laplace once said if the initial condition
(space and velocity) of each atom would be known,  
the entire future 
could be calculated. In other words, if a system is 
build on equations, which is a deterministic way 
to describe an event, the outcome should just 
depend on the values of the variables. 
Thus, the future, for as long as it is desired could be predicted 
or computed exactly. To briefly summarize this conversion, 
Albert Einstein once remarked that God would not play dice. Nils 
Bohr replied that it 
would be presumptuous of us human beings to prescribe to the Almighty 
how he is to take his decisions. A more in-depth introduction to 
this subject is provided by \cite{Argyris2017}. 
Nevertheless, by doing literature research, one way to 
visually distinguish between
randomness and chaos was found \cite{Boeing2016}. 
Yet, in \cite{Boeing2016} the method was only 
deployed on a logistic map. Hence, further research 
is required here. \newline

As explained, a clear definition of chaos does not exist. 
However, some parts of definitions do occur regularly, e.g., 
the already mentioned \glsfirst{sdic}. Other definition parts are the following: Chaotic 
motion is \textbf{aperiodic} and based on a \textbf{deterministic} system.
An aperiodic system is not repeating any 
previous \textbf{trajectory} and a deterministic system is 
described by governing equations. A trajectory is the evolution 
of a dynamical system over time. For instance, a dynamical system 
consisting of 3 variables is denoted as a 3-dimensional dynamical system.
Each of the variables has its own representation axis.
Assuming these 
3 variables capture space, motion in the x-,y- and z-direction 
is possible. For each point in a defined time range, there is one set of x, y and z values, which fully describes the output of the dynamical system or the position at a chosen time point. 
Simply put, the trajectory is the movement 
or change of the variables of the differential equation over time. Usually, the 
trajectory is displayed in the phase space, i.e., the axis represents the state or values of the variables of a dynamical system. An example can be observed in section \ref{subsec_1_1_3_first_CNMc}. \newline


One misconception which is often believed \cite{Taylor2010} 
and found, e.g., in
Wikipedia \cite{Wiki_Chaos} is that
strange attractors would only appear as a consequence of 
chaos. Yet, Grebogi et al. \cite{Grebogi1984} proved
otherwise. According to 
\cite{Boeing2016,Taylor2010} strange attractors exhibit 
self-similarity. This can be understood visually by imaging any shape 
of a trajectory. Now by zooming in or out, the exact same shape 
is found again. The amount of zooming in or out and consequently 
changing the view scale, will not change the perceived 
shape of the trajectory. Self-similarity happens to be
one of the fundamental properties of a geometry 
in order to be called a fractal \cite{Taylor2010}. 
In case one believes,
strange attractors would always be chaotic and knows that by definition strange attractors phase 
space is self-similar, then 
something further misleading is concluded.
Namely, if a geometry is turned out not only
to be self-similar but also to be a fractal, this 
would demand interpreting every fractal to be
chaotic. \newline 

To refute this, consider the Gophy 
attractor \cite{Grebogi1984}. 
It exhibits the described self-similarity,  
moreover, it is a fractal, and it is also a 
strange attractor. However, the Gophy 
attractor is not chaotic. The reason is found, when 
calculating the Lyapunov exponent, which is negative
\cite{Taylor2010}. Latter tells us that two neighboring 
trajectories are not separating exponentially fast 
from each other. Thus, it does not obey the 
sensitive dependence 
of initial conditions requirement and is 
regarded to be non-chaotic. The key messages are 
that a chaotic attractor surely is a strange 
attractor and a strange attractor is not necessarily 
chaotic. A strange attractor refers to a fractal
geometry in which chaotic behavior may
or may not exist \cite{Taylor2010}. 
Having acquired the knowledge that strange attractors 
can occur in chaotic systems and form a fractal, 
one might infer another question. If a chaotic 
strange attractor always generates a geometry, which 
stays constant when scaled, can chaos be 
regarded to be random?\newline 


This question will not be discussed in detail here, but for the sake of completeness, the 3 known types of nonstrange attractors 
shall be mentioned. These are
the fixed point attractor, the limit cycle attractor, and the
torus attractor \cite{Taylor2010}. 
A fixed point attractor is one point in the phase space, which attracts or pulls nearby trajectories to itself. 
Inside the fix-point attractor, there is no motion, meaning 
the derivative of the differential equation is zero. 
In simpler words,
once the trajectory runs into a fix-point, the trajectory ends there. 
This is because no change over time can be found here. 
A limit cycle can be expressed as an endlessly repeating loop, e.g. in the shape of a circle. 
The trajectory can start at
any given initial condition, still, it can go through a place in the phase space, from where the trajectory is continued as an infinitely
repeating loop. 
For a visualization of the latter and the tours, as well more 
detail the reader is referred to \cite{Argyris2017, Kutz2022, Strogatz2019, Taylor2010}.