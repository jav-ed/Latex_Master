\documentclass[a4paper, 11pt]{scrreprt}

%Region
\usepackage[utf8]{inputenc}
\usepackage{amsmath}
\usepackage{amsfonts}
\usepackage{amssymb}

% To get some features - page styling
\usepackage{fancyhdr}
\usepackage[left= 3.5cm,right = 2cm, bottom = 3 cm]{geometry}

% Clickable Ref and fast navigation through table of
% recognition of a table of content
% see: https://tex.stackexchange.com/questions/50747/options-for-appearance-of-links-in-hyperref
\usepackage{hyperref}
 \hypersetup{
    %  deactivate the border
     colorlinks=true,
    %  color of the refs
     linkcolor=black,
     filecolor=blue,
     citecolor = black,
     urlcolor=cyan,
     }


% graphics library
\usepackage{graphicx}

% both packages required for Inkscape pic imports
\usepackage{transparent}
\usepackage{color}

% bold symbols in math mode
\usepackage{bm}

% %  So the graphics can be found
\graphicspath{{2_Figures/},
               {2_Figures/0_Deco/}, 
               {2_Figures/1_Task/},
               {2_Figures/2_Task/},
               {2_Figures/2_Task/0_N_Study},
               {2_Figures/2_Task/1_Tracking},
               {2_Figures/3_Task/},
               {2_Figures/3_Task/1_Tracking},
               {2_Figures/3_Task/2_Mod_CPE},
               {2_Figures/3_Task/3_SVD_QT},
               {2_Figures/3_Task/4_SLS},
               {2_Figures/3_Task/5_Models},
               }
               

% \graphicspath{{2_Figures/*} }


% Um den Namen des Autos aus Bibtex anzuzeigen
%\usepackage[round, authoryear]{natbib}
\usepackage{natbib}

%Zitationsstil
%\bibliographystyle{unsrt}
%\bibliographystyle{unsrtnat}
% \bibliographystyle{plainnat}
% \bibliographystyle{apacite}
\bibliographystyle{plain} %[plain]


\usepackage{multirow}

% % --------------- Table and Color ---------------
% % for tables which needs more than one page
\usepackage{longtable}

% % Table coloring
% \usepackage[table]{xcolor}
% \definecolor{lightgray}{gray}{0.9}
% \definecolor{lightblue}{rgb}{0.93,0.95,1.0}
% \definecolor{orange}{rgb}{0.93,0.95,1.0}

% %\usepackage{tabu}
% \usepackage[table]{xcolor}

% \definecolor{tableHeader}{RGB}{211, 47, 47}
% \definecolor{tableLineOne}{RGB}{245, 245, 245}
% \definecolor{tableLineTwo}{RGB}{249, 199, 29}
% % --------------- Table and Color- END ---------------

% For splited lists
\usepackage{multicol}

%If you want to generate a caption without being inside of a float object
\usepackage{caption}

% for packe subfigure
\usepackage{subcaption}

% For optimization problems definition's
\usepackage{optidef}

\usepackage{printlen}

% to include full pdf pages
\usepackage{pdfpages}


% % Code-Implementation matlab
% \usepackage[numbered,framed]{matlab-prettifier}

% % Load Tikz
% \usepackage{pgfplots}

% % Abbreviation-Table
% \usepackage{acronym}

\usepackage[acronym, toc]{glossaries}
\makeglossaries

% \usepackage{mfirstuc}
% % ---------------------- XDSM --------------------------

\usepackage{tikz}

% % Optional packages such as sfmath set through python interface
\usepackage{sfmath}

\usepackage{blindtext, rotating}


% % Define the set of TikZ packages to be included in the architecture diagram document
\usetikzlibrary{arrows,chains,positioning,scopes,shapes.geometric,shapes.misc,shadows}

% % Set the border around all of the architecture diagrams to be tight to the diagrams themselves
% % (i.e. no longer need to tinker with page size parameters)
% \usepackage[active,tightpage]{preview}
% \PreviewEnvironment{tikzpicture}
% \setlength{\PreviewBorder}{5pt}


% % ---------------------- XDSM  END --------------------------
% to automatically ensure floats do not go into the next section.
\usepackage[section]{placeins}



% --------------- page header ---------------
\pagestyle{fancy}
\fancyhf{}

% on the left side - section name
\fancyhead[L]{\rightmark}

% pagenumber
\fancyhead[R]{\thepage}


% This sets the header line thickness to 0pt.
\renewcommand{\headrulewidth}{0pt}
%--------------- Page Header END ---------------


% Shortcuts - see:
% https://github.com/James-Yu/LaTeX-Workshop/wiki/Snippets#Font-commands

% For optimization
\DeclareMathOperator*{\argmin}{argmin} 
%Endregion

% % ------------------------ BEGIN DOCUMENT ------------------------------------
\begin{document}


% % % ---------------- Deco Stuff  ------------------------------

    % \includegraphics[width=0.42\textwidth]{./2_Figures/TUBraunschweig_4C.pdf} & 

    \begin{center}
      \begin{tabular}{p{\textwidth}}
      
      \begin{minipage}{\textwidth}
      % \centering
      \includegraphics[width=0.4\textwidth]{./2_Figures/TUBraunschweig_4C.pdf}
      \end{minipage}
      % \begin{minipage}{0.5\textwidth}
      % \centering
      % \includegraphics[width=0.5\textwidth]{./2_Figures/0_Deco/dlr_Logo.jpeg}
      % \end{minipage}
      
      
      \vspace{1cm}  
      
      \\
      
      \begin{center}
      \large{\textsc{
        Master thesis number: 486\\
      }}
      \end{center}

      \begin{center}
      \LARGE{\textsc{
        Flow predictions using control-oriented cluster-based network modeling\\
      }}
      \end{center}
      
      \\
      
      
      \begin{center}
      \large{Technische Universität Braunschweig \\
      Institute of Fluid Mechanics 
      }
      \end{center}
      
      
      \begin{center}
      \textbf{\Large{Master Thesis}}
      \end{center}
 
      
      \begin{center}
      written by
      \end{center}
      
      \begin{center}
      \large{\textbf{Javed Arshad Butt}} \\
      
      \large{5027847} \\
      \end{center}
      
      \begin{center}
      \large{born on 20.05.1996 in Gujrat}
      \end{center}
      
      \vspace{3cm}  
      \begin{center}
      \begin{tabular}{lll}
      \textbf{Submission date:} & & 29.04.2022\\
      \textbf{Supervisor :} & & Dr. Richard Semaan \\
      \textbf{Examiner :} & & Prof. Dr.-Ing. R. Radespiel\\
      
      
      \end{tabular}
      \end{center}
      
      \end{tabular}
      \end{center}
      %Damit die erste Seite = Deckblatt nicht nummeriert wird.
      \thispagestyle{empty}




      

\chapter*{Acknowledgments}

All praise and thanks to the \textbf{ONE}, Who does neither need my praise nor my thanks. 
To the \textbf{ONE}, Who is independent of everything and everyone, but on Whom everything and everyone depends.

\vspace{1cm}
Thank you, Dr. Semaan - you provided me with the possibility to work on such a compelling and challenging topic. Even though the difficult tasks were not always pleasant, I very much appreciate the opportunity to have worked on these captivating tasks. 
Thank you for the time and effort you invested in this work.
Also, thank you for the weekly English exercises and for explaining to me how to pronounce methodology correctly :D 


% % % % ---------------- Selbstständigkeitserklärung ------------------------
\input{1_Latex_Files/0_Deco/1_erkl.tex}

% % % % ---------------- Aufgabenstellung ------------------------
\includepdf[pages=-]{1_Latex_Files/0_Deco/4_Mast.pdf}
\includepdf[pages=-]{1_Latex_Files/0_Deco/5_Mast.pdf}

% % ------------------------ Abstract ------------------------------------------
\chapter*{Abstract}
In this master thesis, a data-driven modeling technique is proposed. 
It enables making predictions for general dynamic systems for unknown model parameter values or operating conditions.
The tool is denoted as \gls{cnmc}.
The most recent developed version delivered promising results for the chaotic Lorenz system \cite{lorenz1963deterministic}.
Since, the earlier work was restricted to the application of only one dynamical system, with this contribution the first major improvement was to allow \gls{cnmc} to be utilized for any general dynamical system. 
For this, \gls{cnmc} was written from scratch in a modular manner. 
The limitation of the number of the dimension and the shape of the trajectory of the dynamical systems are removed.
Adding a new dynamic system was designed such that it should be as straightforward as possible. 
To affirm this point, 10 dynamic systems, most of which are chaotic systems, are included by default. 
To be able to run \gls{cnmc} on arbitrary dynamic systems in an automated way, a parameter study for the modal decomposition method \gls{nmf} was implemented.
However, since a single \gls{nmf} solution took up to hours, a second option was added, i.e., \gls{svd}. 
With \gls{svd} the most time-consuming task could be brought to a level of seconds.
The improvements introduced, allow \gls{cnmc} to be executed on a general dynamic system on a normal computer in a reasonable time. 
Furthermore, \gls{cnmc} comes with its integrated post-processor in form of HTML files to inspect the generated plots in detail.
All the parameters used in \gls{cnmc} some additional beneficial features can be controlled via one settings file. 


% % ------------------------ LISTS ---------------------------------------------
\tableofcontents

% \clearpage %\cleardoublepage %for openright
\phantomsection
\addcontentsline{toc}{chapter}{\listfigurename}
\listoffigures

% \clearpage %\cleardoublepage %for openright
\phantomsection
\addcontentsline{toc}{chapter}{\listtablename}
\listoftables
% \clearpage %\cleardoublepage %for openright


% % ------------------------ Abbrev --------------------------------------------
% Just required to get all the symbol numbers in
% order to define the symbol numbers for the arabic language
% % abbreviations:
\newacronym{ode}{ODE}{\glstextformat{\textbf{O}}rdinary \glstextformat{\textbf{D}}ifferential \glstextformat{\textbf{E}}quation}

\newacronym{cnm}{CNM}{\glstextformat{\textbf{C}}luster-based \glstextformat{\textbf{N}}etwork \glstextformat{\textbf{M}}odeling}

\newacronym{cnmc}{\glstextformat{\emph{CNMc}}}{\glstextformat{\textbf{c}}ontrol-oriented \glstextformat{\textbf{C}}luster-based \glstextformat{\textbf{N}}etwork \glstextformat{\textbf{M}}odeling}

\newacronym[]{cmm}{CMM}{\glstextformat{\textbf{C}}luster \glstextformat{\textbf{M}}arkov-based \glstextformat{\textbf{M}}odeling}

\newacronym{cfd}{CFD}{\glstextformat{\textbf{C}}omputational \glstextformat{\textbf{F}}luid \glstextformat{\textbf{D}}ynamics}

\newacronym{rans}{RANS}{\glstextformat{\textbf{R}}eynolds \glstextformat{\textbf{A}}veraged \glstextformat{\textbf{N}}avier \glstextformat{\textbf{S}}tockes}

\newacronym{dlr}{DLR}{German Aerospace Center}

\newacronym{gpu}{GPU}{\glstextformat{\textbf{G}}raphics \glstextformat{\textbf{P}}rocessing \glstextformat{\textbf{U}}nit}

\newacronym{cpu}{CPU}{\glstextformat{\textbf{C}}omputer \glstextformat{\textbf{P}}rocessing \glstextformat{\textbf{U}}nit}

\newacronym[]{sdic}{SDIC}{\glstextformat{\textbf{S}}ensitive  \glstextformat{\textbf{D}}ependence on \glstextformat{\textbf{I}}nitial \glstextformat{\textbf{C}}onditions}

\newacronym[]{nmf}{NMF}{\glstextformat{\textbf{N}}on-negative \glstextformat{\textbf{M}}atrix \glstextformat{\textbf{F}}actorization}

\newacronym[]{svd}{SVD}{\glstextformat{\textbf{S}}ingular \glstextformat{\textbf{V}}alue \glstextformat{\textbf{D}}ecomposition}

\newacronym[]{rf}{RF}{\glstextformat{\textbf{R}}andom \glstextformat{\textbf{F}}orest}

\newacronym[]{cpd}{CPD}{\glstextformat{\textbf{C}}luster \glstextformat{\textbf{P}}robability \glstextformat{\textbf{D}}istribution}

\newacronym[]{cpevol}{CPE}{\glstextformat{\textbf{C}}entroid \glstextformat{\textbf{P}}osition \glstextformat{\textbf{E}}volution}


\newacronym[]{dtw}{DTW}{\glstextformat{\textbf{D}}ynamical \glstextformat{\textbf{T}}ime \glstextformat{\textbf{W}}arping}

\newacronym[]{knn}{KNN}{\glstextformat{\textbf{K}-\textbf{N}}earest \glstextformat{\textbf{N}}eighbor}


\printglossary[type=\acronymtype, nonumberlist, title=Abbreviations]

% =====================================================================
% ========================= Main Tasks ================================
% =====================================================================

% % % ---------------- Task 1  ------------------------------


\chapter{Introduction}
\label{chap_1_Intro}
In this work, a tool called \glsfirst{cnmc} is further developed.
The overall goal, in very brief terms, is to generate a model, which is able to 
predict the trajectories of general dynamical systems. The model
shall be capable of predicting the trajectories when a model parameter
value is changed. 
Some basics about dynamical systems are covered in 
subsection \ref{subsec_1_1_1_Principles} and in-depth explanations about \gls{cnmc} are given in
chapter \ref{chap_2_Methodlogy}.\newline 

However, for a short and broad introduction to \gls{cnmc} the workflow depicted in figure \ref{fig_1_CNMC_Workflow} shall be highlighted. 
The input it receives is data of a dynamical system or space state vectors for a range of model parameter values. The two main important outcomes are some accuracy measurements and the predicted trajectory for each desired model parameter value.
Any inexperienced user may only have a look at the predicted trajectories to 
quickly decide visually whether the prediction matches the trained data. Since \gls{cnmc} is written in a modular manner, meaning it can be regarded as 
a black-box function, it can easily be integrated into other existing codes or 
workflows. \newline

\begin{figure}[!h]
  \def\svgwidth{\linewidth}
  \input{2_Figures/1_Task/1_CNMc.pdf_tex}
  \caption{Broad overview: Workflow of \gls{cnmc}}
  \label{fig_1_CNMC_Workflow}
\end{figure}

% ==============================================================================
% ==================== Motivation ==============================================
% ==============================================================================
\section{Motivation}
\label{sec_Motivation}
\gls{cfd} is an 
indispensable technique, when aimed to obtain information about aerodynamic properties, such 
as drag and lift distributions. Modern \gls{cfd} solvers, such as \gls{dlr}'s \emph{TAU}
\cite{Langer2014} often solves 
the \gls{rans} equations to obtain one flow-field. Advanced solvers like \emph{TAU} apply advanced 
mathematical knowledge to speed up calculations and
heavily exploit multiple \glspl{cpu} in an optimized manner. Nevertheless,
depending on the size of the object and accuracy demands or in other terms mesh grid size, the computation often is not economically 
efficient enough. If the object for which a flow field is desired is a full aircraft, then even with a big cluster and making use of symmetry properties of the shape of the airplane, if such exists, the computation of one single
flow field can still easily cost one or even multiple months in computation time. \newline 

In modern science, there is a trend towards relying on \glspl{gpu} instead of \glspl{cpu}. Graphic cards possess much 
more cores than a CPU. However, even with the utilization of \glspl{gpu} and GPU-optimized \gls{cfd} solvers, the computation is still very expensive. Not only in time but also 
in electricity costs. 
Running calculations on a cluster for multiple months is such expensive that wind tunnel measurements can be considered to be the economically more
efficient choice to make.
Regarding accuracy, wind tunnel measurements and \gls{cfd} simulations with state-of-the-art solvers can be considered to be 
equally useful. When using \gls{cfd} solvers, there is one more thing to keep 
in mind. 
Each outcome is only valid for one single set of input parameters.
Within the set of input parameters, the user often is only interested 
in the impact of one parameter, e.g., the angle of attack. Consequently,
wanting to capture the effect of the change of the angle of attack on the flow field, 
multiple \gls{cfd} calculations need to be performed, i.e., for each desired 
angle of attack.
Based on the chosen angle of attack the solver might be able to converge faster to a solution. However, the calculation time 
needs to be added up for each desired angle of attack. 
In terms of time and energy costs, this could again be more expensive than wind-tunnel
measurements. Wind tunnel measurements are difficult to set up, but once a 
configuration is available, measuring flow field properties with it, in general, is known to be faster and easier than running \gls{cfd} simulations.\newline 

% ------------------------------------------------------------------------------
Within the scope of this work, a data-driven tool was developed that allows predictions for dynamic systems. 
In \cite{Max2021} the first version of it showed promising results. 
However, it was dedicated to the solution of one single dynamical system, i.e., the Lorenz system \cite{lorenz1963deterministic}.
Due to the focus on one singular dynamical system, the proposed \glsfirst{cnmc} was not verified for other dynamical systems.
Hence, one of the major goals of this thesis is to enable \gls{cnmc} to be applied to any general dynamical system. 
For this, it is important to state that because of two main reasons \gls{cnmc} was not built upon the first version of \gls{cnmc}, but written from scratch.
First, since the initial version of \gls{cnmc} was designed for only a single dynamic system, extending it to a general \gls{cnmc} was considered more time-consuming than starting fresh.
Second, not all parts of the initial version of \gls{cnmc} could be executed without errors.
The current \gls{cnmc} is therefore developed in a modular manner, i.e., on the one hand, the implementation of any other dynamical system is straightforward. 
To exemplify this, 10 different dynamic systems are available by default, so new dynamic systems can be added analogously.\newline 

The second important aspect for allowing \gls{cnmc} to be utilized in any general dynamical system is the removal of the two limitations.
In the first version of \gls{cnmc} the behavior of the dynamical systems had to be circular as, e.g., the ears of the Lorenz system \cite{lorenz1963deterministic} are.
Next, its dimensionality must be strictly 3-dimensional.
Neither is a general dynamical system is not bound to exhibit a circular motion nor to be 3-dimensional.
By removing these two limitations \gls{cnmc} can be leveraged on any dynamical system.
However, the first version of \gls{cnmc} employed \glsfirst{nmf} as the modal decomposition method. 
The exploited \gls{nmf} algorithm is highly computationally intensive, which makes a universal \gls{cnmc} application economically inefficient.
Therefore, the current \gls{cnmc} has been extended by the option to choose between the \gls{nmf} and the newly implemented \glsfirst{svd}.
The aim is not only that \gls{cnmc} is returning results within an acceptable timescale, but also to ensure that the quality of the modal decomposition remains at least at an equal level.
Proofs for the latter can be found in section \ref{sec_3_3_SVD_NMF}.\newline

With these modifications, the current \gls{cnmc} is now able to be used in any dynamical system within a feasible time frame.
The next addressed issue is the B-spline interpolation. 
It is used in the propagation step of \glsfirst{cnm} \cite{Fernex2021} to smooth the predicted trajectory. 
However, as already noted in \cite{Max2021}, when the number of the clustering centroids $K$ is $K \gtrapprox 15$, the B-spline interpolation embeds oscillations with unacceptable high deviations from the original trajectories.
To resolve this problem, the B-spline interpolation is replaced with linear interpolation. 
By preventing the occurrence of outliers caused by the B-spline interpolation, neither the autocorrelation defined in subsection \ref{subsec_1_1_3_first_CNMc} nor the predicted trajectories are made impractical.
Apart from the main ability of \gls{cnmc} a high number of additional features are available, e.g., the entire pipeline of \gls{cnmc} with all its parameters can be adjusted via one file (\emph{settings.py}), an incorporated log file, storing results at desired steps, the ability to execute multiple dynamical models consequentially and activating and disabling each step of \gls{cnmc}.
The latter is particularly designed for saving computational time.
Also, \gls{cnmc} comes with its own post-processor. 
It is optional to generate and save the plots. 
However, in the case of utilizing this feature, the plots are available as HTML files which, e.g., allow extracting further information about the outcome or rotating and zooming in 3d plots.

% ===================================================
% ==================== STATE OF THE ART =============
% ===================================================
\section{State of the art}
\label{sec_1_1_State}

The desire to get fast \gls{cfd} output is not new and also
a data-driven approach is found in the literature. 
This section aims to describe some evolutionary steps of \glsfirst{cnmc}.  Given that this work is built upon the most recent advancements,
they will be explained in particular detail.
Whereas the remaining development stages are briefly 
summarized to mainly clarify the differences and  
mention the reasons why improvements were desired. Since, this topic 
demands some prior knowledge to follow \gls{cnmc}'s workflow and goal, some basic principles about important topics shall be given in their subsection.\newline

The first data-driven approach, which is known to the author, 
is by \cite[]{Kaiser2014} and shall be called \gls{cmm}.
\gls{cnmc} is not directly built upon \gls{cmm} but on the latest version 
of \gls{cnm} and is described in \cite[]{Fernex2021}.
\gls{cnmc} invokes \gls{cnm} many times in order to use 
its outcome for further progress. Therefore, it's evident that only if \gls{cnm} is understood, CNMc's
progress can be followed. \gls{cmm} on the other hand has only a historical link to \gls{cnmc}, but no line of code of \gls{cmm} is invoked in \gls{cnmc}'s workflow. Consequently, \gls{cnm} will be explained in more detail than \gls{cmm}.

\subsection{Principles}
\label{subsec_1_1_1_Principles}
CNM \cite[]{Fernex2021} is a method that uses some machine learning 
techniques, graphs, and probability theory to mirror the behavior of 
complex systems. These complex systems are described often by dynamical systems, which themselves are simply a set of 
differential equations. Differential equations are useful to 
capture motion. Thus, a dynamical system can be seen as a synonym for motion
over time. Some differential equations can be 
solved in closed form, meaning analytically. However, for most of them 
either it is too difficult to obtain an analytical solution or the 
analytical solution is very unhandy or unknown. Unhandy in terms of the solution
being expressed in too many terms. Therefore, in most 
cases, differential equations are solved numerically. Since 
the purpose of \gls{cnm} is not to be only used for analytically 
solvable equations, a numerical ordinary differential integrator 
is used. \newline 

The default solver is \emph{SciPy}'s \emph{RK45} solver.
It is a widely deployed solver and can also be applied to 
chaotic systems for integration 
over a certain amount of time.
Another option for solving chaotic \gls{ode}s is 
\emph{LSODA}. The developers of \emph{pySindy} \cite{Silva2020, Kaptanoglu2022} 
state on their homepage \cite{pysindy_Home} that 
\emph{LSODA} even outperforms the default \emph{RK45} when it comes to chaotic dynamical systems. The reasons why for \gls{cnmc} still \emph{RK45} was chosen will be given in
section 
\ref{sec_2_2_Data_Gen}. 
It is important to remember that turbulent flows are chaotic.
This is the main reason why in this work \gls{cnmc}, has been designed to handle not only general dynamical systems but also general chaotic attractors.
Other well-known instances where chaos is found are, e.g., the weather, the 
motion of planets and also the financial market is believed to be chaotic.
For more places, where chaos is found the reader is referred to \cite{Argyris2017}.\newline 

Note that \gls{cnmc} is designed for all kinds of dynamical systems, it is not restricted to linear, nonlinear or chaotic systems. 
Therefore, chaotic systems shall be recorded to be only one application example of \gls{cnmc}.
However, because chaotic attractors were primarily exploited in the context of the performed investigations in this work, a slightly lengthier introduction to chaotic systems is provided in the appendix \ref{ch_Ap_Chaotic}.
Two terms that will be used extensively over this entire thesis are called model parameter value $\beta$ and a range of model parameter values $\vec{\beta}$. A regular differential equation can be expressed as 
in equation \eqref{eq_1_0_DGL}, where $F$ is denoted as the function which describes the dynamical system. 
The vector $\vec{x}(t)$ is the state vector. 
The form in which differential equations are viewed in this work is given in equation \eqref{eq_1_1_MPV}.

\begin{equation}
    F = \dot{\vec{x}}(t) = \frac{\vec{x}(t)}{dt} = f(\vec{x}(t))
    \label{eq_1_0_DGL}
\end{equation}
\begin{equation}
    F_{\gls{cnmc}} = \left(\dot{\vec{x}}(t), \, \vec{\beta} \right) = 
    \left( \frac{\vec{x}(t)}{dt}, \, \vec{\beta} \right) =
     f(\vec{x}(t), \, \vec{\beta} )
    \label{eq_1_1_MPV}
\end{equation}

Note the vector $\vec{\beta}$ indicates a range of model parameter values,  i.e., the differential equation is solved for each model parameter value $\beta$ separately.
The model parameter value $\beta$ is a constant and does not depend on the time, but rather it is a user-defined value. 
In other terms, it remains unchanged over the entire timeline for which the dynamical system is solved.
The difference between $F$ and $F_{\gls{cnmc}}$ is that $F$ is the differential equation for only one $\beta$, while $F_{\gls{cnmc}}$ can be considered as the same differential equation, however, solved, for a range of individual $\beta$ values. 
The subscript \gls{cnmc} stresses that fact that \gls{cnmc} is performed for a range of model parameter values $\vec{\beta}$.
Some dynamical systems, which will be used for \gls{cnmc}'s validation can be found in section \ref{sec_2_2_Data_Gen}. They are written as a set of differential equations in the $\beta$ dependent form.
Even a tiny change in $\beta$ can result in the emergence of an entirely different trajectory. \newline
% The behavior could exhibit such strong alterations, such 
% that one might believe to require new underlying differential equations. 
% These heavy transitions are called bifurcations.\newline

% Although bifurcations 
% cause the trajectory to vary seemingly arbitrary, 
% there exist canonical bifurcation types. Explanations for  
% deriving their equations and visualization are well covered in literature
% and can be found, 
% e.g., in \cite{Argyris2017,Kutz2022,Strogatz2019}. Although a detailed coverage of bifurcations is not feasible within the scope of this thesis, the method of how the trajectory is changed in such a significant way shall be outlined.
% Namely, bifurcations can replace, remove and generate new attractors, e.g., the above introduced fix-point, limit cycle and torus attractor. 
% Bifurcations were mentioned here only for the sake of completeness. Indeed, one of the final goals for \gls{cnmc} is the extension to handle bifurcations. However, the latter is not part of this thesis.\newline 

In summary, the following key aspects can be concluded. The reason why \gls{cnmc} in future releases is believed to be able to manage real \gls{cfd} fluid flow data and make predictions for unknown model parameter values $\beta$ is that turbulent flows are chaotic. Thus, allowing \gls{cnmc} to work with chaotic attractors in the course of this thesis is considered to be the first step toward predicting entire flow fields. 
% The second point is that there is no real unified definition of chaos, but there are some aspects that are more prevalent in the literature.
% =================================================
% ================ Meet \gls{cnm} =======================
% =================================================
\section{Cluster-based Network Modeling (CNM)}
\label{sec_1_1_2_CNM}
In this subsection, the workflow of \gls{cnm} \cite{Fernex2021} will be elaborated, as well as the previous attempt to expand the algorithm to accommodate a range of model parameter values $\vec{\beta}$.
\gls{cnm} \cite{Fernex2021} is the basis on which \gls{cnmc} is built or rather
\gls{cnmc} invokes \gls{cnm} multiple times for one of its preprocessing steps.
CNM can be split up into 4 main tasks, which are 
data collection, clustering, calculating 
transition properties and propagation.
The first step is to collect the data, which can be provided from any dynamic system or numerical simulations. 
In this study, only dynamical systems are investigated.
Once the data for the dynamical system is passed to \gls{cnm}, the data is clustered, e.g., with k-means++ algorithm \cite{Arthur2006}. 
A detailed elaboration about this step is given in section \ref{sec_2_3_Clustering}. \gls{cnm} exploits graph theory for approximating the trajectory as a movement on nodes. 
These nodes are equivalent to the centroids, which are acquired through clustering. 
Next, the motion, i.e., movement from one centroid to another, shall be clarified.\newline

In order to fully describe the motion on the centroids, the time at which 
one centroid is visited is exited, and also the order of movement must be known. 
Note, when saying the motion is on the centroids, that 
means the centroids or characteristic nodes do not move 
at all. The entire approximated motion of the original trajectory
on the nodes is described with the transition 
property matrices $\bm Q$ and $\bm T$.
The matrices $\bm Q$ and $\bm T$ are the transition probability and transition time matrices, respectively. 
$\bm Q$ is used to apply probability theory for predicting the next following most likely centroid. In other words, if 
the current location is at any node $c_i$, 
$\bm Q$ will provide all possible successor centroids 
with their corresponding transition probabilities.
Thus, the motion on the centroids 
through $\bm Q$ is probability-based.
In more detail, the propagation of the motion on the centroids can be described as equation \eqref{eq_34}. 
The variables are denoted as the propagated $\vec{x}(t)$ trajectory, time $t$, centroid positions $\vec{c}_k,\, \vec{c}_j$, the time $t_j$ where centroid $\vec{c}_j$ is left and the transition time $T_{k,j}$ from $\vec{c}_j$ to $\vec{c}_k$  \cite{Fernex2021}.
Furthermore, for the sake of a smooth trajectory, the motion between the centroids is interpolated through a spline interpolation.\newline

\begin{equation}
    \vec{x}(t) = \alpha_{kj} (t) \, \vec{c}_k + [\, 1 - \alpha_{kj} (t)\,] \, \vec{c}_j, \quad \alpha_{kj} (t) = \frac{t-t_j}{T_{k,j}}
    \label{eq_34}
\end{equation}


The $\bm Q$ matrix only contains non-trivial transitions, i.e., 
if after a transition the centroid remains on the same centroid, the transition is not considered to be a real transition in \gls{cnm}. 
This idea 
is an advancement to the original work of Kaiser et al. \cite{Kaiser2014}.
In Kaiser et al. \cite{Kaiser2014} the transition is modeled 
as a Markov model. Markov models enable non-trivial transitions. Consequently,
the diagonals of the resulting non-direct transition matrix $\bm{Q_n}$  
exhibits the highest values. The diagonal elements stand for non-trivial 
transitions which lead to idling on the same centroid 
many times. Such behavior is encountered and described by Kaiser et al. \cite{Kaiser2014}.\newline 


There are 3 more important aspects that come along when 
adhering to Markov models. First, the propagation of motion is done 
by matrix-vector multiplication. In the case of the existence of a 
stationary state, the solution 
will converge to the stationary state, with an increasing number of iterations, where no change with time happens.
A dynamical system can only survive as long as change with time exists.
In cases where no change with respect to time is encountered, equilibrium 
or fixed points are found.
 Now, if a stationary state or fixed point 
exists in the considered dynamical system, the propagation 
will tend to converge to this fixed point. However, the nature of 
Markov models must not necessarily be valid for general dynamical systems. 
Another way to see that is by applying some linear algebra. The 
long-term behavior of the Markov transition matrix can be obtained 
with equation \eqref{eq_3_Infinite}. Here, $l$ is the number 
of iterations to get from one stage to another. Kaiser et al. 
\cite{Kaiser2014} depict in a figure, how the values of  
$\bm{Q_n}$ evolves after $1 \mathrm{e}{+3}$ steps. $\bm{Q_n}$ has 
become more uniform. 

\begin{equation}
    \label{eq_3_Infinite}
    \lim\limits_{l \to \infty} \bm{Q_n}^l
\end{equation}

If the number of steps is increased even further 
and all the rows would have the same probability value, 
$\bm{Q_n}$ would converge to a stationary point. What
also can be concluded from rows being equal is that it does not matter 
from where the dynamical system was started or what its 
initial conditions were. The probability 
to end at one specific state or centroid is constant as 
the number of steps approaches infinity. Following that,
it would violate the sensitive dependency on initial conditions, 
which often is considered to be mandatory for modeling chaotic systems. Moreover, chaotic
systems amplify any perturbation exponentially, whether at time 
$t = 0$ or at time $t>>0$. \newline 

Thus, a stationary transition matrix $\bm{Q_n}$ is prohibited by chaos at any time step.
This can be found to be one of the main reasons, why 
the  \textbf{C}luster \textbf{M}arkov based \textbf{M}odeling (\gls{cmm}) 
often fails to 
predict the trajectory. 
Li et al. \cite{Li2021} summarize this observation 
compactly as after some time the initial condition 
would be forgotten and the asymptotic distribution would be reached.
Further, they stated, that due to this fact, \gls{cmm} would 
not be suited for modeling dynamical systems. 
The second problem which is involved, when deploying 
regular Markov modeling is that the future only depends
on the current state. However, \cite{Fernex2021} has shown
with the latest \gls{cnm} version that incorporating also past 
centroid positions for predicting the next centroid position 
increases the prediction quality. The latter effect is especially 
true when systems are complex.\newline 


However, for multiple consecutive time steps 
the trajectories position still could be assigned to the same 
centroid position (trivial transitions).
Thus, past centroids are those centroids that are found when going 
back in time through only non-trivial transitions. The number of incorporated 
past centroids is given as equation \eqref{eq_5_B_Past}, where $L$ is denoted 
as the model order number. It represents the number of all 
considered centroids, where the current and all the past centroids are included, with which the prediction of the successor centroid 
is made.

\begin{equation}
    B_{past} = L -1
    \label{eq_5_B_Past}
\end{equation}

Furthermore, in \cite{Fernex2021} it is not simply believed that an 
increasing model 
order $L$ would increase the outcome quality in every case. 
Therefore, a study on the number of $L$ and the clusters $K$
was conducted. The results proved that the choice of 
$L$ and $K$ depend on the considered dynamical system.
\newline

The third problem encountered when Markov models are used is 
that the time step must be provided. This time step is used 
to define when a transition is expected. In case 
the time step is too small, some amount of iterations is 
required to transit to the next centroid. Thus, non-trivial 
transitions would occur. In case the time step is too high, 
the intermediate centroids would be missed. Such behavior
would be a coarse approximation of the real dynamics. Visually this can 
be thought of as jumping from one centroid to another while
having skipped one or multiple centroids. The reconstructed 
trajectory could lead to an entirely wrong representation of the 
state-space.
CNM generates the transition time matrix $\bm T$ from data 
and therefore no input from the user is required.\newline

A brief review of how the $\bm Q$ is built shall be provided. 
Since the concept of 
model order, $L$ has been explained, it can be clarified that
it is not always right to call $\bm Q$ and $\bm T$ matrices. 
The latter is only correct, if $L = 1$, otherwise it must be 
denoted as a tensor. $\bm Q$ and $\bm T$ can always be 
referred to as tensors since a tensor incorporates matrices, i.e., a matrix is a tensor of rank 2. 
In order to generate $\bm Q$, 
$L$ must be defined, such that the shape of $\bm Q$ is 
known. The next step is to gather all sequences of clusters 
$c_i$. To understand that, we imagine the following scenario, 
$L = 3$, which means 2 centroids from the past and the 
current one are 
incorporated to predict the next centroid.
Furthermore, imagining that two cluster sequence scenarios were found,
$c_0 \rightarrow c_1 \rightarrow c_2 $ and  $c_5 \rightarrow c_1 \rightarrow c_2 $. 
These cluster sequences tell us that the current centroid is $c_2$ and the remaining centroids belong to the past. 
In order to complete the sequence for $L = 3$, the successor cluster also needs 
to be added, $c_0 \rightarrow c_1 \rightarrow c_2 \rightarrow c_5 $ and $c_5 \rightarrow c_1 \rightarrow c_2 \rightarrow c_4$.
The following step is to calculate the likelihood 
of a transition to a specific successor cluster. This is done with equation \eqref{eq_4_Poss}, where $n_{k, \bm{j}}$
is the amount of complete sequences, where also the successor
is found. The index $j$ is written as a vector in order 
to generalize the equation for $L \ge 1$. It then contains 
all incorporated centroids from the past and the current centroid.
The index $k$ represents the successor centroid ($\bm{j} \rightarrow k$).
Finally, $n_{\bm{j}}$ counts all the matching incomplete sequences.

\begin{equation}
    \label{eq_4_Poss}
     P_{k, \bm j} = \frac{n_{k,\bm{j}}}{n_{\bm{j}}}
\end{equation}

After having collected all the possible complete cluster sequences with their corresponding probabilities $\bm Q$, the transition time tensors $\bm T$ can be inferred from the data.
With that, the residence time on each cluster is known and can be 
used for computing the transition times for every 
single transition. At this stage, it shall be highlighted again, 
CNM approximates its data fully with only two 
matrices or when $L \ge 2$ tensors, $\bm Q$ and $\bm T$. The 
final step is the prorogation following equation \eqref{eq_34}. 
For smoothing the propagation between two centroids the B-spline interpolation 
is applied.

% It can be concluded that one of the major differences between \gls{cnm} and \gls{cmm} is that {cnm} dismissed Markov modeling. 
% Hence, only direct or non-trivial transition are possible.
% Fernex et al. \cite{Fernex2021} improved \cite{Li2021} by 
% rejecting one more property of Markov chains, namely 
% that the future state could be inferred exclusively from the current state. 
% Through the upgrade of \cite{Fernex2021}, incorporating past states for the prediction of future states could be exploited.

\subsection{First version of CNMc}
\label{subsec_1_1_3_first_CNMc}
Apart from  this thesis, there already has been an 
attempt to build \glsfirst{cnmc}. 
The procedure, progress and results of the most recent effort are described in \cite{Max2021}.
Also, in the latter, the main idea was to predict the trajectories
for dynamical systems with a control term or a model parameter value $\beta$. 
In this subsection, a review of 
\cite{Max2021} shall be given with pointing out which parts need to be improved. In addition, some distinctions between the previous version of \gls{cnmc} and the most recent version are named. 
Further applied modifications are provided in chapter \ref{chap_2_Methodlogy}.\newline

To avoid confusion between the \gls{cnmc} version described in this thesis and the prior \gls{cnmc} version, the old version will be referred to as \emph{first CNMc}. 
\emph{First CNMc} starts by defining a range of model parameter values
$\vec{\beta}$. 
It was specifically designed to only be able to make predictions for the Lorenz attractor \cite{lorenz1963deterministic}, which is described with the set of equations \eqref{eq_6_Lorenz} given in section \ref{sec_2_2_Data_Gen}. 
An illustrative trajectory is of the Lorenz system \cite{lorenz1963deterministic} with $\beta = 28$ is depicted in figure \ref{fig_2_Lorenz_Example}.\newline
%
% ==============================================================================
% ============================ PLTS ============================================
% ==============================================================================
\begin{figure}[!h]
    \centering
    \includegraphics[width =\textwidth]
    % In order to insert an eps file - Only_File_Name (Without file extension)
    {2_Figures/1_Task/2_Lorenz.pdf}
    \caption{Illustrative trajectory of the Lorenz attractor \cite{lorenz1963deterministic}, $\beta = 28$}
    \label{fig_2_Lorenz_Example}
\end{figure}
%

Having chosen a range of model parameter values $\vec{\beta}$, the Lorenz system was solved numerically and its solution was supplied to \gls{cnm} in order to run k-means++ on all received trajectories.
% It assigns each data point to a cluster and
% calculates all the $K$ cluster centroids for all provided trajectories.
% Each cluster has an identity that in literature is known as a label, with which it can be accessed.
The centroid label allocation by the k-means+ algorithm is conducted randomly. 
Thus, linking or matching centroid labels from one model parameter value $\beta_i$ to another model parameter value $\beta_j$, where $i \neq j$, is performed in 3 steps.
The first two steps are ordering the $\vec{\beta}$ in ascending
order and transforming the Cartesian coordinate system into a spherical coordinate system. 
With the now available azimuth angle, each centroid is labeled in increasing order of the azimuth angle. 
The third step is to match the centroids across $\vec{\beta}$, i.e., $\beta_i$ with $\beta_j$.
For this purpose, the centroid label from the prior model parameter value
is used as a reference to match its corresponding nearest centroid in the next model parameter value. 
As a result, one label can be assigned to one centroid across the available $\vec{\beta}$.\newline


Firstly, \cite{Max2021} showed that ambiguous regions can 
occur. Here the matching of the centroids across the $\vec{\beta}$ can 
not be trusted anymore.
Secondly, the deployed coordinate transformation is assumed to only work properly in 3 dimensions. There is the possibility to set one 
or two variables to zero in order to use it in two or one dimension, respectively. 
However, it is not known, whether such an artificially decrease of dimensions yields a successful outcome for lower-dimensional (2- and 1-dimensional) dynamical systems.  In the event of a 4-dimensional or even higher dimensional case, the proposed coordinate transformation cannot be used anymore. 
In conclusion, the transformation is only secure to be utilized in 3 dimensions. 
Thirdly, which is also acknowledged by \cite[]{Max2021} is that the 
coordinate transformation forces the dynamical system to have 
a circular-like trajectory, e.g., as the in figure \ref{fig_2_Lorenz_Example} depicted Lorenz system does. 
Since not every dynamical system is forced to have a circular-like trajectory, it is one of the major parts which needs to be improved, when \emph{first CNMc} is meant to be leveraged for all kinds of dynamical systems. 
Neither the number of dimensions nor the shape of the trajectory should matter for a generalized \gls{cnmc}.\newline 


Once the centroids are matched across all the available $\vec{\beta}$ pySINDy \cite{Brunton2016,Silva2020, Kaptanoglu2022} is used
to build a regression model. This regression model serves the purpose 
of capturing all centroid positions of the calculated model parameter
values $\vec{\beta }$ and making predictions for unseen $\vec{\beta}_{unseen}$. 
Next, a preprocessing step is performed on the
transition property tensors $\bm Q$ and $\bm T$. Both are
scaled, such that the risk of a bias is assumed to be reduced. 
Then, on both \glsfirst{nmf} \cite{Lee1999} is
applied. 
Following equation \eqref{eq_5_NMF} \gls{nmf} \cite{Lee1999} returns
two matrices, i.e., $\bm W$ and $\bm H$. 
The matrices exhibit a physically
relevant meaning. $\bm W$ corresponds to a mode collection and $\bm H$ contains 
the weighting factor for each corresponding mode.\newline

\begin{equation}
    \label{eq_5_NMF}
    \bm {A_{i \mu}} \approx \bm A^{\prime}_{i \mu}  = (\bm W  \bm H)_{i \mu}  = \sum_{a = 1}^{r} 
    \bm W_{ia} \bm H_{a \mu}
\end{equation}

The number of modes $r$ depends on the underlying dynamical system.
Firstly, the \gls{nmf} is utilized by deploying optimization. 
The goal is to satisfy the condition that, the deviation between the original matrix and the approximated matrix shall be below a chosen threshold. 
For this purpose, the number of required optimization iterations easily can be 
in the order of $\mathcal{O} (1 \mathrm{e}+7)$. The major drawback here is that such a high number of iterations is computationally very expensive.
Secondly, for \emph{first CNMc} the number of modes $r$ must be known beforehand. 
Since in most cases this demand cannot be fulfilled two issues arise. 
On the one hand, running \gls{nmf} on a single known $r$ can already be considered to be computationally expensive.
On the other hand, conducting a study to find the appropriate $r$ involves even more computational effort. 
Pierzyna \cite[]{Max2021} acknowledges this issue and defined it to be one of the major limitations. \newline


The next step is to generate a regression model with \glsfirst{rf}.
Some introductory words about \gls{rf} are given in subsection \ref{subsec_2_4_2_QT}. 
As illustrated in \cite{Max2021}, \gls{rf} was able to reproduce the training data reasonably well.
However, it faced difficulties to approximate spike-like curves. 
Once the centroid positions and the two transitions property tensors $\bm Q$ and $\bm T$ are known, they are passed to \gls{cnm} to calculate the predicted trajectories. 
For assessing the prediction quality two methods are used, i.e., the autocorrelation and the \glsfirst{cpd}. 
\gls{cpd} outlines the probability of being on one of the $K$ clusters. 
The autocorrelation given in equation \eqref{eq_35} allows comparing two trajectories with a phase-mismatch \cite{protas2015optimal} and it measures how well a point in trajectory correlates with a point that is some time steps ahead.
The variables in equation \eqref{eq_35} are denoted as time lag $\tau$, state space vector $\bm x$, time $t$ and the inner product $(\bm x, \bm y) = \bm x \cdot \bm{y}^T$.  \newline

\begin{equation}
    R(\tau) = \frac{1}{T - \tau} \int\limits_{0}^{T-\tau}\, (\bm{x} (t), \bm{x}(t+ \tau))    dt, \quad \tau \in [\, 0, \, T\,]
    \label{eq_35}
\end{equation}

\emph{First CNMc} proved to work well for the Lorenz system only for the number of centroids up to $K=10$ and small $\beta$. 
Among the points which need to be improved is the method to match the centroids across the chosen $\vec{\beta}$. 
Because of this, two of the major problems occur, i.e., the limitation to 3 dimensions and the behavior of the trajectory must be circular, similar to the Lorenz system \cite{lorenz1963deterministic}. 
These demands are the main obstacles to the application of \emph{first CNMc} to all kinds of dynamical systems. 
The modal decomposition with \gls{nmf} is the most computationally intensive part and should be replaced by a faster alternative.




% % % ---------------- Task 2  ------------------------------
\chapter{Methodology}
\label{chap_2_Methodlogy}
In this chapter, the entire pipeline for designing the proposed
\gls{cnmc} is elaborated. For this purpose, the ideas behind 
the individual processes are explained. 
Results from the step tracking onwards will be presented in chapter \ref{ch_3}.
Having said that, \gls{cnmc} consists of multiple main process steps or stages. 
First, a broad overview of the \gls{cnmc}'s workflow shall be given.
Followed by a detailed explanation for each major operational step. The 
implemented process stages are presented in the same order as they are 
executed in \gls{cnmc}. However, \gls{cnmc} is not forced 
to go through each stage. If the output of some steps is 
already available, the execution of the respective steps can be skipped. \newline

The main idea behind such an implementation is to prevent computing the same task multiple times.
Computational time can be reduced if the output of some \gls{cnmc} steps are available. 
Consequently, it allows users to be flexible in their explorations. 
It could be the case that only one step of \textsc{CNMc} is desired to be examined with different settings or even with newly implemented functions without running the full \gls{cnmc} pipeline. 
Let the one \gls{cnmc} step be denoted as C, then it is possible to skip steps A and B if their output is already calculated and thus available.
Also, the upcoming steps can be skipped or activated depending on the need for their respective outcomes. 
Simply put, the mentioned flexibility enables to load data for A and B and execute only C. Executing follow-up steps or loading their data is also made selectable.
%
%------------------------------- SHIFT FROM INTRODUCTION ----------------------
%
Since the tasks of this thesis required much coding, 
it is important to 
mention the used programming language and the dependencies. 
As for the programming language,
\emph{Python 3} \cite{VanRossum2009} was chosen. For the libraries, only a few important libraries will be mentioned, because the number of used libraries is high. Note, each used module is 
freely available on the net and no licenses are required to be purchased.
\newline 

The important libraries in terms of performing actual calculations are  
\emph{NumPy} \cite{harris2020array}, \emph{SciPy} \cite{2020SciPy-NMeth}, \emph{Scikit-learn} \cite{scikit-learn}, \emph{pySindy} \cite{Silva2020, Kaptanoglu2022}, for multi-dimensional sparse matrix management \emph{sparse} and for plotting only \emph{plotly} \cite{plotly} was deployed. One of the reason why \emph{plotly} is preferred over \emph{Matplotlib} \cite{Hunter:2007} are post-processing capabilities, which now a re available. Note, the previous \emph{\gls{cmm}c} version used \emph{Matplotlib} \cite{Hunter:2007}, which in this work has been fully replaced by \emph{plotly} \cite{plotly}. More reasons why this modification is useful and new implemented post-processing capabilities will be given in the upcoming sections.\newline

For local coding, the author's Linux-Mint-based laptop with the following hardware was deployed: CPU: Intel Core i7-4702MQ \gls{cpu}@ 2.20GHz × 4, RAM: 16GB.
The Institute of fluid dynamics of the Technische Universität Braunschweig 
also supported this work by providing two more powerful computation resources.
The hardware specification will not be mentioned, due to the fact, that all computations and results elaborated in this thesis can be obtained by 
the hardware described above (authors laptop). However, the two provided 
resources shall be mentioned and explained if \gls{cnmc} benefits from 
faster computers. The first bigger machine is called \emph{Buran}, it is a 
powerful Linux-based working station and access to it is directly provided by 
the chair of fluid dynamics. \newline 

The second resource is the high-performance 
computer or cluster available across the Technische Universität Braunschweig 
\emph{Phoenix}. The first step, where the dynamical systems are solved through an \gls{ode} solver 
is written in a parallel manner. This step can if specified in the \emph{settings.py} file, be performed in parallel and thus benefits from 
multiple available cores. However, most implemented \gls{ode}s are solved within 
a few seconds. There are also some dynamical systems implemented whose
ODE solution can take a few minutes. Applying \gls{cnmc} on latter dynamical 
systems results in solving their \gls{ode}s for multiple different model parameter values. Thus, deploying the parallelization can be advised in the latter mentioned time-consuming \gls{ode}s.\newline 

By far the most time-intensive part of the improved \gls{cnmc} is the clustering step. The main computation for this step is done with
{Scikit-learn} \cite{scikit-learn}. It is heavily parallelized and the 
computation time can be reduced drastically when multiple threads are available.
Other than that, \emph{NumPy} and \emph{SciPy} are well-optimized libraries and 
are assumed to benefit from powerful computers. In summary, it shall be stated that a powerful machine is for sure advised when multiple dynamical 
systems with a range of different settings shall be investigated since parallelization is available. Yet executing \gls{cnmc} on a single dynamical system, a regular laptop can be regarded as 
a sufficient tool. 

%------------------------------- SHIFT FROM INTRODUCTION ----------------------

% =====================================================================
% ============= Workflow ==============================================
% =====================================================================
\section{CNMc's data and workflow}
\label{sec_2_1_Workflow}
In this section, the 5 main points that characterize \gls{cnmc} will be discussed. 
Before diving directly into \gls{cnmc}'s workflow some remarks 
are important to be made.
First, \gls{cnmc} is written from scratch, it is not simply an updated version of the described \emph{first CNMc} in subsection
\ref{subsec_1_1_3_first_CNMc}.
Therefore, the workflow described in this section for \gls{cnmc} will not match that of \emph{first CNMc}, e.g., \emph{first CNMc} had no concept of \emph{settings.py} and it was not utilizing \emph{Plotly} \cite{plotly} to facilitate post-processing capabilities.
The reasons for a fresh start were given in subsection \ref{subsec_1_1_3_first_CNMc}.
However, the difficulty of running \emph{first CNMc} and the time required to adjust \emph{first CNMc} such that a generic dynamic system could be utilized were considered more time-consuming than starting from zero. \newline 

Second, the reader is reminded to have the following in mind. 
Although it is called pipeline or workflow, \gls{cnmc} is not obliged to run the whole workflow. With \emph{settings.py} file, which will be explained below, it is possible to run only specific selected tasks. 
The very broad concept of \gls{cnmc} was already provided at the beginning of chapter \ref{chap_1_Intro}.
However, instead of providing data of dynamical systems for different model parameter values, the user defines a so-called \emph{settings.py} file and executes \gls{cnmc}.
The outcome of \gls{cnmc} consists, very broadly, of the predicted trajectories and some accuracy measurements as depicted in figure 
\ref{fig_1_CNMC_Workflow}.
In the following, a more in-depth view shall be given.\newline 


 The extension of \emph{settings.py} is a regular \emph{Python} file. However, it is a dictionary, thus there is no need to acquire and have specific knowledge about \emph{Python}. 
 The syntax of \emph{Python's} dictionary is quite similar to that of the \emph{JSON} dictionary, in that the setting name is supplied within a quote mark 
 and the argument is stated after a colon. In order to understand the main points of \gls{cnmc}, its main data and workflow are depicted \ref{fig_3_Workflow} as an XDSM diagram \cite{Lambe2012}. 
  \newline 
 
 % ============================================
 % ================ 2nd Workflow ==============
 % ============================================
 \begin{sidewaysfigure} [!]
    \hspace*{-2cm} 
     \resizebox{\textwidth}{!}{
     
%%% Preamble Requirements %%%
% \usepackage{geometry}
% \usepackage{amsfonts}
% \usepackage{amsmath}
% \usepackage{amssymb}
% \usepackage{tikz}

% Optional packages such as sfmath set through python interface
% \usepackage{sfmath}

% \usetikzlibrary{arrows,chains,positioning,scopes,shapes.geometric,shapes.misc,shadows}

%%% End Preamble Requirements %%%

\input{/home/jav/Progs/Virt_Env/writing/lib/python3.12/site-packages/pyxdsm/diagram_styles.tex}
\begin{tikzpicture}

\matrix[MatrixSetup]{
%Row 0
\node [DataIO] (output_cnmc) {$\begin{array}{c}\text{settings.py}\end{array}$};&
&
&
&
&
&
&
\\
%Row 1
\node [Optimization] (cnmc) {$\begin{array}{c}\text{CNMc}\end{array}$};&
\node [DataInter] (cnmc-data) {$\begin{array}{c}\text{data st}\end{array}$};&
\node [DataInter] (cnmc-clust) {$\begin{array}{c}\text{clustering st}\end{array}$};&
\node [DataInter] (cnmc-track) {$\begin{array}{c}\text{tracking st}\end{array}$};&
\node [DataInter] (cnmc-model) {$\begin{array}{c}\text{modeling st}\end{array}$};&
\node [DataInter] (cnmc-pred) {$\begin{array}{c}\text{prediction st}\end{array}$};&
&
\\
%Row 2
\node [DataInter] (data-cnmc) {$\begin{array}{c}\text{Informer,} \\ \text{log file}\end{array}$};&
\node [Function] (data) {$\begin{array}{c}\text{Data Generation}\end{array}$};&
&
&
&
&
&
\node [DataIO] (right_output_data) {$\begin{array}{c}\text{SOP}\end{array}$};\\
%Row 3
&
&
\node [Function] (clust) {$\begin{array}{c}\text{Clustering}\end{array}$};&
&
&
&
&
\node [DataIO] (right_output_clust) {$\begin{array}{c}\text{SOP}\end{array}$};\\
%Row 4
&
&
&
\node [Function] (track) {$\begin{array}{c}\text{Tracking}\end{array}$};&
&
&
&
\node [DataIO] (right_output_track) {$\begin{array}{c}\text{SOP}\end{array}$};\\
%Row 5
&
&
&
&
\node [Function] (model) {$\begin{array}{c}\text{Modeling}\end{array}$};&
&
&
\node [DataIO] (right_output_model) {$\begin{array}{c}\text{SOP}\end{array}$};\\
%Row 6
&
&
&
&
&
\node [Function] (pred) {$\begin{array}{c}\text{Prediction}\end{array}$};&
&
\node [DataIO] (right_output_pred) {$\begin{array}{c}\text{SOP}\end{array}$};\\
%Row 7
&
&
&
&
&
&
&
\\
};

% XDSM process chains


\begin{pgfonlayer}{data}
\path
% Horizontal edges
(cnmc) edge [DataLine] (cnmc-data)
(cnmc) edge [DataLine] (cnmc-clust)
(cnmc) edge [DataLine] (cnmc-track)
(cnmc) edge [DataLine] (cnmc-model)
(cnmc) edge [DataLine] (cnmc-pred)
(data) edge [DataLine] (data-cnmc)
(data) edge [DataLine] (right_output_data)
(clust) edge [DataLine] (right_output_clust)
(track) edge [DataLine] (right_output_track)
(model) edge [DataLine] (right_output_model)
(pred) edge [DataLine] (right_output_pred)
% Vertical edges
(cnmc-data) edge [DataLine] (data)
(cnmc-clust) edge [DataLine] (clust)
(cnmc-track) edge [DataLine] (track)
(cnmc-model) edge [DataLine] (model)
(cnmc-pred) edge [DataLine] (pred)
(data-cnmc) edge [DataLine] (cnmc)
(cnmc) edge [DataLine] (output_cnmc);
\end{pgfonlayer}

\end{tikzpicture}

     }
     \caption{\gls{cnmc} general workflow overview}
     \label{fig_3_Workflow}
 \end{sidewaysfigure}

The first action for executing \gls{cnmc} is to define \emph{settings.py}. It contains descriptive information about the entire pipeline, e.g., which dynamical system to use, which model parameters to select for training, which for testing, which method to use for modal decomposition and mode regression. 
To be precise, it contains all the configuration attributes of all the 5 main \gls{cnmc} steps and some other handy extra functions. It is written in 
a very clear way such that settings to the corresponding stages of \gls{cnmc} 
and the extra features can be distinguished at first glance.
First, there are separate dictionaries for each of the 5 steps to ensure that the desired settings are made where they are needed. 
Second, instead of regular line breaks, multiline comment blocks with the stage names in the center are used. 
Third, almost every \emph{settings.py} attribute is explained with comments. 
Fourth, there are some cases, where 
a specific attribute needs to be reused in other steps. 
The user is not required to adapt it manually for all its occurrences, but rather to change it only on the first occasion, where the considered function is defined. 
\emph{Python} will automatically ensure that all remaining steps receive the change correctly. 
Other capabilities implemented in \emph{settings.py} are mentioned when they are actively exploited.
In figure \ref{fig_3_Workflow} it can be observed that after passing \emph{settings.py} a so-called \emph{Informer} and a log file are obtained. 
The \emph{Informer} is a file, which is designed to save all user-defined settings in \emph{settings.py} for each execution of \gls{cnmc}.
Also, here the usability and readability of the output are important and have been formatted accordingly. It proves to be particularly useful when a dynamic system with different settings is to be calculated, e.g., to observe the influence of one or multiple parameters. \newline 

One of the important attributes which 
can be arbitrarily defined by the user in \emph{settings.py} and thus re-found in the \emph{Informer} is the name of the model. 
In \gls{cnmc} multiple dynamical systems are implemented, which can be chosen by simply changing one attribute in \emph{settings.py}. 
Different models could be calculated with the same settings, thus this clear and fast possibility to distinguish between multiple calculations is required. 
The name of the model is not only be saved in the \emph{Informer} but it will 
be used to generate a folder, where all of \gls{cnmc} output for this single 
\gls{cnmc} workflow will be stored. 
The latter should contribute to on the one hand that the \gls{cnmc} models can be easily distinguished from each other and on the other hand that all results of one model are obtained in a structured way.
\newline 

When executing \gls{cnmc} many terminal outputs are displayed. This allows the user to be kept up to date on the current progress on the one hand and to see important results directly on the other. 
In case of unsatisfying results, \gls{cnmc} could be aborted immediately, instead of having to compute the entire workflow. In other words, if a computation expensive \gls{cnmc} task shall be performed, knowing about possible issues in the first steps can 
be regarded as a time-saving mechanism. 
The terminal outputs are formatted to include the date, time, type of message, the message itself and the place in the code where the message can be found. 
The terminal outputs are colored depending on the type of the message, e.g., green is used for successful computations. 
Colored terminal outputs are applied for the sake of readability. 
More relevant outputs can easily be distinguished from others. 
The log file can be considered as a memory since, in it, the terminal outputs are saved.\newline

The stored terminal outputs are in the format as the terminal output described above, except that no coloring is utilized.
An instance, where the log file can be very helpful is the following. Some implemented quality measurements give very significant information about prediction reliability. Comparing different settings in terms of prediction capability would become very challenging if the terminal outputs would be lost whenever the \gls{cnmc} terminal is closed. The described \emph{Informer} and the log file can be beneficial as explained, nevertheless, they are optional.
That is, both come as two of the extra features mentioned above and can be turned off in \emph{settings.py}.\newline 

Once \emph{settings.py} is defined, \gls{cnmc} will filter the provided input, adapt the settings if required and send the corresponding parts to their respective steps. 
The sending of the correct settings is depicted in figure \ref{fig_3_Workflow}, where the abbreviation \emph{st} stands for settings. 
The second abbreviation \emph{SOP} is found for all 5 stages and denotes storing output and plots. All the outcome is stored in a compressed form such that memory can be saved. All the plots are saved as HTML files. There are many reasons to do so, however, to state the most crucial ones. First, the HTML file can be opened on any operating system. 
In other words, it does not matter if Windows, Linux or Mac is used. 
Second, the big difference to an image is that HTML files can be upgraded with, e.g., CSS, JavaScript and PHP functions. 
Each received HTML plot is equipped with some post-processing features, e.g., zooming, panning and taking screenshots of the modified view. When zooming in or out the axes labels are adapted accordingly. Depending on the position of 
the cursor, a panel with the exact coordinates of one point and other information such as the $\beta $ are made visible. \newline 

In the same way that data is stored in a compressed format, all HTML files are generated in such a way that additional resources are not written directly into the HTML file, but a link is used so that the required content is obtained via the Internet.  
Other features associated with HTML plots and which data are saved will be explained in their respective section in this chapter. 
The purpose of \gls{cnmc} is to generate a surrogate model with which predictions can be made for unknown model parameter values ${\beta}$. 
For a revision on important terminology as model parameter value $\beta$
the reader is referred to subsection \ref{subsec_1_1_1_Principles}. 
Usually, in order to obtain a sound predictive model, machine learning methods require a considerable amount of data. Therefore, the \gls{ode} is solved for a set of $\vec{\beta }$. An in-depth explanation for the first is provided in 
section \ref{sec_2_2_Data_Gen}.
The next step is to cluster all the received trajectories deploying kmeans++ \cite{Arthur2006}. Once this has been done, tracking can take be performed.
Here the objective is to keep track of the positions of all the centroids when $\beta$ is changed over the whole range of $\vec{\beta }$.
A more detailed description is given in section \ref{sec_2_3_Tracking}.\newline 


The modeling step is divided into two subtasks, which are not displayed as such in figure \ref{fig_3_Workflow}. The first subtask aims to get a model that yields all positions of all the $K$ centroids for an unseen $\beta_{unseen}$, where an unseen $\beta_{unseen}$ is any $\beta$ that was not used to train the model. In the second subtask, multiple tasks are performed. 
First, the regular \gls{cnm} \cite{Fernex2021} shall be applied to all the tracked clusters from the tracking step. For this purpose, the format of the tracked results is adapted in a way such that \gls{cnm} can be executed without having to modify \gls{cnm} itself. By running \gls{cnm} on the tracked data of all $\vec{\beta }$, the transition property tensors $\bm Q$ and $\bm T$ for all $\vec{\beta }$ are received. \newline 

Second, all the $\bm Q$ and the $\bm T$ tensors are stacked to form $\bm {Q_{stacked}}$ and $\bm {T_{stacked}}$ matrices. 
These stacked matrices are subsequently supplied to one of the two possible implemented modal decomposition methods. 
Third, a regression model for the obtained modes is constructed. 
Clarifications on the modeling stage can be found in section \ref{sec_2_4_Modeling}.\newline

The final step is to make the actual predictions for all provided $\beta_{unseen}$ and allow the operator to draw conclusions about the trustworthiness of the predictions. 
For the trustworthiness, among others, the three quality measurement concepts explained in subsection 
\ref{subsec_1_1_3_first_CNMc} 
are leveraged. Namely, comparing the \gls{cnmc} and \gls{cnm} predicted trajectories by overlaying them directly. The two remaining techniques, which were already applied in regular \gls{cnm} \cite{Fernex2021}, are the \glsfirst{cpd} and the autocorrelation.\newline 

The data and workflow in figure \ref{fig_3_Workflow} do not reveal one additional feature of the implementation of \gls{cnmc}. That is, inside the folder \emph{Inputs} multiple subfolders containing a \emph{settings.py} file, e.g., different dynamical systems, can be inserted to allow a sequential run. In the case of an empty subfolder, \gls{cnmc} will inform the user about that and continue its execution without an error. 
As explained above, each model will have its own folder where the entire output will be stored. 
To switch between the multiple and a single \emph{settings.py} version, the {settings.py} file outside the \emph{Inputs} folder needs to be modified. The argument for that is \emph{multiple\_Settings}.\newline 

Finally, one more extra feature shall be mentioned. After having computed expensive models, it is not desired to overwrite the log file or any other output. 
To prevent such unwanted events, it is possible to leverage the overwriting attribute in \emph{settings.py}. If overwriting is disabled, \gls{cnmc} would verify whether a folder with the specified model name already exists. 
In the positive case, \gls{cnmc} would initially only propose an alternative model name. Only if the suggested model name would not overwrite any existing folders, the suggestion will be accepted as the new model name. 
Both, whether the model name was chosen in \emph{settings.py} as well the new final replaced model name is going to be printed out in the terminal line.\newline 

In summary, the data and workflow of \gls{cnmc} are shown in Figure \ref{fig_3_Workflow} and are sufficient for a broad understanding of the main steps.
However, each of the 5 steps can be invoked individually, without having to run the full pipeline. Through the implementation of \emph{settings.py} \gls{cnmc} is highly flexible. All settings for the steps and the extra features can be managed with \emph{settings.py}. 
A log file containing all terminal outputs as well a summary of chosen settings is stored in a separate file called \emph{Informer} are part of \gls{cnmc}'s tools.


% =====================================================================
% ============= Data generation =======================================
% =====================================================================
\section{Data generation}
\label{sec_2_2_Data_Gen}
In this section, the first main step of the 5 steps shall be explained.
The idea of \gls{cnmc} is to create a surrogate model such that predictions for unseen $\beta_{unseen}$ can be made. 
An unseen model parameter value $\beta_{unseen}$ is defined to be not incorporated in the training data. Generally in machine learning, the more linear independent data is available the higher the trustworthiness of the surrogate model is assumed to be.
Linear independent data is to be described as data which provide new information. 
Imagining any million times a million data matrix $\bm {A_{n\, x\, n}}$, where  $n = 1 \mathrm{e}{+6}$. 
On this big data matrix $\bm A$  a modal decomposition method, e.g., the \glsfirst{svd} \cite{Brunton2019,gerbrands1981relationships}, shall be applied.\newline 

To reconstruct the original matrix $\bm A$ fully with the decomposed matrices only the non-zero modes are required. The number of the non-zero modes $r$ is often much smaller than the dimension of the original matrix, i.e., $r << n$.
If $r << n$, the measurement matrix $\bm A$ contains a high number of linear dependent data. This has the advantage of allowing the original size to be reduced. The disadvantage, however, is that $\bm A$ contains duplicated entries (rows, or columns). For this reason, $\bm A$ includes data parts which do not provide any new information. 
In the case of $r = n$ only meaningful observations are comprised and $\bm A$ has full rank. 
Part of feature engineering is to supply the regression model with beneficial training data and filter out redundant copies. 
The drawback of $r = n$ is observed when the number of representative modes is chosen to be smaller than the full dimension $r < n$. 
Consequently, valuable measurements could be lost. \newline


Moreover, if the dimension $n$ is very large, accuracy demands may make working with matrices unavoidable. 
As a result, more powerful computers are required and the computational time is expected to be increased.
For this work, an attempt is made to represent non-linear differential equations by a surrogate model. 
In addition, trajectories of many $\vec{\beta }$ can be handled quite efficiently. 
Therefore, it attempted to provide sufficient trajectories as training data.
Having said that the data and workflow of this step, i.e., data generation, shall be described. 
The general overview is depicted in figure \ref{fig_4_Data_Gen}. 
Data generation corresponding settings are passed to its step, which invokes the \gls{ode} solver for the range of selected $\vec{\beta}$. 
The trajectories are plotted and, both, all the obtained trajectories $F_(\vec{\beta})$ and their plots are saved. Note that $\vec{\beta}$ indicates that one differential equation is solved for selected $\beta$ values within a range of model parameter values $\vec{\beta}$.\newline 
%
% ==============================================
% ========== Constraint Viol Workflow =========
% ==============================================
\begin{figure} [!h]
    \hspace*{-4cm} 
    \resizebox{1.2\textwidth}{!}{
    
%%% Preamble Requirements %%%
% \usepackage{geometry}
% \usepackage{amsfonts}
% \usepackage{amsmath}
% \usepackage{amssymb}
% \usepackage{tikz}

% Optional packages such as sfmath set through python interface
% \usepackage{sfmath}

% \usetikzlibrary{arrows,chains,positioning,scopes,shapes.geometric,shapes.misc,shadows}

%%% End Preamble Requirements %%%

\input{/home/jav/Progs/Virt_Env/writing/lib/python3.12/site-packages/pyxdsm/diagram_styles.tex}
\begin{tikzpicture}

\matrix[MatrixSetup]{
%Row 0
\node [DataIO] (output_data) {$\begin{array}{c}\text{data settings}\end{array}$};&
&
&
\\
%Row 1
\node [Function] (data) {$\begin{array}{c}\text{Data Generation}\end{array}$};&
\node [DataInter] (data-ode) {$\begin{array}{c}\text{model information,} \\ $$\vec{\beta}$$\end{array}$};&
&
\\
%Row 2
&
\node [Function,stack] (ode) {$\begin{array}{c}$$ODE(\vec{\beta})$$\end{array}$};&
&
\node [DataIO] (right_output_ode) {$\begin{array}{c}\text{save: trajectories,} \\ \text{plots}\end{array}$};\\
%Row 3
&
&
&
\\
};

% XDSM process chains
{ [start chain=process]
 \begin{pgfonlayer}{process} 
\chainin (data);
\chainin (ode) [join=by ProcessHVA];
\end{pgfonlayer}
}

\begin{pgfonlayer}{data}
\path
% Horizontal edges
(data) edge [DataLine] (data-ode)
(ode) edge [DataLine] (right_output_ode)
% Vertical edges
(data-ode) edge [DataLine] (ode)
(data) edge [DataLine] (output_data);
\end{pgfonlayer}

\end{tikzpicture}

    }
    \caption{Data and workflow of the first step: Data generation}
    \label{fig_4_Data_Gen}
\end{figure}
%
%
A detailed description will be given in the following. 
First, in order to run this task, it should be activated in \emph{settings.py}.
Next, the user may change local output paths, define which kind of plots shall be generated, which dynamical model should be employed and provide the range  $\vec{\beta}$.
As for the first point, the operator can select the path where the output of this specific task shall be stored. Note, that this is an optional attribute. Also, although it was only tested on Linux, the library \emph{pathlib} was applied.
Therefore, if the output is stored on a Windows or Mac-based operating system, which uses a different path system, no errors are expected.
\newline

Regarding the types of plots, first, for each type of plot, the user is enabled to define if these plots are desired or not. Second, all the plots are saved as HTML files. Some reasons for that were provided at the beginning of this chapter and others which are important for trajectory are the following. 
With in-depth explorations in mind, the user might want to highlight specific regions in order to get detailed and correct information. 
For trajectories, this can be encountered when e.g., coordinates of some points within a specified region shall be obtained. Here zooming, panning, rotation and a panel that writes out additional information about the current location of the cursor can be helpful tools. The first type of plot is the trajectory itself with the initial condition as a dot in the state-space.\newline 

If desired, arrows pointing in the direction of motion of the trajectory can be included in the plots.
The trajectory, the initial state sphere and the arrows can be made invisible by one click on the legend if desired. The second type of representation is an animated plot, i.e., each trajectory $ F(\beta)$ is available as the animated motion. The final type of display is one plot that contains all $F(\vec{\beta})$ as a sub-figure. 
The latter type of visualization is a very valuable method to see the impact of $\beta$ across the available $\vec{\beta }$ on the trajectories $ F(\vec{\beta})$.
Also, it can be leveraged as fast sanity check technique, i.e., if any $F(\beta )$ is from expectation, this can be determined quickly by looking at the stacked trajectory plots.
\newline

If for presentation HTML files are not desired, clicking on a button will provide a \emph{png} image of the current view state of the trajectory. Note, that the button will not be on the picture. 
Finally, modern software, especially coding environments, understood that staring at white monitors is eye-straining. Consequently, dark working environments are set as default. For this reason, all the mentioned types of plots have a toggle implemented. 
It allows switching between a dark default and a white representation mode.\newline  

For choosing a dynamical system, two possibilities are given. 
On the one hand, one of the 10 incorporated models can be selected by simply selecting a number, which corresponds to an integrated dynamical system. 
On the other hand, a new dynamical system can be implemented. 
This can be achieved without much effort by following the syntax of one of the 10 available models. The main adjustment is done by replacing the \gls{ode}. 
The differential equations of all 10 dynamic systems that can be selected by default are given in equations \eqref{eq_6_Lorenz} to \eqref{eq_13_Insect} and the 3 sets of equations  \eqref{eq_8_Chen} to \eqref{eq_14_VDP} are found the Appendix.
The latter 3  sets of equations are provided in the Appendix because they are not used for validating \gls{cnmc} prediction performance.
Next to the model's name, the reference to the dynamical system can be seen.
The variables $a$ and $b$ are constants. 
Except for the Van der Pol, which is given in the Appendix \ref{ch_Ap_Dyna} as equation \eqref{eq_14_VDP}, all dynamical systems are 3-dimensional.\newline

% ==============================================================================
% ============================ EQUATIONS =======================================
% ==============================================================================
\begin{enumerate}
    \item  \textbf{Lorenz} \cite{lorenz1963deterministic}:
    \begin{equation}
        \label{eq_6_Lorenz}
        \begin{aligned}
            \dot x &= a\, (y - x)  \\
            \dot y &= x\, (\beta - z -y) \\
            \dot z &= x y -\beta z
        \end{aligned}
    \end{equation}

    \item  \textbf{Rössler} \cite{Roessler1976}:
    \begin{equation}
        \label{eq_7_Ross}
        \begin{aligned}
            \dot x &= -y -z \\
            \dot y &= x + ay \\
            \dot z &= b +z \, (x-\beta)\\
        \end{aligned}
    \end{equation}

   

    \item \textbf{Two Scroll} \cite{TwoScroll}:
    \begin{equation}
        \label{eq_9_2_Scroll}
        \begin{aligned}
            \dot x &= \beta \, (y-x) \\
            \dot y &= x z \\
            \dot z &= a - by^4
        \end{aligned}
        \end{equation}

    \item \textbf{Four Wing} \cite{FourWing}:
    \begin{equation}
        \label{eq_10_4_Wing}
        \begin{aligned}
            \dot x &= \beta x +y +yz\\
            \dot y &= yz - xz \\
            \dot z &= a + bxy -z 
        \end{aligned}
    \end{equation}

    \item \textbf{Sprott\_V\_1} \cite{sprott2020we}:
    \begin{equation}
        \label{eq_11_Sprott_V_1}
        \begin{aligned}
            \dot x &= y \\
            \dot y &= -x - sign(z)\,y\\
            \dot z &= y^2 - exp(-x^2) \, \beta
        \end{aligned}
    \end{equation}


    \item \textbf{Tornado} \cite{sprott2020we}:
    \begin{equation}
        \label{eq_12_Tornado}
        \begin{aligned}
            \dot x &= y \, \beta \\
            \dot y &= -x - sign(z)\,y\\
            \dot z &= y^2 - exp(-x^2) 
        \end{aligned}
    \end{equation}


    \item \textbf{Insect} \cite{sprott2020we}:
    \begin{equation}
        \label{eq_13_Insect}
        \begin{aligned}
            \dot x &= y \\\
            \dot y &= -x - sign(z)\,y \, \beta\\
            \dot z &= y^2 - exp(-x^2) 
        \end{aligned}
    \end{equation}


\end{enumerate}
% ==============================================================================
% ============================ EQUATIONS =======================================
% ==============================================================================

Sprott\_V\_1, Tornado and Insect in equations \eqref{eq_11_Sprott_V_1} to \eqref{eq_13_Insect} are not present in the cited reference \cite{sprott2020we} in this expressed form. 
The reason is that the introduced equations are a modification of the chaotic attractor proposed in \cite{sprott2020we}. The curious reader is invited to read \cite{sprott2020we} and to be convinced about the unique properties.
The given names are made up and serve to distinguish them. 
Upon closer inspection, it becomes clear that they differ only in the place where $\beta$ is added. 
All 3 models are highly sensitive to $\beta $, i.e., a small change in $\beta $ results in bifurcations. 
For follow-up improvements of \gls{cnmc}, these 3 systems can be applied as performance benchmarks for bifurcation prediction capability.\newline 

Showing the trajectories of all 10 models with different $\vec{\beta} $ would claim too much many pages. Therefore, for demonstration purposes the 
3 above-mentioned models, i.e., Sprott\_V\_1, Tornado and Insect are 
displayed in figures \ref{fig_5_Sprott} to \ref{fig_11_Insect}.
Figure \ref{fig_5_Sprott} depicts the dynamical system Sprott\_V\_1 \eqref{eq_11_Sprott_V_1} with $\beta =9$. 
Figures \ref{fig_6_Tornado} to \ref{fig_8_Tornado} presents the Tornado 
\eqref{eq_12_Tornado}
with $\beta =16.78$ with 3 different camera perspectives.
Observing these figures, the reader might recognize why the name Tornado was chosen. The final 3 figures \ref{fig_9_Insect} to \ref{fig_11_Insect} display the Insect \eqref{eq_13_Insect} with $\beta =7$ for 3 different perspectives.
Other default models will be displayed in subsection \ref{subsec_3_5_2_Models}, as they were used for performing benchmarks. \newline

% ==============================================================================
% ============================ PLTS ============================================
% ==============================================================================
\begin{figure}[!h]
    \centering
    \includegraphics[width =\textwidth]
    % In order to insert an eps file - Only_File_Name (Without file extension)
    {2_Figures/2_Task/2_Sprott_V1.pdf}
    \caption{Default model: Sprott\_V\_1 \eqref{eq_11_Sprott_V_1} with $\beta =9$}
    \label{fig_5_Sprott}
\end{figure}



\begin{figure}[!h]
    %\vspace{0.5cm}
    \begin{minipage}[h]{0.47\textwidth}
        \centering
        \includegraphics[width =\textwidth]{2_Figures/2_Task/3_Tornado.pdf}
        \caption{Default model: Tornado \eqref{eq_12_Tornado} with $\beta =16.78$, view: 1}
        \label{fig_6_Tornado}    
    \end{minipage}
    \hfill
    \begin{minipage}{0.47\textwidth}
        \centering
        \includegraphics[width =\textwidth]{2_Figures/2_Task/4_Tornado.pdf}
        \caption{Default model: Tornado \eqref{eq_12_Tornado} with $\beta =16.78$, view: 2}
        \label{fig_7_Tornado}    
    \end{minipage}
\end{figure}

\begin{figure}[!h]
    \begin{minipage}[h]{0.47\textwidth}
        \centering
        \includegraphics[width =\textwidth]{2_Figures/2_Task/5_Tornado.pdf}
        \caption{Default model: Tornado \eqref{eq_12_Tornado} with $\beta =16.78$, view: 3}
        \label{fig_8_Tornado}    
    \end{minipage}
    \hfill
    \begin{minipage}{0.47\textwidth}
        \centering
        \includegraphics[width =\textwidth]{2_Figures/2_Task/6_Insect.pdf}
        \caption{Default model: Insect \eqref{eq_13_Insect} with $\beta =7$, view: 1}
        \label{fig_9_Insect}    
    \end{minipage}
\end{figure}

\begin{figure}[!h]
    \begin{minipage}[h]{0.47\textwidth}
        \centering
        \includegraphics[width =\textwidth]{2_Figures/2_Task/7_Insect.pdf}
        \caption{Default model: Insect \eqref{eq_13_Insect} with $\beta =7$, view: 2}
        \label{fig_10_Insect}    
    \end{minipage}
    \hfill
    \begin{minipage}{0.47\textwidth}
        \centering
        \includegraphics[width =\textwidth]{2_Figures/2_Task/8_Insect.pdf}
        \caption{Default model: Insect \eqref{eq_13_Insect} with $\beta =7$, view: 3}
        \label{fig_11_Insect}    
    \end{minipage}
\end{figure} 
% ==============================================================================
% ============================ PLTS ============================================
% ==============================================================================
\newpage
Having selected a dynamical system, the model parameter values for which the system shall be solved must be specified in \emph{settings.py}. 
With the known range $\vec{\beta}$ the problem can be described, as already mentioned in subsection \ref{subsec_1_1_1_Principles}, with equation \eqref{eq_1_1_MPV}.

\begin{equation}
    F_{\gls{cnmc}} = \left(\dot{\vec{x}}(t), \, \vec{\beta} \right) = 
    \left( \frac{\vec{x}(t)}{dt}, \, \vec{\beta} \right) =
     f(\vec{x}(t), \, \vec{\beta} )
     \tag{\ref{eq_1_1_MPV}}
\end{equation}

The solution to \eqref{eq_1_1_MPV} is obtained numerically by applying \emph{SciPy's RK45} \gls{ode} solver. If desired \gls{cnmc} allows completing this task in parallel. Additional notes on executing this task in parallel are given in section \ref{sec_Motivation}. The main reason for relying on \emph{RK45} is that it is commonly known to be a reliable option. 
Also, in \cite{Butt2021} \emph{RK45} was directly compared with \emph{LSODA}. The outcome was that \emph{LSODA} was slightly better, however, the deviation between \emph{RK45's} and \emph{LSODA's} performance was found to be negligible. 
In other words, both solvers fulfilled the accuracy demands. 
Since chaotic systems are known for their \glsfirst{sdic} any deviation, even in the  $\mathcal{O} (1 \mathrm{e}{-15})$, will be amplified approximately exponentially and finally will become unacceptably high.
Therefore, it was tested, whether the \emph{RK45} solver would allow statistical variations during the solution process. 
For this purpose, the Lorenz system \cite{lorenz1963deterministic} was solved multiple times with different time ranges. The outcome is that \emph{RK45} has no built-in statistical variation.
Simply put, the trajectory of the Lorenz system for one constant $\beta $ will not differ when solved multiple times on one computer.\newline


Comparing \emph{first CNMc} and \gls{cnmc} the key takeaways are that \gls{cnmc} has 10 in-built dynamical systems. 
However, desiring to implement a new model is also achieved in a way that is considered relatively straightforward. 
Important settings, such as the model itself, the $\vec{\beta }$, plotting and storing outcome can be managed with the \emph{settings.py}. The plots are generated and stored such that post-processing capabilities are supplied.

% ==============================================================================
% =========================== Clustering =======================================
% ==============================================================================
\section{Clustering}
\label{sec_2_3_Clustering}
In this section, the second step, the clustering of all trajectories $(\vec{\beta})$, is explained. 
The main idea is to represent $F(\vec{\beta})$ through movement on centroids.
The data and workflow of clustering are very similar to the previous step of the data generation. It can be comprehended with figure \ref{fig_12_Clust}.
All settings for this step can be individually configured in \emph{settings.py}.
The $ F(\vec{\beta})$ and cluster-algorithm specific parameters are filtered and provided to the clustering algorithm. The solutions are plotted and both, the plots and the clustered output are saved.\newline 

% ==============================================
% ========== Clustering Workflow ===============
% ==============================================
\begin{figure} [!h]
    \hspace*{-4cm} 
    \resizebox{1.2\textwidth}{!}{
    \input{2_Figures/2_Task/9_Clust.tikz}
    }
    \caption{Data and workflow of the second step: Clustering}
    \label{fig_12_Clust}
\end{figure}
%
%

Data clustering is an unsupervised machine learning technique. 
There are a variety of approaches that may be used for this, e.g., 
k-means, affinity propagation, mean shift, 
spectral clustering and Gaussian mixtures. All the 
methods differ in their use cases, scalability,  
metric or deployed norm and required input parameters. The latter 
is an indicator of customization abilities. Since k-means can be used for very large
data sets and enables easy and fast implementation, k-means is preferred. Furthermore, David Arthur et al. 
\cite{Arthur2006} introduced k-means++, which is known 
to outperform k-means. Therefore, \gls{cnmc} uses k-means++ 
as its default method for data clustering. 
Note, applying k-means++ is not new in \gls{cnmc}, but it was already applied in the regular \gls{cnm} \cite{Fernex2021}.\newline

In order to 
cover the basics of k-means and k-means++, two terms 
should be understood.
Picturing a box with 30 points in it, where 10 are located on the left,  10
in the middle and 10 on the right side of the box. Adhering to such a 
constellation, it is appealing to create 3 groups, one for 
each overall position (left, center and right). Each group would 
contain 10 points. These groups are called clusters and the 
geometrical center of each cluster is called a centroid. 
A similar thought experiment is visually depicted in \cite{Sergei_Visual}.
Considering a dynamical system, the trajectory is retrieved by integrating the \gls{ode} numerically at discrete time steps. 
For each time step the obtained point is described with one x-, y- and z-coordinate. 
Applying the above-mentioned idea on, e.g., the Lorenz system \cite{lorenz1963deterministic}, defined as the set of equations in \eqref{eq_6_Lorenz}, then the resulting centroids can be seen in figure \ref{fig_13_Clust}.
The full domains of the groups or clusters are color-coded in figure \ref{fig_14_Clust}.\newline

\begin{figure}[!h]
    %\vspace{0.5cm}
    \begin{minipage}[h]{0.47\textwidth}
        \centering
        \includegraphics[width =\textwidth]{2_Figures/2_Task/10_Clust.pdf}
        \caption{Centroids of the Lorenz system  \eqref{eq_6_Lorenz} with $\beta =28$}
        \label{fig_13_Clust}    
    \end{minipage}
    \hfill
    \begin{minipage}{0.47\textwidth}
        \centering
        \includegraphics[width =\textwidth]{2_Figures/2_Task/11_Clust.pdf}
        \caption{Cluster domains of the Lorenz system \eqref{eq_6_Lorenz} with $\beta =28$}
        \label{fig_14_Clust}    
    \end{minipage}
\end{figure}


Theoretically, 
the points which are taken to calculate a center could be assigned 
weighting factors. However, this is not done in \gls{cnmc} and therefore only 
be outlined as a side note. After being familiar with the concept of 
clusters and centroids, the actual workflow of k-means shall be explained.
For initializing 
k-means, a number of clusters and an initial guess for the centroid 
positions must be provided. Next, the distance between all the data 
points and the centroids is calculated. The data points closest to a 
centroid are assigned to these respective clusters. In other words, each data point is assigned to that cluster for which 
the corresponding centroid exhibits the smallest distance 
to the considered data point. 
The geometrical mean value for all clusters is subsequently determined for all cluster-associated residents' data points. With the 
new centroid positions, the clustering is 
performed again. \newline 

Calculating the mean of the clustered 
data points (centroids) and performing clustering based on the 
distance between each data point and the centroids 
is done iteratively. The iterative process stops when 
the difference between the prior and current 
centroids position is equal to zero or 
satisfies a given threshold. Other explanations with pseudo-code and  
visualization for better understanding can be found\cite{Frochte2020} 
and \cite{Sergei_Visual}, respectively\newline

% ------------------------------------------------------------------------------
% --------------------- PART 2 -------------------------------------------------
% ------------------------------------------------------------------------------
Mathematically k-means objective can be expressed 
as an optimization problem with the centroid 
position $\bm{\mu_}j$ as the design variable. That is given in equation
\eqref{eq_1_k_means} (extracted from \cite{Frochte2020}), where 
$\bm{\mu_}j$ and  $\mathrm{D}^{\prime}_j$ denote the centroid or 
mean of the \emph{j}th cluster and the data points 
belonging to the \emph{j}th cluster, respectively.
The distance between all the \emph{j}th cluster data points 
and its corresponding \emph{j}th centroid is 
stated as $\mathrm{dist}(\bm{x}_j, \bm{\mu}_j)$.

\begin{equation}
    \label{eq_1_k_means}
    \argmin_{\bm{\mu}_j}\sum_{j=1}^k \; \sum_{\bm{x}_j \in \mathrm{D}^{\prime}_j }
    \mathrm{dist}(\bm{x}_j, \bm \mu_j)
\end{equation}

Usually, the k-means algorithm is deployed with a Euclidean metric 
and equation \eqref{eq_1_k_means} becomes \eqref{eq_2_k_Means_Ly}, which
is known as the Lloyd algorithm \cite{Frochte2020, Lloyd1982}. The
Lloyd algorithm can be understood as the minimization of the variance.
Thus, it is not necessarily true that k-means is equivalent to reducing 
the variance. It is only true when the Euclidean norm is used.

\begin{equation}
    \label{eq_2_k_Means_Ly}
    \argmin_{\bm{\mu}_j}\sum_{j=1}^k \; \sum_{\bm{x}_j \in \mathrm{D}^{\prime}_j }
    \| \bm x_j - \bm \mu_j \|^2
\end{equation}

The clustering algorithm highly depends on the provided 
initial centroids positions. Since in most cases, these 
are guessed, there is no guarantee of a reliable outcome.
Sergei Vassilvitskii, one of the founders of 
k-means++, says in one of his presentations \cite{Sergei_Black_Art},
finding a good set of initial points would be black art.
Arthur et al. \cite{Arthur2006} state,
that the speed and simplicity of k-means would be appealing, not 
its accuracy. There are many natural examples for which 
the algorithm generates arbitrarily bad clusterings \cite{Arthur2006}.\newline 


An alternative or improved version of k-means is the already
mentioned k-means++, which
only differs in the initialization step. Instead of providing 
initial positions for all centroids, just one centroid's 
position is supplied. The remaining are calculated based on 
maximal distances. In concrete, the distance between all 
data points and the existing centroids is computed. The data point 
which exhibits the greatest distance is added to the 
list of collected centroids. This is done until all $k$ 
clusters are generated. A visual depiction of this process 
is given by Sergei Vassilvitskii in \cite{Sergei_Visual}.
Since the outcome of k-means++ is more reliable than 
k-means, k-means++ is deployed in \gls{cnmc}.\newline

After having discussed some basics of k-means++, it shall be 
elaborated on how and why the solution of the dynamical system should be 
clustered. The solution of any dynamical system returns a trajectory.
If the trajectory repeats itself or happens to come close 
to prior trajectories without actually touching them,  
characteristic sections can be found.
Each characteristic section in the phase space is 
captured by a centroid. The movement from one 
centroid to another is supposed to portray the original
trajectory. With a clustering algorithm, these representative 
characteristic locations in the phase space are obtained. 
Since the clusters shall capture an entire trajectory, it is 
evident that the number of clusters is an
essential parameter to choose. Latter fact becomes even 
more clear when recalling that a trajectory can be multi-modal or complex.\newline

In the case of a highly non-linear 
trajectory, it is obvious that many clusters are demanded in
order to minimize the loss of the original
trajectories. The projection of the real trajectory 
to a cluster-based movement can be compared to 
a reduced-order model of the trajectory. In this context, 
it is plausible to refer to the centroids as 
representative characteristic locations. Furthermore, \gls{cnm} and thus, \gls{cnmc}, exploits graph theory. 
Therefore, the centroids can be denoted as nodes or characteristic nodes.\newline

The remaining part of this section will be devoted exclusively to the application of \gls{cnmc}. First, the leveraged kmeans++ algorithm is from the machine learning \emph{Python} library \emph{Scikit-learn} \cite{scikit-learn}. 
Crucial settings, e.g., the number of clusters $K$, the maximal number of iterations, the tolerance as a convergence criterion and the number of different centroid seeds with which k-means is executed.
The operator can decide if the clustering step shall be performed or skipped.
The path for outputs can be modified and generating plots is also optional.
For the clustering stage, there are 4 types of plots available. 
Two types of plots are depicted in figures \ref{fig_13_Clust} and \ref{fig_14_Clust}.
With the generated HTML plots the same features as described in section \ref{sec_2_2_Data_Gen} are available, e.g., receiving more information through pop-up panels and
switching between a dark and white mode. 
 \newline 

The other two types of charts are not displayed here because they are intended to be studied as HTML graphs where the output can be viewed from multiple angles.
The first type shows the clustered output of one system for two different $\beta$ next to each other. 
The centroids are labeled randomly as will be shown in subsection \ref{subsec_2_2_1_Parameter_Study}. 
Consequently, the centroids representing the immediate neighbors across the two separate $\beta $ have separate labels. 
In the second remaining type of HTML graph, the closest centroids across the two different $\beta $  are connected through lines.
Also, in the same way, as it was done for the first step in the data generation an additional HTML file containing all $\vec{\beta } $ charts is generated.
\newline

It can be concluded that the clustering step is performed by employing \emph{Scikit-learn's} kmeans++ implementation, which is well suited for a great number of points. As usual, all important settings can be controlled with \emph{settings.py}.

\subsection{Parameter Study}
\label{subsec_2_2_1_Parameter_Study}
In this subsection, the effects on the clustering step caused by the parameter \emph{n\_init} shall be named. After that, the random labeling of the centroids is to be highlighted.
With the parameter \emph{n\_init} it is possible to define how often the k-means algorithm will be executed with different centroid seeds \cite{scikit-learn}.
For a reliable clustering quality \emph{n\_init} should be chosen high. However, the drawback is that with increasing \emph{n\_init} the calculation time increases unacceptably high. Having chosen \emph{n\_init} too high, the clustering part becomes the new bottleneck of the entire \gls{cnmc} pipeline. \newline

To conduct the parameter study, clustering was performed using the following \emph{n\_init} values: 
$\text{\emph{n\_init}} = \{100,\, 200, \, 400,\, 800,\, 1000, \, 1200, \, 1500 \}$. 
Some results are presented in figures \ref{fig_15} to \ref{fig_20}.
It can be observed that when all the different \emph{n\_init} values are compared, visually no big difference in the placing of the centroids can be perceived. 
A graphical examination is sufficient because even with \emph{n\_init} values that differ by only by the number one ($n_{init,1} - n_{init,2} = 1 $), the centroid positions are still expected to vary slightly. 
Simply put, only deviations on a macroscopic scale, which can be noted visually are searched for. As a conclusion it can be said that $\text{\emph{n\_init}} = 100$ and $\text{\emph{n\_init}} = 1500$ can be considered as an equally valuable clustering outcome. 
Hence, \emph{n\_init} the computational expense can be reduced by deciding on a reasonable small value for \emph{n\_init}.\newline

The second aim of this subsection was to highlight the fact that the centroids are labeled randomly. For this purpose, the depicted figures \ref{fig_15} to \ref{fig_20} shall be examined. Concretely, any of the referred figures can be compared with the remaining figures to be convinced that the labeling is not obeying any evident rule.


% ==============================================================================
% ============================ PLTS ============================================
% ==============================================================================
\begin{figure}[!h]
    \begin{minipage}[h]{0.47\textwidth}
        \centering
        \includegraphics[width =\textwidth]{2_Figures/2_Task/0_N_Study/0_K_100.pdf}
        \caption{Lorenz \eqref{eq_6_Lorenz}, $\beta =28$, $\text{n\_init}= 100$}
        \label{fig_15}    
    \end{minipage}
    \hfill
    \begin{minipage}{0.47\textwidth}
        \centering
        \includegraphics[width =\textwidth]{2_Figures/2_Task/0_N_Study/1_K_200.pdf}
        \caption{Lorenz \eqref{eq_6_Lorenz}, $\beta =28$, $\text{n\_init}= 200$}
        \label{fig_16}    
    \end{minipage}
\end{figure}

\begin{figure}[!h]
    \begin{minipage}[h]{0.47\textwidth}
        \centering
        \includegraphics[width =\textwidth]{2_Figures/2_Task/0_N_Study/2_K_400.pdf}
        \caption{Lorenz \eqref{eq_6_Lorenz}, $\beta =28$, $\text{n\_init}= 400$}
        \label{fig_17}    
    \end{minipage}
    \hfill
    \begin{minipage}{0.47\textwidth}
        \centering
        \includegraphics[width =\textwidth]{2_Figures/2_Task/0_N_Study/3_K_1000.pdf}
        \caption{Lorenz \eqref{eq_6_Lorenz}, $\beta =28$, $\text{n\_init}= 1000$}
        \label{fig_18}    
    \end{minipage}
\end{figure}

\begin{figure}[!h]
    \begin{minipage}[h]{0.47\textwidth}
        \centering
        \includegraphics[width =\textwidth]{2_Figures/2_Task/0_N_Study/4_K_1200.pdf}
        \caption{Lorenz \eqref{eq_6_Lorenz}, $\beta =28$, $\text{n\_init}= 1200$}
        \label{fig_19}    
    \end{minipage}
    \hfill
    \begin{minipage}{0.47\textwidth}
        \centering
        \includegraphics[width =\textwidth]{2_Figures/2_Task/0_N_Study/5_K_1500.pdf}
        \caption{Lorenz \eqref{eq_6_Lorenz}, $\beta =28$, $\text{n\_init}= 1500$}
        \label{fig_20}    
    \end{minipage}
\end{figure} 
% ==============================================================================
% ============================ PLTS ============================================
% ==============================================================================


\section{Tracking}
\label{sec_2_3_Tracking}
In this section, it is the pursuit to explain the third step, tracking, by initially answering the following questions. 
What is tracking, why is it regarded to be complex and why is it important? 
In the subsection \ref{subsec_2_3_1_Tracking_Workflow} the final workflow will be elaborated. Furthermore, a brief discussion on the advancements in tracking of \gls{cnmc} to \emph{first CNMc} shall be given. 
Since the data and workflow of tracking are extensive, for the sake of a better comprehension the steps are visually separated with two horizontal lines in the upcoming subsection. 
The lines introduce new tracking subtasks, which are intended to provide clear guidance to orient readers within the workflow.
Note, the tracking results will be presented in subsection \ref{sec_3_1_Tracking_Results}. \newline

To define the term tracking some explanations from subsection \ref{subsec_1_1_3_first_CNMc} shall be revisited.
Through the clustering step, each centroid is defined with a label.
The label allocation is performed randomly as showed in subsection \ref{subsec_2_2_1_Parameter_Study}.
Thus, matching centroid labels from one model parameter value $\beta_i$ to 
another model parameter value $\beta_j$, where $i \neq j$, becomes an issue.
In order first, to understand the term tracking, figure \ref{fig_21} shall be considered. 
The centroids of the Lorenz system \eqref{eq_6_Lorenz} for two $\beta$ values $\beta_i = 31.333$ in green and $\beta_j = 32.167$ in yellow are plotted next to each other.
The objective is to match each centroid of $\beta_i$ with one corresponding centroid of $\beta _j$.
It is important to understand that the matching must be done across the two $\beta $ values $\beta_i$ and $\beta_j$ and not within the same $\beta$.\newline 

% ==============================================================================
% ============================ PLTS ============================================
% ==============================================================================
\begin{figure}[!h]
    \centering
    \includegraphics[width =0.8\textwidth]{2_Figures/2_Task/1_Tracking/0_Matching.pdf}
    \caption{Unrealistic tracking example for the Lorenz system with $\beta_i=31.333, \, \beta_j=32.167, \, K = 10$}
    \label{fig_21}    
\end{figure}

By inspecting the depicted figure closer it can be observed that each green centroid $\beta_i$ has been connected with a corresponding yellow centroid $\beta_j$ with an orange line. 
The centroids which are connected through the orange lines shall be referred to as \emph{inter} $\beta$ \emph{connected} centroids. 
Determining the \emph{inter} $\beta$ \emph{connected} centroids is the outcome of tracking. Thus, it is matching centroids across different model parameter values $\beta$ based on their corresponding distance to each other. The closer two \emph{inter} $\beta $ centroids are, the more likely they are to be matched. 
The norm for measuring distance can be chosen from 
one of the 23 possible norms defined in \emph{SciPy} \cite{2020SciPy-NMeth}. 
However, the default metric is the euclidean norm which is defined as equation \eqref{eq_16}.\newline

 \begin{equation}
     \label{eq_16}
     d(\vec x, \vec y) = \sqrt[]{\sum_{i=1}^n \left(\vec{x}_i - \vec{y}_i\right)^2}
 \end{equation}

\vspace{0.2cm}
The orange legend on the right side of figure \ref{fig_21} outlines the tracked results.
In this rare and not the general case, the \emph{inter} $\beta$ labeling is straightforward in two ways. First, the closest centroids from $\beta_i$ to $\beta_j$ have the same label. Generally, this is not the case, since the centroid labeling is assigned randomly.
Second, the \emph{inter} $\beta$ centroid positions can be matched easily visually. 
Ambiguous regions, where visually tracking is not possible, are not present.
To help understand what ambiguous regions could look like, figure \ref{fig_22} shall be viewed. It illustrates the outcome of the Lorenz system \eqref{eq_6_Lorenz} with $\beta_i=39.004,\, \beta_j = 39.987$ and with a number of centroids of $K= 50$.
Overall, the tracking seems to be fairly obvious, but paying attention to the centroids in the center, matching the centroids becomes more difficult. 
This is a byproduct of the higher number of centroids $K$. 
With more available centroids, more centroids can fill a small area.
As a consequence, multiple possible reasonable matchings are allowed.
Note, that due to spacing limitations, not all tracking results are listed in the right orange legend of figure \ref{fig_22}. 
The emergence of such ambiguous regions is the main driver why tracking is considered to be complex.\newline

\begin{figure}[!h]
    \centering
    \includegraphics[width =\textwidth]
    {2_Figures/2_Task/1_Tracking/2_Ambiguous_Regions.pdf}
    \caption{Ambiguous regions in the tracking example for the Lorenz system with $\beta_i=39.004,\, \beta_j = 39.987,\, K= 50$}
    \label{fig_22}
\end{figure}


In general, it can be stated that the occurrence of ambiguous regions can be regulated well with the number of centroids $K$. 
$K$ itself depends on the underlying dynamical system. 
Thus, $K$ should be only as high as required to capture the complexity of the dynamical system.
Going above that generates unnecessary many centroids in the state space. 
Each of them increases the risk of enabling ambiguous regions to appear. Consequently, incorrect tracking results can arise.\newline 


In figure \ref{fig_23} a second example of tracked outcome for the Lorenz system \eqref{eq_6_Lorenz} with $\beta_i=30.5,\, \beta_j=31.333, \, K = 10$ is given. 
Here it can be inspected that the immediate \emph{inter} $\beta $ centroid neighbors do not adhere to the same label. Hence, it is representative of a more general encountered case. The latter is only true when the $K$ is chosen in a reasonable magnitude. The reason why centroids are tracked by employing distance measurements is grounded in the following. 
If the clustering parameter \emph{n\_init} is chosen appropriately (see \ref{subsec_2_2_1_Parameter_Study}), the position of the centroids are expected to change only slightly when $\beta $ is changed. 
In simpler terms, a change in $\beta$ should not move a centroid much, if the clustering step was performed satisfactorily in the first place.

\begin{figure}[!h]
    \centering
    \includegraphics[width =0.8\textwidth]{2_Figures/2_Task/1_Tracking/1_Non_Matching.pdf}
    \caption{Realistic tracking example for the Lorenz system with $\beta_i=30.5$ and $\beta_j=31.333$}
    \label{fig_23}    
\end{figure}

The next point is to answer the question, of why tracking is of such importance to \gls{cnmc}. 
The main idea of \gls{cnmc} is to approximate dynamical systems and allow prediction trajectories for unseen $\beta_{unseen}$. 
The motion, i.e., the trajectory, is replicated by the movement from one centroid to another centroid. Now, if the centroids are labeled wrong, the imitation of the motion is wrong as well. 
For instance, the considered dynamical system is only one movement from left to right. 
For instance, imagining a dynamical system, where the trajectory is comprised of solely one movement. 
Namely, moving from left to right.
Following that, labeling the left centroid $c_l$ to be the right centroid $c_r$, would fully change the direction of the movement, i.e. $(c_l \rightarrow c_r) \neq (c_r \rightarrow c_l)$.
In one sentence, the proper tracking is vital because otherwise \gls{cnmc} cannot provide reliably predicted trajectories. \newline

\subsection{Tracking workflow}
\label{subsec_2_3_1_Tracking_Workflow}
In this subsection, the main objective is to go through the designed tracking workflow. As side remarks, other attempted approaches to tracking and the reason for their failure will be mentioned briefly.\newline


To recapitulate on the term tracking, it is a method to match centroids across a set of different model parameter values $\vec{\beta}$ based on their respective distances. One obvious method for handling this task could be \gls{knn}. However, having implemented it, undesired results were encountered. 
Namely, one centroid label could be assigned to multiple centroids within the same $\beta$. The demand for tracking, on the hand, is that, e.g., with $K=10$, each of the 10 available labels is found exactly once for one $\beta $. 
Therefore, it can be stated that \gls{knn} is not suitable for tracking, as it might not be possible to impose \gls{knn} to comply with the tracking demand.\newline

The second approach was by applying \gls{dtw}. The conclusion is that DTW's tracking results highly depended on the order in which the inter $\beta $ distances are calculated. Further, it can be stated that DTW needs some initial wrong matching before the properly tracked outcomes are provided. 
The initial incorrect matching can be seen as the reason, why DTW is mostly used when very long signals, as in speech recognition, are provided.
In these cases, some mismatching is tolerated. For \gls{cnmc}, where only a few $K$ centroids are available, a mismatch is strongly unwelcome.\newline

The third method was based on the sequence of the labels. 
The idea was that the order of the movement from one centroid to another centroid is known. In other terms, if the current position is at centroid $c_i$ and the next position for centroid $c_{i+1}$ is known. Assuming that the sequences across the $\vec{\beta}$ should be very similar to each other, a majority vote should return the correct tracking results. It can be recorded that this was not the case and the approach was dismissed.\newline


After having explained 3 methods, which did not lead to a satisfactory outcome, the data and workflow of the final successful approach shall be presented.
First very concisely, followed by an in-depth account.
For that, figure \ref{fig_24_Tracking_Workflow} shall be analyzed.
The initial input is obtained through \emph{settings.py}, where execution, storage and plotting attributes are defined. 
For the further sub-steps, it shall be noted that the index big O stands for output of the respective sub-step.
The clustered results from step two, described in section \cite{sec_2_3_Clustering} are used as input for the so-called ordering step.
The ordered state can be stored and plotted if desired and exploited to calculate a cost matrix $\bm A (\vec{\beta})$. \newline

 % =============================================================================
 % ================ Tracking Workflow ==========================================
 % =============================================================================
 \begin{sidewaysfigure} [!]
    \hspace*{-2cm} 
     \resizebox{\textwidth}{!}{
     
%%% Preamble Requirements %%%
% \usepackage{geometry}
% \usepackage{amsfonts}
% \usepackage{amsmath}
% \usepackage{amssymb}
% \usepackage{tikz}

% Optional packages such as sfmath set through python interface
% \usepackage{sfmath}

% \usetikzlibrary{arrows,chains,positioning,scopes,shapes.geometric,shapes.misc,shadows}

%%% End Preamble Requirements %%%

\input{/home/jav/Progs/Virt_Env/writing/lib/python3.12/site-packages/pyxdsm/diagram_styles.tex}
\begin{tikzpicture}

\matrix[MatrixSetup]{
%Row 0
\node [DataIO] (output_track) {$\begin{array}{c}\text{tracking settings}\end{array}$};&
&
&
&
&
&
&
&
\\
%Row 1
\node [Function] (track) {$\begin{array}{c}\text{Tracking}\end{array}$};&
\node [DataInter] (track-ordering) {$\begin{array}{c}\text{clustered data}$$(\vec{\beta}),$$  \\ \text{ordering settings}\end{array}$};&
&
&
&
&
&
&
\\
%Row 2
&
\node [Function,stack] (ordering) {$\begin{array}{c}\text{ordering}$$\,(\vec{\beta})$$\end{array}$};&
\node [DataInter] (ordering-cost_A) {$\begin{array}{c}\text{ord}$$_O\,(\vec{\beta})$$ \end{array}$};&
&
&
&
&
&
\node [DataIO] (right_output_ordering) {$\begin{array}{c}\text{save: trajectories,} \\ \text{plots}\end{array}$};\\
%Row 3
&
&
\node [Function] (cost_A) {$\begin{array}{c}\text{calc. } $$A\, (\vec{\beta})$$\end{array}$};&
\node [DataInter] (cost_A-path) {$\begin{array}{c}$$A\,(\vec{\beta})$$ \end{array}$};&
&
&
&
&
\\
%Row 4
&
&
&
\node [Function] (path) {$\begin{array}{c}\text{best path} $$\,(\vec{\beta})$$\end{array}$};&
\node [DataInter] (path-validity) {$\begin{array}{c}\text{path}$$_O\,(\vec{\beta})$$ \end{array}$};&
&
&
&
\\
%Row 5
&
&
&
&
\node [Function] (validity) {$\begin{array}{c}\text{validity} \\ \text{check}\end{array}$};&
\node [DataInter] (validity-final) {$\begin{array}{c}\text{outcome} \end{array}$};&
&
&
\\
%Row 6
&
&
&
&
&
\node [Function] (final) {$\begin{array}{c}\text{truncate,} \\ \text{final path}\end{array}$};&
\node [DataInter] (final-transform) {$\begin{array}{c}\text{final path}$$_O$$ \end{array}$};&
&
\\
%Row 7
&
&
&
&
&
&
\node [Function,stack] (transform) {$\begin{array}{c}\text{transform}\end{array}$};&
&
\node [DataIO] (right_output_transform) {$\begin{array}{c}\text{save: trajectories,} \\ \text{plots}\end{array}$};\\
%Row 8
&
&
&
&
&
&
&
&
\\
};

% XDSM process chains
{ [start chain=process]
 \begin{pgfonlayer}{process} 
\chainin (track);
\chainin (ordering) [join=by ProcessHVA];
\chainin (cost_A) [join=by ProcessHVA];
\chainin (path) [join=by ProcessHVA];
\chainin (validity) [join=by ProcessHVA];
\chainin (final) [join=by ProcessHVA];
\chainin (transform) [join=by ProcessHVA];
\end{pgfonlayer}
}

\begin{pgfonlayer}{data}
\path
% Horizontal edges
(track) edge [DataLine] (track-ordering)
(ordering) edge [DataLine] (ordering-cost_A)
(cost_A) edge [DataLine] (cost_A-path)
(path) edge [DataLine] (path-validity)
(validity) edge [DataLine] (validity-final)
(final) edge [DataLine] (final-transform)
(ordering) edge [DataLine] (right_output_ordering)
(transform) edge [DataLine] (right_output_transform)
% Vertical edges
(track-ordering) edge [DataLine] (ordering)
(ordering-cost_A) edge [DataLine] (cost_A)
(cost_A-path) edge [DataLine] (path)
(path-validity) edge [DataLine] (validity)
(validity-final) edge [DataLine] (final)
(final-transform) edge [DataLine] (transform)
(track) edge [DataLine] (output_track);
\end{pgfonlayer}

\end{tikzpicture}

}
     \caption{General data and workflow overview of the third step, tracking}
     \label{fig_24_Tracking_Workflow}
 \end{sidewaysfigure}

The tracking demand is applied on $\bm A (\vec{\beta})$, e.g., each row element must be matched to exactly one column element with the constraint that their additive sum is minimized. 
It will return a suggested best path, i.e., the proposed optimized tracking path.
It is possible that the proposed optimized tracking path may not be feasible concerning a linking condition, it undergoes a validity check.
If required the suggested path will be chopped off and replaced such that the linking condition is met. The final path is then imposed to a transition such that the centroid labeling across all available $\vec{\beta}$ matches.
The transformed final paths are designated as the tracked outcome and can be saved and plotted.\newline

Since the fast description leaves some open questions, the in-depth explanation shall be tackled. Defining \emph{settings.py} is 
analogously done to the two previous steps, i.e. data generation \ref{sec_2_2_Data_Gen} and clustering \ref{sec_2_3_Clustering}. 
Therefore, accounts regarding the sub-tasks \emph{settings.py} and the clustered data are not repeated.\newline 

\textbf{1. Ordering$\,(\vec{\boldsymbol\beta})$}
\hrule
\vspace{0.05cm}
\hrule
\vspace{0.25cm}
The ordering of the clustered data can be understood by viewing figures \ref{fig_25_Clust}  and \ref{fig_26_Ord}.
Both depict the clustered Lorenz system \eqref{eq_6_Lorenz} for $\beta = 30.5$. 
The difference between the two figures is that figure \ref{fig_25_Clust} shows the clustering as it is obtained from the clustering step. It shall be referred to as the initial state.
Figure \ref{fig_26_Ord} on the other shows the ordered state, i.e. the state after applying the ordering sub-step. The labeling of the ordered step represents to some degree the actual movement of the trajectory. 
It can be observed that moving from label $1$ up to $6$ in a consecutive manner the resulting trajectory is generating the left ear of the Lorenz system. 
Analogously, moving from label $7$ up to $10$ produces the right ear of the Lorenz system. Furthermore, the transition from centroid $6$ to $7$ captures the transition from one ear to the other.\newline 

\begin{figure}[!h]
     %\vspace{0.5cm}
     \begin{minipage}[h]{0.47\textwidth}
         \centering
         \includegraphics[width =\textwidth]{2_Figures/2_Task/1_Tracking/4_Clus_30_5.pdf}
         \caption{Initial State - centroids of the Lorenz system  \eqref{eq_6_Lorenz} $\beta =30.5$}
         \label{fig_25_Clust}    
     \end{minipage}
     \hfill
     \begin{minipage}{0.47\textwidth}
         \centering
         \includegraphics[width =\textwidth]{2_Figures/2_Task/1_Tracking/5_Ordered_30_5.pdf}
         \caption{Ordered State - centroids of the Lorenz system  \eqref{eq_6_Lorenz} $\beta =30.5$}
         \label{fig_26_Ord}    
     \end{minipage}
 \end{figure}

 The way the ordered state is retrieved is as follows. The entire sequence of the motion along the centroids is available. In simpler terms, the first centroid from where the trajectory will start, all the upcoming centroids and the order in which they will be visited are known. 
 Therefore, the initial centroid can be labeled as $1$, the second as $2$ and so on. 
 However, it is important to note that with modifying one label of the trajectory sequence, the same label needs to be found in the entire sequence and modified as well. 
 Otherwise, the ordered-state is true for one turn and a wrong initial-ordered-state mixture is kept for the remaining turns. 
 Such an approach would also falsify the trajectory.
 The labeling in the ordered state provides information about the trajectory. 
Further, the original motion of the trajectory is untouched. Labeling the characteristic centroids with different numbers or with Greek letters does not impose any change on the original dynamics. For that to be fully true, the newly introduced labeling must be consistent across the entire sequence. 
Although it is obvious, nevertheless \gls{cnmc} performs a sanity check, i.e., it is verified, whether the resulting trajectory in the ordered state is the same as the original trajectory. 
Note, that all the same 4 types of plots stated in section \ref{sec_2_3_Clustering} are also available for visualizing the ordered state.
\newline

\textbf{2. Calculating $\bm A \, (\vec{\boldsymbol\beta})$ \& best path$\,(\vec{\boldsymbol\beta})$  }
\hrule
\vspace{0.05cm}
\hrule
\vspace{0.25cm}
In the next sub-task the cost or similarity matrix $\bm A(\vec{\beta})$ is calculated. 
First, the assignment problem shall be elaborated.
Let $\beta_1$ and $\beta_2$ be two different model parameter values $\beta_1 \neq \beta_2$ and both shall consist of $K$ centroids. Each centroid is not only associated with a label but described fully with a position. 
The goal is to match each centroid from $\beta_1$ to exactly one corresponding centroid from $\beta_2$ such that the overall spatial distance is minimized. 
This idea was given as part of the definition of the term tracking itself. 
The difference between tracking and the assignment problem is that first, tracking solves the assignment problem multiple times and thus the assignment problem is only a part of tracking.
Second, the tracked results are also feasible and transformed, which will be covered later in this subsection.\newline

For construction an illustrative cost matrix $\bm A(\vec{\beta})$, 
3 pairwise different $\beta$ values, $\beta_1, \, \beta_2, \beta_3$ with $(\beta_1,\neq \beta_2) \neq \beta_3$ shall be considered.
Again, each $\beta_i$, where $i = \{1,2,3\}$, consists of $K$ centroid positions.
The assignment problem is solved by exploiting \emph{SciPy} \cite{2020SciPy-NMeth}. 
Its solution, e.g., for $\beta_1$ and $\beta_2$ only matches the centroids from the two different $\beta$ such that the overall spatial distance is minimized. 
The addition of the spatial distances of $\beta_1$ and $\beta_2$ shall be designated as the cost value $\beta_{i=1,j=2}$.
With this level of understanding, the similarity matrix given in equation \eqref{eq_17_Dist_A} can be read.\newline

\begin{equation}
     \bm A_{3\times 3}\,(\vec{\beta}) = 
     \begin{bmatrix}
      \beta_{1,1} & \beta_{1,2} & \beta_{1,3}\\
       \beta_{2,1}  &\beta_{2,2} & \beta_{2,3}\\
       \beta_{3,1} & \beta_{3,2} &\beta_{3,3}
     \end{bmatrix}
     \label{eq_17_Dist_A}
   \end{equation}

Considering equation \eqref{eq_18_Expl}, if the assignment problem is solved for equal $\beta \Rightarrow \beta_i = \beta_j$, the centroid positions overlap exactly. 
As a consequence, the distance between all the centroids across the two same $\beta$ is zero. 
Further, adding up the zero spatial distances yields a cost of zero $\beta_{i,i} = 0$.\newline 

\begin{equation}
  \begin{aligned}
    i &= j \\ 
    \Rightarrow \beta_i &= \beta_j  \\
    \Rightarrow \beta_{i,j} &=  \beta_{i,i} = 0 
  \end{aligned}
    \label{eq_18_Expl}
\end{equation}

The cost matrix $\bm A\,(\vec{\beta})$ compares each $\beta_i$ with all the remaining $\beta_j$, where $i = \{1, \,2, \cdots, n_{\beta}\}, \; j = \{1, \,2, \cdots, n_{\beta}\}$ and $ n_{\beta}$ denotes the number of the pairwise different $\vec{\beta}$.
The outcome of each possible comparison $\beta_i$ with $\beta_j$ is a cost value representing a similarity between $\beta_i$ and $\beta_j$. 
Obviously, in the case trivial as described above $\beta_i = \beta_j$, the similarity is maximized and the cost is minimized. 
To find the best path, i.e., proposed tracking results, the trivial entries on the diagonal entries must be prohibited. Obeying that the cost matrix $\bm A\,(\vec{\beta})$ can be reformulated as equation \eqref{eq_19}. 
Moreover, $\bm A\,(\vec{\beta})$ is symmetrical, therefore computing one triangular part of the cost matrix is sufficient. 
The triangular part can be filled by mirroring along with the diagonal entries $\beta_{i,i}$ as outlined for the lower triangular matrix in equation \eqref{eq_19}. 
\newline

\begin{equation}
     \bm A_{3\times 3}\,(\vec{\beta}) = 
     \begin{bmatrix}
       \infty & \beta_{1,2} & \beta_{1,3}\\
       \beta_{2,1} = \beta_{1,2} & \infty & \beta_{2,3}\\
       \beta_{3,1} =  \beta_{1,3} & \beta_{3,2} =\beta_{2,3}  & \infty\\
     \end{bmatrix}
     \label{eq_19}
\end{equation}
The objective behind exploiting symmetry is to reduce computation time. 
Having defined the cost matrix $\bm A\,(\vec{\beta})$ as given in equation \eqref{eq_19}, it can be used to again solve the assignment problem.
Its output is denoted as path$_O\,(\vec{\beta })$ in figure \ref{fig_24_Tracking_Workflow}.\newline

\textbf{3. Validity check}
\hrule
\vspace{0.05cm}
\hrule
\vspace{0.25cm}
The validity check can also be regarded as a feasibility investigation.
To grasp what the feasibility constraint is table \ref{tab_1} shall be analyzed.\newline 

% ------------------------------------------------------------------------------
% ------------------------------ TABLE -----------------------------------------
% ------------------------------------------------------------------------------
\begin{table}[!h]
  
  \begin{minipage}{.3\linewidth}
    \caption{\emph{Direct feasible}}
    \centering
      \begin{tabular}{c|c}
        \multicolumn{1}{c}{$\beta_i$} &  \multicolumn{1}{c}{$\beta_j$} \\
          \hline
          \hline
          1 & 2 \\
          \hline
          2 & 3 \\
          \hline
          3 & 4 \\
          \hline
          4 & 5 \\
          \hline
          5 & 6 \\
          \hline
          6 & 7 
          \label{tab_1}
      \end{tabular}
  \end{minipage}
  \begin{minipage}{.3\linewidth}
    \centering
    \caption{feasible}
    \begin{tabular}{c|c}
      \multicolumn{1}{c}{$\beta_i$} &  \multicolumn{1}{c}{$\beta_j$} \\
        \hline
        \hline
        1 & 2 \\
        \hline
        2 & 3 \\
        \hline
        3 & 6 \\
        \hline
        4 & 7 \\
        \hline
        5 & 4 \\
        \hline
        6 & 5 
        \label{tab_2}
      \end{tabular}
    \end{minipage} 
  \begin{minipage}{.3\linewidth}
    \centering
    \caption{infeasible}
    \begin{tabular}{c|c}
      \multicolumn{1}{c}{$\beta_i$} &  \multicolumn{1}{c}{$\beta_j$} \\
        \hline
        \hline
        1 & 2 \\
        \hline 
        2 & 1 \\
        \hline
        3 & 5 \\
        \hline
        4 & 6 \\
        \hline
        5 & 7 \\
        \hline
        6 & 3 
        \label{tab_3}
      \end{tabular}
  \end{minipage} 
  \caption{Examples for feasible and infeasible best tracking paths}
  \label{tab_4}
\end{table}

% ------------------------------------------------------------------------------
% ------------------------------ TABLE -----------------------------------------
% ------------------------------------------------------------------------------


It can be observed that in total the 7 model parameter values $(\vec{\beta}, \, n_{\beta}=7)$ were chosen.
The overall goal is to provide one centroid label and get its corresponding centroid positions across all the 7 model parameter values $\vec{\beta }$. 
Therefore, a \emph{feasible linking path}, which allows the linking of all centroids of all $\beta_i$ to all the other $\beta_{\vec{j}}$ centroids, is required.
The latter description shall be elaborated step-wise in the following.
For instance, if the first $\beta_i = 1$, a linking to the remaining $\beta_{\vec{j}} = \{2, \, 3, \, 4, \, 5, \, 6, \, 7 \}$ is mandatory.
The first item of table \ref{tab_1} outlines that the centroids from $\beta_i = 1$ are tracked with the centroids $\beta_j=2$.
In other words, a clear relationship between the centroids across  $\beta_i = 1$ and $\beta_j=2$ is established. Leveraging this relationship, the proper tracked centroid position across the two $ \beta = 1$ and $\beta= 2$, are returned.\newline


Because the centroid labels of $\beta_i = 1$ are used as the reference to match the centroid labels of $\beta_j=2$, the \emph{known linked path} can be stated as $L_{known}= \{1,\, 2\}$. The next model parameter value $\beta_j = 3$ and it is tracked with $\beta_i =2$. 
Since $\beta_i =2$ is already incorporated in the \emph{known linked path}, the \emph{known linking path} can be extended to $L_{known}= \{1,\, 2, \, 3\}$. The next model parameter value $\beta_j = 4$ and its corresponding tracked counterpart is $\beta_i =3$. 
Again, $\beta_i =3$ is found in the \emph{known linked path}, therefore the \emph{known linking path} can be extended to $L_{known}= \{1,\, 2, \, 3, \, 4\}$. The next model parameter value $\beta_j = 5$ and its corresponding tracked $\beta_i =4$ and so this procedure can be performed until the last $\beta_j = 7$.
Having completed the scheme, the \emph{known linking path} is of full rank, i.e. with $n_{\beta}= 7$ all the 7 pairwise different model parameter values $\vec{\beta}$ are captured in the \emph{known linking path} $L_{known}$.
The information gained through a full ranked $L_{known, full}$ is that all centroids of all $\beta_i$ are linked to all the other $\beta_{\vec{j}}$ centroids. 
\newline 

After having introduced the full rank $L_{known, full}$, the more straightforward definition for \emph{feasible linking path} can be stated as follows. 
A \emph{feasible linking path} is given when $L_{known}$ has full rank $L_{known, full}$. \emph{Direct feasible} cases as shown in table \ref{tab_1} are one way of \emph{feasible linking paths}. Another, more general feasible case is provided in table \ref{tab_2}. Here, up to $\beta_i = 2$ and $\beta_j = 3$ the scheme of the \emph{direct feasible} linking path is followed.
However, with $\beta_i = 4$ and $\beta_j = 7$ the obstacle that $\beta_j = 7$ is not present in the current $L_{known}= \{1,\, 2,\, 3,\, 6\}$, occurs. 
This issue can be circumvented by viewing $\beta_i = 6$ and $\beta_j = 5$.
Since $\beta_i = 6$  is in the current state of $L_{known}= \{1,\, 2,\, 3,\, 6\}$, $L_{known}$ can be extended with $\beta_j = 5$, i.e., $L_{known}= \{1,\, 2,\, 3,\, 5, \, 6\}$. 
Note, having acquired the relationship between $\beta_i$ to $\beta_j$ is the same as knowing the relationship between $\beta_j$ to $\beta_i$.
Applying the newly added linking perspective, it can be seen that table \ref{tab_2} also demonstrates a fulled ranked $L_{known, full}$  or a  \emph{feasible linking path}.\newline

In table \ref{tab_3} an example for an incomplete linking path or an \emph{infeasible linking path} is provided, where $L_{known}$ has no full rank.
The aim of the sub-task, validity, is to determine, whether the proposed optimized tracking path is feasible by extracting information about the rank of the final $L_{known}$. 
Internally in \gls{cnmc}, this is achieved through logical queries utilizing mainly if statements.
One major advantage which was not mentioned when the examples above were worked out is the following. $\beta_{i,ref} = 1$ is not necessarily the best choice for being the reference. 
The reference $\beta_{i,ref}$ is chosen such that it has the overall highest similarity or least cost to all the other $(n_{\beta} -1)$ available $\vec{\beta}$.
Hence, a \emph{feasible linking path} with a lower sum of cost sum is generated.\newline 

This feature of a flexible reference is only providing better \emph{feasible linking paths}, when the proposed optimized tracking path is infeasible, which in general is the case. Therefore, in most cases, it is advantageous to leverage the flexible reference.
One of the outputs of \gls{cnmc} is the percentage cost savings that could be achieved with the flexible approach. In others, by what percentage could the costs be decreased when the flexible approach is compared with the rigid approach. 
In the rigid approach, $\beta_{i,ref} = 1$ is chosen as the reference.
Further, in the rigid approach, the $\vec{\beta}$ are linked in increasing order, i.e. $\beta_1$ with $\beta_1$, $\beta_2$  with $\beta_2$, $\beta_3$  with $\beta_4$ and so on. Exploiting the flexible approach yields cost savings of around $20\%$ to $50\%$ An example of coping with a flexible reference is provided in the description of the following sub-step.
\newline


\textbf{4. Truncate, final path}
\hrule
\vspace{0.05cm}
\hrule
\vspace{0.25cm}
If the proposed optimized tracking path is feasible (\emph{feasible linking path}), i.e. $L_{known}$ has full rank $L_{known, full}$, the truncation can be skipped. 
Consequently, the final path is the same as the proposed optimized tracking path.
However, as mentioned, in general, this is not the expected case.
Therefore, an example with an incomplete $L_{known}$ shall be portrayed to explain the workflow with active truncation.\newline 


First, the final incomplete $L_{known}$ will be used as the starting point. 
It will be filled until full rank is reached. 
Allowing a flexible reference $\beta_{i,ref}$ the incomplete \emph{known linked path} could be, e.g., $L_{known} = \{3, \, 4, \, 5\}$. 
To get full rank, the remaining $L_{missing} = \{1, \, 2, \, 6, \, 7\}$ are inferred through the following concept.
The cost $\beta_{i,j}$ between all $L_{known}$ and $L_{missing}$ are known
through the cost matrix $\bm A\,(\vec{\beta })$.
The one $\beta_j$ entry from $L_{missing}$ which has the highest similarity or lowest cost $\beta_{i,j}$ to the one entry $\beta_j$ of the $L_{known}$, is removed from $L_{missing}$ and added to $L_{known}$. 
Now, the same procedure can be applied to the modified $L_{known}$ and $L_{missing}$ until $L_{missing}$ is empty and $L_{known}$ has reached full rank.
The inferred $L_{known, full}$ is then used as the final path and sent to the next sub-task.\newline


\textbf{5. Transform}
\hrule
\vspace{0.05cm}
\hrule
\vspace{0.25cm}
Once the final path is determined, it is known which $\beta_i$ is linked to which $\beta_j$. 
For all the $\beta_{i},\, \beta_j$ matches in the final path, the linear sum assignment problem is solved again.
Two illustrative solutions are provided in section \ref{sec_3_1_Tracking_Results}.
For further explanation, table \ref{tab_2} shall be revisited. 
The first $\beta_{i},\, \beta_j$ link is defined as $\beta_i = 1$ and $\beta_j = 2$. Moreover, for this scenario, it is assumed that $\beta_i = \beta_{ref} = 1$. Therefore, the $\beta_{i} = 1,\, \beta_j= 2$ is denoted as a direct match.
In simpler terms, a direct pairwise $\beta_{i},\, \beta_j$ relation, is obtained when  $\beta_i$ or $\beta_j$ is directly traced back to the reference.
For a pairwise direct $\beta_{i},\, \beta_j$ link the 
transformation, i.e., relabeling without changing the dynamics of the system, as explained for the ordering sub-step, is applied directly and only once.\newline


Now, considering the next $\beta_{i},\, \beta_j$ match, i.e., $\beta_i = 2$ and $\beta_j = 3$. 
Linking the centroids from $\beta_j = 3$ to $\beta_i = 2$ directly would have no connection to the reference $\beta_{ref} = 1$. 
Therefore, the solution to its linear sum assignment problem must experience the same transformations as $\beta_i = 2$ did. 
In this case it is only the transformation caused by the $(\beta_i = 1,\,\beta_j = 2)$ match.
The idea behind the transformation stays the same, however, if no direct relation is seen, respective multiple transformations must be performed.
Once the final path has undergone all required transformations, the output is the desired tracked state. 
The output can be stored and plotted if desired. 
Some types of plots, which can be generated, will be shown in the section 
\ref{sec_3_1_Tracking_Results}.\newline

Finally, in short, the difference between \emph{first CNMc} and this \gls{cnmc} version shall be mentioned. 
The proposed tracking algorithm is neither restricted to any dimension nor to a specific dynamical system. Thus, two major limitations of \emph{first CNMc} could be removed in the current \gls{cnmc} version. 
Also, the flexible approach yields a better feasible linking path. 

\section{Modeling}
\label{sec_2_4_Modeling}
In this section, the fourth main step of \gls{cnmc}, i.e., modeling, is elaborated.
The data and workflow is described in figure \ref{fig_42}.
It comprises two main sub-tasks, which are modeling the \glsfirst{cpevol} and modeling the transition properties tensors $\bm Q / \bm T$. 
The settings are as usually defined in \emph{settings.py} and the extracted attributes are distributed to the sub-tasks. 
Modeling the \gls{cpevol} and the $\bm Q/ \bm T$ tensors can be executed separately from each other. 
If the output of one of the two modeling sub-steps is at hand, \gls{cnmc} is not forced to recalculate both modeling sub-steps.
Since the tracked states are used as training data to run the modeling they are prerequisites for both modeling parts.
The modeling of the centroid position shall be explained in the upcoming subsection \ref{subsec_2_4_1_CPE}, followed by the explanation of the transition properties in subsection \ref{subsec_2_4_2_QT}.
A comparison between this \gls{cnmc} and the \emph{first CNMc} version is provided at the end of the respective subsections.
The results of both modeling steps can be found in section
\ref{sec_3_2_MOD_CPE} and \ref{sec_3_3_SVD_NMF}

\begin{figure} [!h]
    \hspace*{-4cm} 
    \resizebox{1.2\textwidth}{!}{
    
%%% Preamble Requirements %%%
% \usepackage{geometry}
% \usepackage{amsfonts}
% \usepackage{amsmath}
% \usepackage{amssymb}
% \usepackage{tikz}

% Optional packages such as sfmath set through python interface
% \usepackage{sfmath}

% \usetikzlibrary{arrows,chains,positioning,scopes,shapes.geometric,shapes.misc,shadows}

%%% End Preamble Requirements %%%

\input{/home/jav/Progs/Virt_Env/writing/lib/python3.12/site-packages/pyxdsm/diagram_styles.tex}
\begin{tikzpicture}

\matrix[MatrixSetup]{
%Row 0
\node [DataIO] (output_data) {$\begin{array}{c}\text{modelling settings}\end{array}$};&
&
&
&
\\
%Row 1
\node [Function] (data) {$\begin{array}{c}\text{Modeling}\end{array}$};&
\node [DataInter] (data-ode) {$\begin{array}{c}\text{CPE settings,} \\ \text{tracked state}$$\,(\beta)$$\end{array}$};&
\node [DataInter] (data-qt) {$\begin{array}{c}$$\boldsymbol{Q} / \boldsymbol{T}$$\text{ settings,} \\ \text{tracked state}$$\,(\beta)$$\end{array}$};&
&
\\
%Row 2
&
\node [Function,stack] (ode) {$\begin{array}{c}\text{Model} \\ \text{CPE}$$\,(\beta)$$\end{array}$};&
&
&
\node [DataIO] (right_output_ode) {$\begin{array}{c}\text{save: CPE}$$\,(\beta)$$ \\ \text{plots}\end{array}$};\\
%Row 3
&
&
\node [Function,stack] (qt) {$\begin{array}{c}\text{Model} \\ $$\boldsymbol{Q} / \boldsymbol{T} \,(\beta)$$\end{array}$};&
&
\node [DataIO] (right_output_qt) {$\begin{array}{c}\text{save:}$$ \boldsymbol{Q} / \boldsymbol{T} \,(\beta)$$ \\ \text{plots}\end{array}$};\\
%Row 4
&
&
&
&
\\
};

% XDSM process chains
{ [start chain=process]
 \begin{pgfonlayer}{process} 
\chainin (data);
\chainin (ode) [join=by ProcessHVA];
\end{pgfonlayer}
}{ [start chain=process]
 \begin{pgfonlayer}{process} 
\chainin (data);
\chainin (qt) [join=by ProcessHVA];
\end{pgfonlayer}
}

\begin{pgfonlayer}{data}
\path
% Horizontal edges
(data) edge [DataLine] (data-ode)
(data) edge [DataLine] (data-qt)
(ode) edge [DataLine] (right_output_ode)
(qt) edge [DataLine] (right_output_qt)
% Vertical edges
(data-ode) edge [DataLine] (ode)
(data-qt) edge [DataLine] (qt)
(data) edge [DataLine] (output_data);
\end{pgfonlayer}

\end{tikzpicture}

    }
    \caption{Data and workflow of the fourth step: Modeling}
    \label{fig_42}
\end{figure}


\subsection{Modeling the centroid position evolution}
\label{subsec_2_4_1_CPE}
In this subsection, the modeling of the \gls{cpevol} is described. 
The objective is to find a surrogate model, which returns all $K$ centroid positions for an unseen $\beta_{unseen}$. 
The training data for this are the tracked centroids from the previous step, which is described in section \ref{sec_2_3_Tracking}. 
To explain the modeling of the \emph{CPE}, figure \ref{fig_43} shall be inspected.
The model parameter values which shall be used to train the model $\vec{\beta_{tr}}$ are used for generating a so-called candidate library matrix $\boldsymbol{\Theta}\,(\vec{\beta_{tr}})$. The candidate library matrix $\boldsymbol{\Theta}\,(\vec{\beta_{tr}})$ is obtained making use of a function of \emph{pySindy} \cite{Silva2020,Kaptanoglu2022,Brunton2016}.
In \cite{Brunton2016} the term $\boldsymbol{\Theta}\,(\vec{\beta_{tr}})$ is explained well. However, in brief terms, it allows the construction of a matrix, which comprises the output of defined functions. 
These functions could be, e.g., a linear, polynomial, trigonometrical or any other non-linear function. Made-up functions that include logical conditions can also be applied. \newline 

\begin{figure} [!h]
    \hspace*{-4cm} 
    \resizebox{1.2\textwidth}{!}{
    
%%% Preamble Requirements %%%
% \usepackage{geometry}
% \usepackage{amsfonts}
% \usepackage{amsmath}
% \usepackage{amssymb}
% \usepackage{tikz}

% Optional packages such as sfmath set through python interface
% \usepackage{sfmath}

% \usetikzlibrary{arrows,chains,positioning,scopes,shapes.geometric,shapes.misc,shadows}

%%% End Preamble Requirements %%%

\input{/home/jav/Progs/Virt_Env/writing/lib/python3.12/site-packages/pyxdsm/diagram_styles.tex}
\begin{tikzpicture}

\matrix[MatrixSetup]{
%Row 0
\node [DataIO] (output_theta) {$\begin{array}{c}\text{tracked state}$$\,(\vec{\beta})$$\end{array}$};&
&
&
&
&
&
\\
%Row 1
\node [Function] (theta) {$\begin{array}{c}$$ \boldsymbol \theta \, (\vec{\beta})  $$\end{array}$};&
\node [DataInter] (theta-linear) {$\begin{array}{c}$$ \boldsymbol \theta \, (\vec{\beta}) $$\end{array}$};&
\node [DataInter] (theta-pysindy) {$\begin{array}{c}$$ \boldsymbol \theta \, (\vec{\beta})  $$\end{array}$};&
\node [DataInter] (theta-elastic) {$\begin{array}{c}$$ \boldsymbol \theta \, (\vec{\beta}) $$\end{array}$};&
&
&
\\
%Row 2
&
\node [Function] (linear) {$\begin{array}{c}\text{Linear}\end{array}$};&
&
&
\node [DataInter] (linear-comp) {$\begin{array}{c}$$ md_{lin} $$\end{array}$};&
&
\\
%Row 3
&
&
\node [Function] (pysindy) {$\begin{array}{c}\text{pySindy}\end{array}$};&
&
\node [DataInter] (pysindy-comp) {$\begin{array}{c}$$ md_{pysind}  $$\end{array}$};&
&
\\
%Row 4
&
&
&
\node [Function] (elastic) {$\begin{array}{c}\text{Elastic net}\end{array}$};&
\node [DataInter] (elastic-comp) {$\begin{array}{c}$$ md_{elast} $$\end{array}$};&
&
\\
%Row 5
&
&
&
&
\node [Function] (comp) {$\begin{array}{c}\text{compare}\end{array}$};&
&
\node [DataIO] (right_output_comp) {$\begin{array}{c}\text{save: best modesl}\end{array}$};\\
%Row 6
&
&
&
&
&
&
\\
};

% XDSM process chains
{ [start chain=process]
 \begin{pgfonlayer}{process} 
\chainin (theta);
\chainin (linear) [join=by ProcessHVA];
\end{pgfonlayer}
}{ [start chain=process]
 \begin{pgfonlayer}{process} 
\chainin (theta);
\chainin (pysindy) [join=by ProcessHVA];
\end{pgfonlayer}
}{ [start chain=process]
 \begin{pgfonlayer}{process} 
\chainin (theta);
\chainin (elastic) [join=by ProcessHVA];
\end{pgfonlayer}
}{ [start chain=process]
 \begin{pgfonlayer}{process} 
\chainin (elastic);
\chainin (comp) [join=by ProcessHVA];
\end{pgfonlayer}
}{ [start chain=process]
 \begin{pgfonlayer}{process} 
\chainin (linear);
\chainin (comp) [join=by ProcessHVA];
\end{pgfonlayer}
}{ [start chain=process]
 \begin{pgfonlayer}{process} 
\chainin (pysindy);
\chainin (comp) [join=by ProcessHVA];
\end{pgfonlayer}
}

\begin{pgfonlayer}{data}
\path
% Horizontal edges
(theta) edge [DataLine] (theta-linear)
(theta) edge [DataLine] (theta-pysindy)
(theta) edge [DataLine] (theta-elastic)
(linear) edge [DataLine] (linear-comp)
(pysindy) edge [DataLine] (pysindy-comp)
(elastic) edge [DataLine] (elastic-comp)
(comp) edge [DataLine] (right_output_comp)
% Vertical edges
(theta-linear) edge [DataLine] (linear)
(theta-pysindy) edge [DataLine] (pysindy)
(theta-elastic) edge [DataLine] (elastic)
(linear-comp) edge [DataLine] (comp)
(pysindy-comp) edge [DataLine] (comp)
(elastic-comp) edge [DataLine] (comp)
(theta) edge [DataLine] (output_theta);
\end{pgfonlayer}

\end{tikzpicture}

    }
    \caption{Data and workflow of modeling the \glsfirst{cpevol}} 
    \label{fig_43}
\end{figure}

Since, the goal is not to explain, how to operate \emph{pySindy} \cite{Brunton2016}, the curious reader is referred to the \emph{pySindy} very extensive online documentation and \cite{Silva2020,Kaptanoglu2022}.
Nevertheless, to understand $\boldsymbol{\Theta}\,(\vec{\beta_{tr}})$ equation \eqref{eq_20} shall be considered. 
In this example, 3 different functions, denoted as $f_i$ in the first row, are employed.
The remaining rows are the output for the chosen $f_i$.
Furthermore, $n$ is the number of samples or the size of $\vec{\beta_{tr} }$, i.e., $n_{\beta,tr} $ and $m$ denotes the number of the features, i.e., the number of the functions $f_i$. \newline

\begin{equation}
    \boldsymbol{\Theta_{exampl(n \times m )}}(\,\vec{\beta_{tr}}) =
    % \renewcommand\arraystretch{3} 
    \renewcommand\arraycolsep{10pt} 
    \begin{bmatrix}
      f_1 = \beta & f_2 = \beta^2 & f_2 = cos(\beta)^2 - exp\,\left(\dfrac{\beta}{-0.856} \right) \\[1.5em]
      1  & 1^2 & cos(1)^2 - exp\,\left(\dfrac{1}{-0.856} \right) \\[1.5em]
      2 &  2^2  & cos(2)^2 - exp\,\left(\dfrac{2}{-0.856} \right) \\[1.5em]
    \end{bmatrix}
    \label{eq_20}
\end{equation}

The actual candidate library matrix $\boldsymbol{\Theta}\,(\vec{\beta_{tr}})$ incorporates  a quadratic polynomial, the inverse $ \frac{1}{\beta}$, the exponential $exp(\beta)$ and 3 frequencies of cos and sin, i.e., $cos(\vec{\beta}_{freq}), \ sin(\vec{\beta}_{freq})$, where $\vec{\beta}_{freq} = [1, \, 2,\, 3]$. 
There are much more complex $\boldsymbol{\Theta}\,(\vec{\beta_{tr}})$  available in \gls{cnmc}, which can be selected if desired. 
Nonetheless, the default $\boldsymbol{\Theta}\,(\vec{\beta_{tr}})$ is chosen as described above.
Once $\boldsymbol{\Theta}\,(\vec{\beta_{tr}})$  is the generated, the system of equations \eqref{eq_21} is solved. 
Note, this is very similar to solving the well-known $\bm A \, \vec{x} =  \vec{y}$ system of equations.
The difference is that the vectors $\vec{x}, \, \vec{y}$ can be vectors in the case of \eqref{eq_21} as well, but in general, they are the matrices $\bm{X} ,\, \bm Y$, respectively. The solution to the matrix $\bm{X}$ is the desired output.
It contains the coefficients which assign importance to the used functions $f_i$. 
The matrix $\bm Y$ contains the targets or the known output for the chosen functions $f_i$.
Comparing $\bm A$ and $ \boldsymbol{\Theta}\,(\vec{\beta_{tr}})$ mathematically, no difference exists.\newline
 
\begin{equation}
    \boldsymbol{\Theta}\,(\vec{\beta_{tr}}) \: \bm X = \bm Y
    \label{eq_21}
\end{equation}

With staying in the \emph{pySindy} environment, the system of equations \eqref{eq_21} is solved by means of the optimizer \emph{SR3}, which is implemented in  \emph{pySindy}.
Details and some advantages of the  \emph{SR3}  optimizer can be found in \cite{SR3}. Nevertheless, two main points shall be stated. 
It is highly unlikely that the $\boldsymbol{\Theta}\,(\vec{\beta_{tr}}),\: \bm X,\, \bm Y$ is going to lead to a well-posed problem, i.e., the number of equations are equal to the number of unknowns and having a unique solution.
In most cases the configuration will be ill-posed, i.e., the number of equations is not equal to the number of unknowns. 
In the latter case, two scenarios are possible, the configuration could result in an over-or under-determined system of equations.\newline 

For an over-determined system, there are more equations than unknowns. 
Thus,  generally, no outcome that satisfies equation \eqref{eq_21} exists.
In order to find a representation that comes close to a solution, an error metric is defined as the objective function for optimization.
There are a lot of error metrics or norms, however, some commonly used \cite{Brunton2019} are given in equations \eqref{eq_22} to \eqref{eq_24}, where $f(x_k)$ are true values of a function and $y_k$ are their corresponding predictions.
The under-determined system has more unknown variables than equations, thus infinitely many solutions exist. 
To find one prominent solution, again, optimization is performed.
Note, for practical application penalization or regularization parameter are exploited as additional constraints within the definition of the optimization problem. 
For more about over- and under-determined systems as well as for deploying optimization for finding a satisfying result the reader is referred to \cite{Brunton2019}.\newline

\begin{equation}
    E_{\infty} = \max_{1<k<n} |f(x_k) -y_k | \quad \text{Maximum Error} \;(l_{\infty})
    \label{eq_22}
\end{equation}

\vspace{0.1cm}
\begin{equation}
    E_{1} = \frac{1}{n} \sum_{k=1}^{n} |f(x_k) -y_k | \quad \text{Mean Absolute Error} \;(l_{1})
    \label{eq_23}
\end{equation}

\vspace{0.1cm}
\begin{equation}
    E_{2} =  \sqrt{\frac{1}{n} \sum_{k=1}^{n} |f(x_k) -y_k |^2 }  \quad \text{Least-squares Error} \;(l_{2})
    \label{eq_24}
\end{equation}
\vspace{0.1cm}

The aim for modeling \emph{CPE} is to receive a regression model, which is sparse, i.e., it is described through a small number of functions $f_i$.
For this to work, the coefficient matrix $\bm X$ must be sparse, i.e., most of its entries are zero. 
Consequently, most of the used functions $f_i$ would be inactive and only a few $f_i$ are actively applied to capture the \emph{CPE} behavior.
The $l_1$ norm as defined in \eqref{eq_23} and the $l_0$ are metrics which promotes sparsity.
In simpler terms, they are leveraged to find only a few important and active functions $f_i$. 
The $l_2$ norm as defined in \eqref{eq_24} is known for its opposite effect, i.e. to assign importance to a high number of $f_i$.
The \emph{SR3} optimizer is a sparsity promoting optimizer, which deploys $l_0$ and $l_1$ regularization.\newline

The second point which shall be mentioned about the \emph{SR3} optimizer is that it can cope with over-and under-determined systems and solves them without any additional input.
One important note regarding the use of \emph{pySindy} is that \emph{pySindy} in this thesis is not used as it is commonly. For modeling the \emph{CPE} only the modules for generating the candidate library matrix $\boldsymbol{\Theta}\,(\vec{\beta_{tr}})$ and the \emph{SR3} optimizer are utilized.\newline 

Going back to the data and workflow in figure \ref{fig_43}, the candidate library matrix $\boldsymbol{\Theta}\,(\vec{\beta_{tr}})$ is generated. 
Furthermore, it also has been explained how it is passed to \emph{pySindy} and how \emph{SR3} is used to find a solution. It can be observed that $\boldsymbol{\Theta}\,(\vec{\beta_{tr}})$ is also passed to a \emph{Linear} and \emph{Elastic net} block. The \emph{Linear} block is used to solve the system of equations \eqref{eq_21} through linear interpolation.
The \emph{Elastic net} solves the same system of equations with the elastic net approach. In this the optimization is penalized with an $l_1$ and $l_2$ norm.
In other words, it combines the Lasso \cite{Lasso, Brunton2019} and Ridge \cite{Brunton2019}, regression respectively. 
The linear and elastic net solvers are invoked from the \emph{Scikit-learn} \cite{scikit-learn} library.\newline 

The next step is not depicted in figure \ref{fig_43}. 
Namely, the linear regression model is built with the full data. For \emph{pySindy} and the elastic net, the models are trained with $90 \%$ of the training data and the remaining $10 \%$ are used to test or validate the model.
For \emph{pySindy} $20$ different models with the linear distributed thresholds starting from $0.1$ and ending at $2.0$ are generated. 
The model which has the least mean absolute error \eqref{eq_23} will be selected as the \emph{pySindy} model.
The mean absolute error of the linear, elastic net and the selected \emph{pySindy} will be compared against each other. 
The one regression model which has the lowest mean absolute error is selected as the final model.\newline

The described process is executed multiple times. 
In 3-dimensions the location of a centroid is given as the coordinates of the 3 axes. 
Since the \emph{CPE} across the 3 different axes can deviate significantly, capturing the entire behavior in one model would require a complex model. 
A complex model, however, is not sparse anymore. 
Thus, a regression model for each of the $K$ labels and for each of the 3 axes is required. 
In total $3 \, K$ regression models are generated. \newline

Finally, \emph{first CNMc} and \gls{cnmc} shall be compared.
First, in \emph{first CNMc} only \emph{pySindy} with a different built-in optimizer. 
Second, the modeling \emph{CPE} was specifically designed for the Lorenz system \eqref{eq_6_Lorenz}. 
Third, \emph{first CNMc} entirely relies on \emph{pySindy}, no linear and elastic models are calculated and used for comparison. 
Fourth, the way \emph{first CNMc} would perform prediction, was by transforming the active $f_i$ with their coefficients to equations such that \emph{SymPy} could be applied. 
The disadvantage is that if $\boldsymbol{\Theta}\,(\vec{\beta_{tr}})$ is changed, modifications for \emph{SymPy} are necessary.
Also, $\boldsymbol{\Theta}\,(\vec{\beta_{tr}})$ can be used for arbitrary defined functions $f_i$, \emph{SymPy} functions, however, are restricted to some predefined functions.
In \gls{cnmc} it is also possible to get the active $f_i$ as equations. 
However, the prediction is obtained with a regular matrix-matrix multiplication as given in equation \eqref{eq_25}. The variables are denoted as the predicted outcome $\bm{\tilde{Y}}$, the testing data for which the prediction is desired $\bm{\Theta_s}$ and the coefficient matrix $\bm X$ from equation \eqref{eq_21}.

\begin{equation}
    \bm{\tilde{Y}} = \bm{\Theta_s} \, \bm X 
    \label{eq_25}
\end{equation}


With leveraging equation \eqref{eq_25} the limitations imposed through \emph{SymPy} are removed.

\subsection{Modeling Q/T}
\label{subsec_2_4_2_QT}
In this subsection, the goal is to explain how the transition properties are modeled. 
The transition properties are the two tensors $\bm Q$ and $\bm T$, which consist of transition probability from one centroid to another and the corresponding transition time, respectively.
For further details about the transition properties, the reader is referred to section \ref{sec_1_1_2_CNM}.
Modeling $\bm Q / \bm T$ means to find surrogate models that capture the trained behavior and can predict the tensors for unseen model parameter values $\bm{\tilde{Q}}(\vec{\beta}_{unseeen}) ,\, \bm{\tilde{T}}(\vec{\beta}_{unseeen})$.
To go through the data and workflow figure \ref{fig_44} shall be considered.\newline


\begin{figure} [!h]
    \hspace*{-4cm} 
    \resizebox{1.2\textwidth}{!}{
    
%%% Preamble Requirements %%%
% \usepackage{geometry}
% \usepackage{amsfonts}
% \usepackage{amsmath}
% \usepackage{amssymb}
% \usepackage{tikz}

% Optional packages such as sfmath set through python interface
% \usepackage{sfmath}

% \usetikzlibrary{arrows,chains,positioning,scopes,shapes.geometric,shapes.misc,shadows}

%%% End Preamble Requirements %%%

\input{/home/jav/Progs/Virt_Env/writing/lib/python3.12/site-packages/pyxdsm/diagram_styles.tex}
\begin{tikzpicture}

\matrix[MatrixSetup]{
%Row 0
\node [DataIO] (output_cnm) {$\begin{array}{c}\text{tracked state}$$\,(\beta)$$\end{array}$};&
&
&
&
&
&
\\
%Row 1
\node [Function,stack] (cnm) {$\begin{array}{c}\text{CNM}$$ \, (\beta)  $$\end{array}$};&
\node [DataInter] (cnm-transform) {$\begin{array}{c}\text{dicts: }$$ \boldsymbol{Q} / \boldsymbol{T} \, (\beta) $$ \end{array}$};&
&
&
&
&
\node [DataIO] (right_output_cnm) {$\begin{array}{c}\text{save}\end{array}$};\\
%Row 2
&
\node [Function,stack] (transform) {$\begin{array}{c}\text{transform}\end{array}$};&
\node [DataInter] (transform-stack) {$\begin{array}{c}$$ \boldsymbol{Q} / \boldsymbol{T} \, (\beta) $$\end{array}$};&
&
&
&
\\
%Row 3
&
&
\node [Function] (stack) {$\begin{array}{c}\text{stacking}\end{array}$};&
\node [DataInter] (stack-decomp) {$\begin{array}{c}$$\boldsymbol{Q} / \boldsymbol{T}_{stack} \, (\beta) $$\end{array}$};&
&
&
\\
%Row 4
&
&
&
\node [Function] (decomp) {$\begin{array}{c}\text{decomposition}\end{array}$};&
\node [DataInter] (decomp-regr) {$\begin{array}{c}\text{modes}\end{array}$};&
&
\node [DataIO] (right_output_decomp) {$\begin{array}{c}\text{save, plots}\end{array}$};\\
%Row 5
&
&
&
&
\node [Function,stack] (regr) {$\begin{array}{c}\text{regression}\end{array}$};&
&
\node [DataIO] (right_output_regr) {$\begin{array}{c}\text{save, plots}\end{array}$};\\
%Row 6
&
&
&
&
&
&
\\
};

% XDSM process chains
{ [start chain=process]
 \begin{pgfonlayer}{process} 
\chainin (cnm);
\chainin (transform) [join=by ProcessHVA];
\chainin (stack) [join=by ProcessHVA];
\chainin (decomp) [join=by ProcessHVA];
\chainin (regr) [join=by ProcessHVA];
\end{pgfonlayer}
}

\begin{pgfonlayer}{data}
\path
% Horizontal edges
(cnm) edge [DataLine] (cnm-transform)
(transform) edge [DataLine] (transform-stack)
(stack) edge [DataLine] (stack-decomp)
(decomp) edge [DataLine] (decomp-regr)
(cnm) edge [DataLine] (right_output_cnm)
(decomp) edge [DataLine] (right_output_decomp)
(regr) edge [DataLine] (right_output_regr)
% Vertical edges
(cnm-transform) edge [DataLine] (transform)
(transform-stack) edge [DataLine] (stack)
(stack-decomp) edge [DataLine] (decomp)
(decomp-regr) edge [DataLine] (regr)
(cnm) edge [DataLine] (output_cnm);
\end{pgfonlayer}

\end{tikzpicture}

    }
    \caption{Data and workflow of $\bm Q / \bm T$ modeling}
    \label{fig_44}
\end{figure}

First, the tracked state data is loaded and adapted in the sense that CNM's data format is received. After that \gls{cnm} can be executed on the tracked state data.
The outcome of \gls{cnm} are the transition property tensors for all the provided model parameter values $\bm Q(\vec{\beta}) ,\, \bm T(\vec{\beta})$.
However, \gls{cnm} does not return tensors as \emph{NumPy} \cite{harris2020array} arrays, but as \emph{Python} dictionaries. 
Thus, the next step is to transform the dictionaries to \emph{NumPy} arrays.
$\bm Q / \bm T$ are highly sparse, i.e., $85 \% - 99\%$ of the entries can be zero. 
The $99\%$ case is seen with a great model order, which for the Lorenz system \eqref{eq_6_Lorenz} was found to be $L \approx 7$.
Furthermore, with an increasing $L$, saving the dictionaries as \emph{NumPy} arrays becomes inefficient and at some $L$ impossible. With $L>7$ the storage cost goes above multiple hundreds of gigabytes of RAM.
Therefore, the dictionaries are converted into sparse matrices. \newline 

Thereafter, the sparse matrices are reshaped or stacked into a single matrix, such that a modal decomposition method can be applied.
Followed by training a regression model for each of the mode coefficients.
The idea is that the regression models receive a $\beta_{unseen}$ and returns all the corresponding predicted modes. 
The regression models are saved and if desired plots can be enabled via \emph{settings.py}. \newline 

In this version of \gls{cnmc} two modal decomposition methods are available.
Namely, the \glsfirst{svd} and the \glsfirst{nmf}.
The difference between both is given in \cite{Lee1999}.
The \gls{svd} is stated in equation \eqref{eq_26}, where the variables are designated as the input matrix $\bm A$ which shall be decomposed, the left singular matrix $ \bm U $, the diagonal singular matrix $ \bm \Sigma $ and the right singular matrix $ \bm V^T $.
The singular matrix $ \bm \Sigma $ is mostly ordered in descending order such that the highest singular value is the first diagonal element. 
The intuition behind the singular values is that they assign importance to the modes in the left and right singular matrices $ \bm U $ and $ \bm {V^T} $, respectively. 

\begin{equation}
    \bm A = \bm U \, \bm \Sigma \, \bm {V^T}
    \label{eq_26}
\end{equation}


The big advantage of the \gls{svd} is observed when the so-called economical \gls{svd} is calculated. 
The economical \gls{svd} removes all zero singular values, thus the dimensionality of all 3 matrices can be reduced. 
However, from the economical \gls{svd} as a basis, all the output with all $r$ modes is available.
There is no need to perform any additional \gls{svd} to get the output for $r$ modes, but rather the economical \gls{svd} is truncated with the number $r$ for this purpose.
\gls{nmf}, given in equation \eqref{eq_5_NMF}, on the other hand, has the disadvantage that there is no such thing as economical NMF. 
For every change in the number of modes $r$, a full \gls{nmf} must be recalculated.\newline

\begin{equation}
    \bm {A_{i \mu}} \approx \bm A^{\prime}_{i \mu}  = (\bm W  \bm H)_{i \mu}  = \sum_{a = 1}^{r} 
    \bm W_{ia} \bm H_{a \mu}
    \tag{\ref{eq_5_NMF}}
\end{equation}


The issue with \gls{nmf} is that the solution is obtained through an iterative optimization process. 
The number of iterations can be in the order of millions and higher to meet the convergence criteria.
Because the optimal $r_{opt}$ depends on the dynamical system, there is no general rule for acquiring it directly. 
Consequently, \gls{nmf} must be run with multiple different $r$ values to find $r_{opt}$. 
Apart from the mentioned parameter study, one single \gls{nmf} execution was found to be more computationally expensive than \gls{svd}. 
In \cite{Max2021} \gls{nmf} was found to be the performance bottleneck of \emph{first CNMc}, which became more evident when $L$ was increased.
In subsection 
\ref{subsec_3_3_1_SVD_Speed}
a comparison between \gls{nmf} and \gls{svd} regarding computational time is given.\newline


Nevertheless, if the user wants to apply \gls{nmf}, only one attribute in \emph{settings.py} needs to be modified.
Because of that and the overall modular structure of \gls{cnmc}, implementation of any other decomposition method should be straightforward.
In \gls{cnmc} the study for finding $r_{opt}$ is automated and thus testing \gls{cnmc} on various dynamical systems with \gls{nmf} should be manageable. 
The benefit of applying \gls{nmf} is that the entries of the output matrices $\bm W_{ia},\, \bm H_{a \mu}$ are all non-zero.
This enables interpreting the $\bm W_{ia}$ matrix since both $\bm Q / \bm T$ tensors cannot contain negative entries, i.e., no negative probability and no negative transition time.\newline

Depending on whether  \gls{nmf} or \gls{svd} is chosen, $r_{opt}$ is found through a parameter study or based on $99 \%$ of the information content, respectively.
The $99 \%$ condition is met when $r_{opt}$ modes add up to $99 \%$ of the total sum of the modes. In \gls{svd} $r_{opt}$ is automatically detected and does not require any new \gls{svd} execution. A comparison between \gls{svd} and \gls{nmf} regarding prediction quality is given in 
section
\ref{subsec_3_3_2_SVD_Quality}.
After the decomposition has been performed, modes that capture characteristic information are available. 
If the modes can be predicted for any $\beta_{unseen}$, the predicted transition properties $\bm{\tilde{Q}}(\vec{\beta}_{unseeen}) ,\, \bm{\tilde{T}}(\vec{\beta}_{unseeen})$ are obtained. 
To comply with this \gls{cnmc} has 3 built-in methods. 
Namely, \textbf{R}andom \textbf{F}orest (RF), AdaBoost, and Gaussian Process.\newline 

First, \gls{rf} is based on decision trees, but additionally deploys
a technique called bootstrap aggregation or bagging. 
Bagging creates multiple sets from the original dataset, which are equivalent in size. 
However, some features are duplicated in the new datasets, whereas others are 
neglected entirely. This allows \gls{rf} to approximate very complex functions 
and reduce the risk of overfitting, which is encountered commonly 
with regular decision trees. 
Moreover, it is such a powerful tool
that, e.g., Kilian Weinberger, a well-known Professor for machine learning 
at Cornell University, considers \gls{rf} in one of his lectures, to be 
one of the most powerful regression techniques that the state of the art has to offer.
Furthermore, \gls{rf} proved to be able to approximate the training data  
acceptable as shown in \cite{Max2021}. 
However, as mentioned in subsection \ref{subsec_1_1_3_first_CNMc}, it faced difficulties to approximate spike-like curves. 
Therefore, it was desired to test alternatives as well.\newline

These two alternatives were chosen to be AdaBoost, and Gaussian Process.
Both methods are well recognized and used in many machine learning applications.
Thus, instead of motivating them and giving theoretical explanations, the reader is referred to \cite{Adaboost}, \cite{Rasmussen2004,bishop2006pattern} for AdaBoost and Gaussian Process, respectively. 
As for the implementation, all 3 methods are invoked through \emph{Scikit-learn} \cite{scikit-learn}.
The weak learner for AdaBoost is  \emph{Scikit-learn's} default decision tree regressor.
The kernel utilized for the Gaussian Process is the \textbf{R}adial \textbf{B}asis \textbf{F}unction (RBF). A comparison of these 3 methods in terms of prediction capabilities is provided in section \ref{subsec_3_3_2_SVD_Quality}.\newline

Since the predicted $\bm{\tilde{Q}}(\vec{\beta}_{unseeen}) ,\, \bm{\tilde{T}}(\vec{\beta}_{unseeen})$ are based on regression techniques, their output will have some deviation to the original $\bm{Q}(\vec{\beta}_{unseeen}) ,\, \bm{T}(\vec{\beta}_{unseeen})$. 
Due to that, the physical requirements given in equations \eqref{eq_31_Q_T_prediction} may be violated. \newline

\begin{equation}
    \begin{aligned}
        0 \leq  \,  \bm Q \leq 1\\
        \bm T \geq 0 \\
        \bm Q \,(\bm T > 0) > 0 \\
        \bm T(\bm Q = 0) = 0 
    \end{aligned}
    \label{eq_31_Q_T_prediction}
\end{equation}

To manually enforce these physical constraints, the rules defined in equation \eqref{eq_32_Rule} are applied. 
The smallest allowed probability is defined to be 0, thus negative probabilities are set to zero. 
The biggest probability is 1, hence, overshoot values are set to 1.
Also, negative transition times would result in moving backward, therefore, they are set to zero. 
Furthermore, it is important to verify that a probability is zero if its corresponding transition time is less than or equals zero.
In general, the deviation is in the order of $\mathcal{O}(1 \mathrm{e}{-2})$, such that the modification following equation \eqref{eq_32_Rule} can be considered reasonable. 
  \newline
\begin{equation}
    \begin{aligned}
        & \bm Q < 0 := 0 \\
        & \bm Q > 1 := 1 \\
        & \bm T < 0 := 0\\
        & \bm Q \, (\bm T \leq 0) := 0
    \end{aligned}
    \label{eq_32_Rule}
    \vspace{0.1cm}
\end{equation}
In conclusion, it can be said that modeling $\bm Q / \bm T$ \gls{cnmc} is equipped with two different modal decomposition methods, \gls{svd} and NMF.
To choose between them one attribute in \emph{settings.py} needs to be modified.
The application of \gls{nmf} is automated with the integrated parameter study.
For the mode surrogate models, 3 different regression methods are available.
Selecting between them is kept convenient, i.e. by editing one property in \emph{settings.py}. 










% % % ---------------- Task 3  ------------------------------
\chapter{Results}
\label{ch_3}
In this chapter, the results achieved with \gls{cnmc} shall be presented and assessed.
First, in section \ref{sec_3_1_Tracking_Results}, the tracking algorithm is evaluated by showing the outcome for 3 different dynamical model configurations.
Second, in section \ref{sec_3_2_MOD_CPE}, statements about the performance of modeling the \glsfirst{cpevol} are made. 
They are supported with some representative outputs. 
Third, in section \ref{sec_3_3_SVD_NMF} the two decomposition methods are compared in terms of computational time and prediction quality in subsection \ref{subsec_3_3_1_SVD_Speed} and \ref{subsec_3_3_2_SVD_Quality}, respectively.
Fourth, it has been mentioned that 3 different regressors for representing the $\bm Q / \bm T$ tensor are available. 
Their rating is given in  section \ref{sec_3_4_SVD_Regression}.
Finally, the \gls{cnmc} predicted trajectories for different models shall be displayed and evaluated in section \ref{sec_3_5_Pred}.\newline


For assessing the performance of \gls{cnmc} some dynamical model with a specific configuration will be used many times. 
In order not to repeat them too often, they will be defined in the following.\newline 

 \textbf{Model configurations}
 \hrule
 \vspace{0.05cm}
 \hrule
 \vspace{0.25cm}
 The first model configuration is denoted as  \emph{SLS}, which stands for \textsl{S}mall \textbf{L}orenz \textsl{S}ystem .
 It is the Lorenz system described with the sets of equations \eqref{eq_6_Lorenz} and the number of centroids is $K=10$. 
 Furthermore, the model was trained with $\vec{\beta }_{tr} = [\beta_0 = 28 ; \, \beta_{end} = 33], \, n_{\beta, tr} = 7$, where the training model parameter values $\vec{\beta}_{tr}$ are chosen to start from $\beta_0 = 28$ and end at $\beta_{end} = 33$, where the total number of linearly distributed model parameter values is $n_{\beta, tr} = 7$.\newline

 The second model is referred to as \emph{LS20}. 
 It is also a Lorenz system \eqref{eq_6_Lorenz}, but with a higher number of centroids $K=20$ and the following model configuration: $\vec{\beta }_{tr} = [\, \beta_0 = 24.75 ; \, \beta_{end} = 53.75  \,], \, n_{\beta, tr} = 60$.\newline

The third model is designated as \emph{FW15}. It is based on the \emph{Four Wing} set of equations \eqref{eq_10_4_Wing} and an illustrative trajectory is given in figure \ref{fig_37}.
The number of centroids is $K=15$ and it is constructed with the following configuration $\vec{\beta }_{tr} = [\, \beta_0 = 8 ; \, \beta_{end} = 11 \,], \, n_{\beta, tr} = 13$.\newline

\begin{figure}[!h]
    \centering
    \includegraphics[width =\textwidth]
    {2_Figures/3_Task/1_Tracking/10_1_Traj_8.pdf}
    \caption{\emph{FW15} \eqref{eq_10_4_Wing} trajectory for $\beta = 8$}
    \label{fig_37}
\end{figure}




\section{Tracking results}
\label{sec_3_1_Tracking_Results}
In this section, some outputs of tracking data and workflow, described in subsection \ref{subsec_2_3_1_Tracking_Workflow}, shall be presented. 
After that, in short, the current \gls{cnmc} shall be compared to \emph{first CNMc} \newline

First, two illustrative solutions for the assignment problem from the final path, as explained in subsection \ref{subsec_2_3_1_Tracking_Workflow}, are provided in figures \ref{fig_27} and \ref{fig_28}. 
The axes are denoted as $c_k$ and $c_p$ and represent the labels of the $\beta_j$ and $\beta_i$ centroids, respectively.
The color bar on the right side informs about the euclidean distance, which is equivalent to the cost.
Above the solution of the assignment problem in figures \ref{fig_27} and \ref{fig_28}, the corresponding $\beta_i$ and $\beta_j$ centroid labels are given in the respective two figures, i.e., \ref{fig_27_1}, \ref{fig_27_2} and \ref{fig_28_1}, \ref{fig_28_2}.

\begin{figure}[!h]
    \begin{subfigure}{0.5\textwidth}
        \centering
        \caption{Ordered state, $\beta_i =32.167$ }
        \includegraphics[width =\textwidth]
        {2_Figures/3_Task/1_Tracking/16_lb_32.167.pdf}
        \label{fig_27_1}
    \end{subfigure}
    \hfill
    \begin{subfigure}{0.5\textwidth}
        \centering
        \caption{Ordered state, $\beta_j = 33$}
        \includegraphics[width =\textwidth]
        {2_Figures/3_Task/1_Tracking/17_lb_33.000.pdf}
        \label{fig_27_2}
    \end{subfigure}
    
    \smallskip
    \centering
    \begin{subfigure}{\textwidth}
        \caption{Solution to the assignment problem}
        \includegraphics[width =\textwidth]{2_Figures/3_Task/1_Tracking/1_LSA.pdf}
        \label{fig_27}    
    \end{subfigure}
    \vspace{-0.3cm}
    \caption{Illustrative solution for the assignment problem, $\beta_i =32.167,\, \beta_j = 33 ,\, K =10$}
    \label{fig_27_All}    
\end{figure}
%
%
The centroid $c_{k=1} (\beta_j = 33)$ has its lowest cost to 
$c_{p=3} (\beta_i = 32.167)$. In this case, this is also the solution for the assignment problem, outlined by the blue cross. 
However, the solution to the linear sum assignment problem is not always to choose the minimal cost for one \emph{inter} $\beta$ match.
It could be that one centroid in $\beta_i$ is to found the closest centroid to multiple centroids in $\beta_j$. 
Matching only based on the minimal distance does not include the restriction that exactly one centroid from $\beta_i$ must be matched with exactly one centroid from $\beta_j$. 
The latter demand is incorporated in the solution of the linear sum assignment problem. \newline 


\begin{figure}[!h]
    \begin{subfigure}{0.5\textwidth}
        \centering
        \caption{Ordered state, $\beta_i =31.333$ }
        \includegraphics[width =\textwidth]
        {2_Figures/3_Task/1_Tracking/18_lb_31.333.pdf}
        \label{fig_28_1}
    \end{subfigure}
    \hfill
    \begin{subfigure}{0.5\textwidth}
        \centering
        \caption{Ordered state, $\beta_j = 32.167$}
        \includegraphics[width =\textwidth]
        {2_Figures/3_Task/1_Tracking/16_lb_32.167.pdf}
        \label{fig_28_2}
    \end{subfigure}
    
    \smallskip
    \centering
    \begin{subfigure}{\textwidth}
        \caption{Solution to the assignment problem}
        \includegraphics[width =\textwidth]{2_Figures/3_Task/1_Tracking/2_LSA.pdf}
        \label{fig_28}    
    \end{subfigure}
    \vspace{-0.3cm}
    \caption{Illustrative solution for the assignment problem, $\beta_i =31.333,\, \beta_j = 32.167, \,K =10 $}
    \label{fig_28_All}    
\end{figure}

Comparing figure \ref{fig_27} with the second example in figure \ref{fig_28}, it can be observed that the chosen \emph{inter} $\beta$ centroid matches can have very different shapes. 
This can be seen by looking at the blue crosses.
Furthermore, paying attention to the remaining possible \emph{inter} $\beta$ centroid matches, it can be stated that there is a clear trend, i.e., the next best \emph{inter} $\beta$ centroid match has a very high increase in its cost.
For example, considering the following \emph{inter} $\beta$ match. With $c_{k=1} (\beta_j = 32.167)$ and $c_{p=1} (\beta_i = 31.333)$, the minimal cost is around $cost_{min} \approx 0.84$. The next best option jumps to $cost_{second} = 13.823$. These jumps can be seen for each \emph{inter} $\beta$ match in figure in both depicted figures \ref{fig_27} and \ref{fig_28}.
The key essence behind this finding is that for the chosen number of centroids $K$ of this dynamical model (Lorenz system \eqref{eq_6_Lorenz}), no ambiguous regions, as explained at the beginning of this chapter, occur.\newline 

Next, the tracking result of 3 different systems shall be viewed.
The tracked state for \emph{SLS} is depicted in figures \ref{fig_29}.
In each of the figures, one centroid is colored blue that denotes 
the first centroid in the sequence of the underlying trajectory.
Within the depicted range $\vec{\beta}$, it can be observed, that each label across the $\vec{\beta}$ is labeled as expected.
No single ambiguity or mislabeling can be seen. 
In other words, it highlights the high performance of the tracking algorithm.
%
%
% ==============================================================================
% ======================= SLS =================================================
% ==============================================================================
\begin{figure}[!h]
    \begin{subfigure}{0.5\textwidth}
        \centering
        \caption{$\beta =28$ }
        \includegraphics[width =\textwidth]
        {2_Figures/3_Task/1_Tracking/3_lb_28.000.pdf}
    \end{subfigure}
    \hfill
    \begin{subfigure}{0.5\textwidth}
        \centering
        \caption{ $\beta = 28.833$}
        \includegraphics[width =\textwidth]
        {2_Figures/3_Task/1_Tracking/4_lb_28.833.pdf}
    \end{subfigure}

    \smallskip
    \begin{subfigure}{0.5\textwidth}
        \centering
        \caption{$\beta = 31.333$}
        \includegraphics[width =\textwidth]
        {2_Figures/3_Task/1_Tracking/15_lb_31.333.pdf}
    \end{subfigure}
    \hfill
    \begin{subfigure}{0.5\textwidth}
        \centering
        \caption{ $\beta = 33$}
        \includegraphics[width =\textwidth]
        {2_Figures/3_Task/1_Tracking/5_lb_33.000.pdf}
    \end{subfigure}
    \vspace{-0.3cm}
    \caption{Tracked states for \emph{SLS}, $K = 10,\, \vec{\beta} = [\, 28, \, 28.333, \, 31.333, \, 31.14, \, 33  \, ]$}
    \label{fig_29}
\end{figure}
% ==============================================================================
% ======================= SLS =================================================
% ==============================================================================
%
The second model is the \emph{LS20}, i.e, $K= 20,\, \vec{\beta }_{tr} = [\, \beta_0 = 24.75 ; \, \beta_{end} = 53.75  \,], \, n_{\beta,tr} = 60$.
The outcome is depicted in figures \ref{fig_32}.
It can be noted that $\beta = 24.75$ and $\beta = 30.648$ exhibit very similar results to the \emph{SLS} model. 
The same is true for intermediate $\beta$ values, i.e., $24.75 \leq \beta \lessapprox 30.648 $. 
However,  with $\beta \gtrapprox 30.64$ as depicted for $\beta =  31.14$, one centroid, i.e. the centroid with the label $20$ in the right ear appears unexpectedly. 
With this, a drastic change to the centroid placing network is imposed.
Looking at the upcoming $\beta$ these erratic changes are found again.\newline 


% ==============================================================================
% ======================= LS20 =================================================
% ==============================================================================
\begin{figure}[!h]
    \begin{subfigure}{0.5\textwidth}
        \centering
        \caption{$\beta =24.75$ }
        \includegraphics[width =\textwidth]
        {2_Figures/3_Task/1_Tracking/6_lb_24.750.pdf}
    \end{subfigure}
    \hfill
    \begin{subfigure}{0.5\textwidth}
        \centering
        \caption{ $\beta = 28.682$}
        \includegraphics[width =\textwidth]
        {2_Figures/3_Task/1_Tracking/7_lb_28.682.pdf}
    \end{subfigure}

    \smallskip
    \begin{subfigure}{0.5\textwidth}
        \centering
        \caption{$\beta = 30.648$}
        \includegraphics[width =\textwidth]
        {2_Figures/3_Task/1_Tracking/7_lb_30.648.pdf}
    \end{subfigure}
    \hfill
    \begin{subfigure}{0.5\textwidth}
        \centering
        \caption{ $\beta = 31.140$}
        \includegraphics[width =\textwidth]
        {2_Figures/3_Task/1_Tracking/8_lb_31.140.pdf}
    \end{subfigure}

    \smallskip
    \begin{subfigure}{0.5\textwidth}
        \centering
        \caption{$\beta = 42.936$}
        \includegraphics[width =\textwidth]
        {2_Figures/3_Task/1_Tracking/9_lb_42.936.pdf}
    \end{subfigure}
    \hfill
    \begin{subfigure}{0.5\textwidth}
        \centering
        \caption{ $\beta = 53.750$}
        \includegraphics[width =\textwidth]
        {2_Figures/3_Task/1_Tracking/10_lb_53.750.pdf}
    \end{subfigure}
    \vspace{-0.3cm}
    \caption{Tracked states for \emph{LS20}, $K = 20,\, \vec{\beta} = [\, 24.75, \, 28.682, \, 30.648, \, 31.14, \, 31.14,$ $42.936, \, 53.75  \, ]$ }
    \label{fig_32}
\end{figure}
% ==============================================================================
% ======================= LS20 =================================================
% ==============================================================================
Generating a tracked state with these discontinuous cluster network deformations even manually can be considered hard to impossible because tracking demands some kind of similarity. 
If two cluster networks differ too much from each other, then necessarily at least tracked label is going to be unsatisfying. 
Hence, it would be wrong to conclude that the tracking algorithm is not performing well, but rather the clustering algorithm itself or the range of $\vec{\beta} $ must be adapted. If the range of $\vec{\beta} $  is shortened, multiple models can be trained and tracked.\newline 

\FloatBarrier
The third model is referred to as \emph{FW15}. 
Figures in \ref{fig_38} show the tracked state for 4 different $\beta$ values. It can be observed that for $\beta = 11$ the centroid placing has changed notably to the other $\beta$ values, thus tracking the centroids in the center for $\beta = 11$ becomes unfavorable. 
Overall, however, the tracked state results advocate the performance of the tracking algorithm.\newline

% ==============================================================================
% ======================= FW15 =================================================
% ==============================================================================
\begin{figure}[!h]
    \begin{subfigure}{0.5\textwidth}
        \centering
        \caption{$\beta =8$ }
        \includegraphics[width =\textwidth]
        {2_Figures/3_Task/1_Tracking/11_lb_8.000.pdf}
    \end{subfigure}
    \hfill
    \begin{subfigure}{0.5\textwidth}
        \centering
        \caption{ $\beta = 8.25$}
        \includegraphics[width =\textwidth]
        {2_Figures/3_Task/1_Tracking/12_lb_8.250.pdf}
    \end{subfigure}

    \smallskip
    \begin{subfigure}{0.5\textwidth}
        \centering
        \caption{$\beta = 10$}
        \includegraphics[width =\textwidth]
        {2_Figures/3_Task/1_Tracking/13_lb_10.000.pdf}
    \end{subfigure}
    \hfill
    \begin{subfigure}{0.5\textwidth}
        \centering
        \caption{ $\beta = 11$}
        \includegraphics[width =\textwidth]
        {2_Figures/3_Task/1_Tracking/14_lb_11.000.pdf}
    \end{subfigure}
    \vspace{-0.3cm}
    \caption{Tracked states for \emph{FW15}, $K = 15,\, \vec{\beta} = [\, 8, \, 8.25, \, 10, \, 11 \, ]$}
    \label{fig_38}
\end{figure}
% ==============================================================================
% ======================= FW15 =================================================
% ==============================================================================

It can be concluded that the tracking algorithm performs remarkably well. However, when the cluster placing network is abruptly changed from one $\beta$ to the other $\beta$, the tracking outcome gets worse and generates sudden cluster network deformation. 
As a possible solution, splitting up the $\vec{\beta}_{tr}$ range into smaller $\vec{\beta}_{tr,i}$ ranges, can be named. This is not only seen for the \emph{LS20}, but also for other dynamical systems as illustratively shown with the center area of the \emph{FW15} system for $\beta= 11$.
\FloatBarrier   


\section{CPE modeling results}
\label{sec_3_2_MOD_CPE}
In this section, results to the \gls{cpevol} modeling explained in subsection \ref{subsec_2_4_1_CPE}, shall be presented and assessed.
First, a selection of equations, which defines the \gls{cpevol} are given for one model configuration.
Next, representative plots of the \gls{cpevol} for different models are analyzed.
Finally, the predicted centroid position is compared with the actual clustered centroid position.\newline 


Modeling the \emph{CPE} returns, among other results, analytical equations. 
These equations describe the behavior of the centroid positions across the range $\vec{\beta}$ and can also be used for making predictions for $\vec{\beta}_{unseen}$.
The model configuration for which they are be presented is \emph{SLS}, i.e. Lorenz system \eqref{eq_6_Lorenz}, $K= 10,\, \vec{\beta }_{tr} = [\, \beta_0 = 28 ; \, \beta_{end} =33  \,], \, n_{\beta, tr} = 7$. 
The analytical \gls{cpevol} expressions are listed in \eqref{eq_27} to \eqref{eq_29} for the centroids  $[\,1,\, 2,\,7\,]$, respectively.
Recalling that the behavior across the 3 different axes (x, y, z) can vary greatly, each axis has its own regression model $(\tilde x,\, \tilde y,\, \tilde z)$.
Thus, for each label, 3 different analytical expressions are provided. \newline


\begin{figure}[!h]
    \begin{minipage}{.47\textwidth}
      \begin{equation}
        \begin{aligned}
            \tilde x &= -0.1661 \, cos(3  \, \beta) \\
            \tilde y &=  -0.1375 \, cos(3 \,  \beta) \\
            \tilde z &=  0.8326 \, \beta 
        \end{aligned}
        \label{eq_27}
      \end{equation}
    \end{minipage}%
    \hfill
    \begin{minipage}{.47\textwidth}
        \centering
        \includegraphics[width =\textwidth]{2_Figures/3_Task/2_Mod_CPE/1_lb_1.pdf}
        \caption{\emph{SLS}, \emph{CPE} model for centroid: 1 }
        \label{fig_45}    
    \end{minipage}
\end{figure}

\begin{figure}[!h]
    \begin{minipage}{.47\textwidth}
        \begin{equation}
            \begin{aligned}
            \tilde x &= 0.1543 \, sin(3 \, \beta) + 0.2446 \, cos(3 \, \beta) \\
            \tilde y &= 0.2638 \, sin(3 \, \beta) + 0.4225 \, cos(3 \, \beta) \\
            \tilde z &= 0.4877 \, \beta
        \end{aligned}
        \label{eq_28}
    \end{equation}
\end{minipage}%
\hfill
\begin{minipage}{.47\textwidth}
    \centering
    \includegraphics[width =\textwidth]{2_Figures/3_Task/2_Mod_CPE/2_lb_2.pdf}
    \caption{\emph{SLS}, \emph{CPE} model for centroid: 2 }
    \label{fig_46}    
\end{minipage}
\end{figure}

\begin{figure}[!h]
    \begin{minipage}{.47\textwidth}
      \begin{equation}
        \begin{aligned}
            \tilde x &= -0.1866 \, \beta + 0.133 \, sin(3 \, \beta) \\
            & \quad + 0.1411 \, cos(3 \, \beta) \\
            \tilde y &= -0.3 \, \beta \\
            \tilde z &= -1.0483+ 0.6358 \,\beta
        \end{aligned}
        \label{eq_29}
      \end{equation}
    \end{minipage}%
    \hfill
    \begin{minipage}{.47\textwidth}
        \centering
        \includegraphics[width =\textwidth]{2_Figures/3_Task/2_Mod_CPE/3_lb_7.pdf}
        \caption{\emph{SLS}, \emph{CPE} model for centroid: 7 }
        \label{fig_47}    
    \end{minipage}
\end{figure}


Right to the equations the corresponding plots are depicted in figures \ref{fig_45} to \ref{fig_47}. 
Here, the blue and green curves indicate true and modeled CPE, respectively.
Each of the figures supports the choice of allowing each axis to be modeled separately.
The z-axis appears to undergo less alteration or to be more linear than the x- and y-axis.
If a model is supposed to be valid for all 3 axes, a more complex model, i.e., a higher of terms, is required.
Although more flexible models fit training data increasingly better, they tend to overfit. 
In other words, complex models capture the trained data well but could exhibit oscillations for $\vec{\beta}_{unseen}$.
The latter is even more severe when the model is employed for extrapolation.
The difference between interpolation and extrapolation is that for extrapolation the prediction is made with $\beta_{unseen}$ which are not in the range of the trained $\vec{\beta}_{tr}$.
Therefore, less complexity is preferred.\newline 

Next, the performance of predicting the centroid for $\vec{\beta}_{unseen}$ is elaborated.
For this purpose, figures \ref{fig_48} to \ref{fig_52} shall be examined.
All figures depict the original centroid positions, which are obtained through the clustering step in green and the predicted centroid positions in blue.
On closer inspection, orange lines connecting the true and predicted centroid positions can be identified. 
Note, that they will only be visible if the deviation between the true and predicted state is high enough.
Figures \ref{fig_48_0} an \ref{fig_48_1} show the outcome for \emph{SLS} with $\beta_{unseen} = 28.5$ and $\beta_{unseen} = 32.5$, respectively.
Visually, both predictions are very close to the true centroid positions.
Because of this high performance in showed in figures \ref{fig_49_0} and \ref{fig_49_1} two examples for extrapolation are given for $\beta_{unseen} = 26.5$ and $\beta_{unseen} = 37$, respectively. 
For the first one, the outcome is very applicable. 
In contrast, $\beta_{unseen} = 37$ returns some deviations which are notably high.
\newline
 


% ----------------- Interpolation ----------------------------------------------
\begin{figure}[!h]
    \begin{subfigure}{0.5\textwidth}
        \centering
        \caption{$\beta_{unseen} = 28.5$ }
        \includegraphics[width =\textwidth]{2_Figures/3_Task/2_Mod_CPE/4_lb_c_28.5.pdf}
        %MSE = 0.622
        \label{fig_48_0}    
    \end{subfigure}%
    \hfill
    \begin{subfigure}{0.5\textwidth}
        \centering
        \caption{$\beta_{unseen} = 32.5$ }
        \includegraphics[width =\textwidth]{2_Figures/3_Task/2_Mod_CPE/5_lb_c_32.5.pdf}
        %MSE = 0.677
        \label{fig_48_1}    
    \end{subfigure}
    \vspace{-0.3cm}
    \caption{\emph{SLS}, original vs. modeled centroid position, $\beta_{unseen} = 28.5$ and $\beta_{unseen} = 32.5$ }
    \label{fig_48}    
\end{figure}

% ----------------- EXTRAPOLATION ----------------------------------------------
\begin{figure}[!h]
    \begin{subfigure}{0.5\textwidth}
        \centering
        \caption{$\beta_{unseen} = 26.5$ }
        \includegraphics[width =\textwidth]{2_Figures/3_Task/2_Mod_CPE/22_lb_c_26.5.pdf}
        %MSE = 0.622
        \label{fig_49_0}    
    \end{subfigure}%
    \hfill
    \begin{subfigure}{0.5\textwidth}
        \centering
        \caption{$\beta_{unseen} = 37$ }
        \includegraphics[width =\textwidth]{2_Figures/3_Task/2_Mod_CPE/23_lb_c_37.0.pdf}
        %MSE = 0.677
        \label{fig_49_1}    
    \end{subfigure}
    \vspace{-0.3cm}
    \caption{\emph{SLS}, original vs. modeled centroid position, extrapolated $\beta_{unseen} = 26.5$ and $\beta_{unseen} = 37$ }
    \label{fig_49}    
\end{figure}


% --------- MODEL LOrenz K= 20
\begin{figure}[!h]
    \begin{subfigure}{.5\textwidth}
        \centering
        \caption{$\beta_{unseen} = 31.75$}
        \includegraphics[width =\textwidth]{2_Figures/3_Task/2_Mod_CPE/6_lb_c_31.75.pdf}
        %MSE = 1.857
    \end{subfigure}%
    \hfill
    \begin{subfigure}{.5\textwidth}
        \centering
        \caption{$\beta_{unseen} = 51.75$ }
        \includegraphics[width =\textwidth]{2_Figures/3_Task/2_Mod_CPE/7_lb_c_51.75.pdf}
        %MSE = 2.536
    \end{subfigure}
    \vspace{-0.3cm}
    \caption{\emph{LS20}, original vs. modeled centroid position, $\beta_{unseen} = 31.75$ and $\beta_{unseen} = 51.75$}
    \label{fig_50}    
\end{figure}

% --------- MODEL 25_Four_Wing_1_K_15 ---------
\begin{figure}[!h]
    \begin{subfigure}{.5\textwidth}
        \centering
        \caption{$\beta_{unseen} = 8.7$}
        \includegraphics[width =\textwidth]{2_Figures/3_Task/2_Mod_CPE/8_lb_c_8.7.pdf}
        %MSE = 1.617
    \end{subfigure}%
    \hfill
    \begin{subfigure}{.5\textwidth}
        \centering
        \caption{$\beta_{unseen} = 10.1$ }
        \includegraphics[width =\textwidth]{2_Figures/3_Task/2_Mod_CPE/9_lb_c_10.1.pdf}
        %MSE = 1.5
    \end{subfigure}
    \vspace{-0.3cm}
    \caption{\emph{FW15}, original vs. modeled centroid position, $\beta_{unseen} = 8.7$ and $\beta_{unseen} = 10.1$}
    \label{fig_52}    
    
\end{figure}

Quantitative measurements are performed by applying the Mean Square Error (MSE) following equation \eqref{eq_30_MSE}. 
The variables are denoted as the number of samples $n$, which in this case is equal to the number of centroids $n = K$, the known $f(x_k)$ and the predicted $y_k$ centroid position.\newline

\begin{equation}
        MSE = \frac{1}{n} \, \sum_{i=1}^n \left(f(x_k) - y_k\right)^2
        \label{eq_30_MSE}
\end{equation}

The measured MSE errors for all displayed results are summarized in table \ref{tab_5_MSE}. 
The MSE for results of $\beta_{unseen} = 28.5$ and $\beta_{unseen} = 32.5$ in figures \ref{fig_48} is $0.622$ and $0.677$, respectively. 
Consequently, the performance of \gls{cnmc} is also confirmed quantitatively.
Figures in \ref{fig_50} illustrate the outcome for \emph{LS20} for $\beta_{unseen} = 31.75$ and $\beta_{unseen} = 51.75$.
In section \ref{sec_3_1_Tracking_Results} it is explained that for  \emph{LS20} cluster network deformations appear. 
Nevertheless, the outcome visually and quantitatively endorses the \emph{CPE} modeling capabilities.
Figures in \ref{fig_52}  depict the outcome for \emph{FW15} for $\beta_{unseen} = 8.7$ and $\beta_{unseen} = 10.1$.
A few orange lines are visible, however overall the outcome is very satisfactory.\newline

\begin{table}
    \centering
    \begin{tabular}{c c c c }
        \textbf{Figure} &\textbf{Model} & $\boldsymbol{\beta_{unseen}}$ & \textbf{MSE} \\
        \hline \\
        [-0.8em]
        \ref{fig_48} & \emph{SLS}& $28.5$  & $0.622$   \\
        \ref{fig_48} & \emph{SLS}& $32.5$  & $0.677$   \\
        \ref{fig_49} & \emph{SLS}& $26.5$  & $1.193$   \\
        \ref{fig_49} & \emph{SLS}& $37$  & $5.452$   \\
        \ref{fig_50} & \emph{LS20}& $31.75$  & $1.857$ \\
        \ref{fig_50} & \emph{LS20}& $51.75$  & $2.536$ \\
        \ref{fig_52} & \emph{FW15}& $8.7$  & $1.617$   \\
        \ref{fig_52} & \emph{FW15}& $10.1$  & $1.5$    
    \end{tabular}
    \caption{MSE for different Model configurations and $\vec{\beta}_{unseen}$}
    \label{tab_5_MSE}
\end{table}

It can be concluded that the \emph{CPE} modeling performance is satisfying.
In the case of a few cluster network deformations, \gls{cnmc} is capable of providing acceptable results.
However, as shown with \emph{SLS}, if the model's training range $\vec{\beta}_{tr}$ and the number of $K$ was selected appropriately, the MSE can be minimized.
\section{Transition properties modeling}
\label{sec_3_3_SVD_NMF}
In the subsection \ref{subsec_2_4_2_QT}, it has been explained that \gls{cnmc} has two built-in modal decomposition methods for the $\bm Q / \bm T$ tensors, i.e., \gls{svd} and NMF.
There are two main concerns for which performance measurements are needed.
First, in subsection \ref{subsec_3_3_1_SVD_Speed}, the computational costs of both methods are examined.
Then in subsection \ref{subsec_3_3_2_SVD_Quality}, the \gls{svd} and \gls{nmf} prediction quality will be presented and assessed.

\subsection{Computational cost}
\label{subsec_3_3_1_SVD_Speed}
In this subsection, the goal is to evaluate the computational cost of the two decomposition methods implemented in \gls{cnmc}.
\gls{nmf} was already used in \emph{first CNMc} and it was found to be one of the most computational expensive tasks.
With an increasing model order $L$ it became the most computational task by far, which is acknowledged by \cite{Max2021}. 
The run time was one of the main reasons why \gls{svd} should be implemented in \gls{cnmc}. 
To see if \gls{svd} can reduce run time, both methods shall be compared.\newline 

First, it is important to mention that \gls{nmf} is executed for one single predefined mode number $r$. 
It is possible that a selected $r$ is not optimal, since $r$ is a parameter that depends not only on the chosen dynamical system but also on other parameters, e.g., the number of centroids $K$ and training model parameter values $n_{\beta, tr}$, as well as \gls{nmf} specific attributes. 
These are the maximal number of iterations in which the optimizer can converge and tolerance convergence.
However, to find an appropriate $r$, \gls{nmf} can be executed multiple times with different values for $r$.
Comparing the execution time of \gls{nmf} with multiple invocations against \gls{svd} can be regarded as an unbalanced comparison.
Even though for a new dynamical system and its configuration the optimal $r_{opt}$ for \gls{nmf} is most likely to be found over a parameter study, for the upcoming comparison, the run time of one single \gls{nmf} solution is measured.\newline 

The model for this purpose is \emph{SLS}. Since \emph{SLS} is trained with the output of 7 pairwise different model parameter values $n_{\beta,tr} = 7$, the maximal rank in \gls{svd} is limited to 7.
Nevertheless, allowing \gls{nmf} to find a solution $r$ was defined as $r=9$, the maximal number of iterations in which the optimizer can converge is 10 million and the convergence tolerance is $1\mathrm{e}{-6}$.
Both methods can work with sparse matrices. 
However, the \gls{svd} solver is specifically designed to solve sparse matrices.
The measured times for decomposing the $\bm Q / \bm T$ tensors for 7 different $L$ are listed in table \ref{tab_6_NMF_SVD}.
It can be observed that for \gls{svd} up to $L=6$, the computational time for both $\bm Q / \bm T$ tensors is less than 1 second.
Such an outcome is efficient for science and industry applications.
With $L=7$ a big jump in time for both $\bm Q / \bm T$ is found.
However, even after this increase, the decomposition took around 5 seconds, which still is acceptable.\newline 

\begin{table}
    \centering
    \begin{tabular}{c| c c |c c  }
        \textbf{$L$} &\textbf{SVD} $\bm Q$ & \textbf{NMF} $\bm Q$
        &\textbf{SVD} $\bm T$ & \textbf{NMF} $\bm T$\\
        \hline \\
        [-0.8em]
         $1$  & $2 \,\mathrm{e}{-4}$ s  & $64$ s & $8 \, \mathrm{e}{-05}$ s & $3 \, \mathrm{e}{-2}$  s \\

         $2$  & $1 \, \mathrm{e}{-4}$ s  & $8 \, \mathrm{e}{-2}$ s  & $1 \, \mathrm{e}{-4}$ s & $1$ h   \\

         $3$  & $2 \, \mathrm{e}{-4}$ s  & $10$ s  & $2 \, \mathrm{e}{-4}$ s & $0.1$ s \\

         $4$  & $4 \, \mathrm{e}{-3}$ s  & $20$ s & $7 \, \mathrm{e}{-3}$ s & $1.5$ h \\

         $5$  & $6 \, \mathrm{e}{-2}$ s  & $> 3$ h   & $3 \, \mathrm{e}{-2}$ s & -   \\

         $6$  & $0.4$ s & -     & $0.4$ s  & -    \\

         $7$  & $5.17$ s & -      & $4.52$ s & -  
    \end{tabular}
    \caption{Execution time for \emph{SLS} of \gls{nmf} and \gls{svd} for different $L$ }
    \label{tab_6_NMF_SVD}
\end{table}

Calculating $\bm Q$ with \gls{nmf} for $L=1$ already takes 64 seconds. 
This is more than \gls{svd} demanded for $L=7$. 
The $\bm T$ tensor on the other is much faster and is below a second.
However, as soon as $L=2$ is selected, $\bm T$ takes 1 full hour, $L=4$ more than 1 hour.
The table for \gls{nmf} is not filled, since running $\bm Q$ for $L=5$ was taking more than 3 hours, but still did not finish. 
Therefore, the time measurement was aborted. 
This behavior was expected since it was already mentioned in \cite{Max2021}.
Overall, the execution time for \gls{nmf} is not following a trend, e.g., computing $\bm T$ for $L=3$ is faster than for $L=2$ and $\bm Q$ for $L=4$ is faster than for $L=1$.
In other words, there is no obvious rule, on whether even a small $L$ could lead to hours of run time.\newline 

It can be concluded that \gls{svd} is much faster than \gls{nmf} and it also shows a clear trend, i.e. the computation time is expected to increase with $L$.
\gls{nmf} on the other hand first requires an appropriate mode number $r$, which most likely demands a parameter study.
However, even for a single \gls{nmf} solution, it can take hours.
With increasing $L$ the amount of run time is generally expected to increase, even though no clear rule can be defined.
Furthermore, it needs to be highlighted that \gls{nmf} was tested on a small model, where $n_{\beta,tr} = 7$. The author of this thesis experienced an additional increase in run time when $n_{\beta,tr}$ is selected higher. 
Also, executing \gls{nmf} on multiple dynamical systems or model configurations might become infeasible in terms of time.
Finally, with the implementation of \gls{svd}, the bottleneck in modeling $\bm Q / \bm T$ could be eliminated.


\subsection{Prediction quality}
\label{subsec_3_3_2_SVD_Quality}
In this subsection, the quality of the \gls{svd} and \gls{nmf} $\bm Q / \bm T$ predictions are evaluated. 
The used model configuration for this aim is \emph{SLS}.
First, only the $\bm Q$ output with \gls{svd} followed by \gls{nmf} shall be analyzed and compared. Then, the same is done for the $\bm T$ output.\newline


In order to see how many modes $r$ were chosen for \gls{svd} the two figures \ref{fig_54} and \ref{fig_55} are shown. 
It can be derived that with $r = 4$, $99 \%$ of the information content could be captured. The presented results are obtained for $\bm Q$ and $L =1$.\newline

\begin{figure}[!h]
    %\vspace{0.5cm}
    \begin{minipage}[h]{0.47\textwidth}
        \centering
        \includegraphics[width =\textwidth]
        {2_Figures/3_Task/2_Mod_CPE/10_lb_Q_Cumlative_E.pdf}
        \caption{\emph{SLS}, \gls{svd}, cumulative energy of $\bm Q$ for $L=1$}
        \label{fig_54}    
    \end{minipage}
    \hfill
    \begin{minipage}{0.47\textwidth}
        \centering
        \includegraphics[width =\textwidth]
        {2_Figures/3_Task/2_Mod_CPE/11_lb_Q_Sing_Val.pdf}
        \caption{\emph{SLS}, \gls{svd}, singular values of $\bm Q$ for $L=1$}
        \label{fig_55}    
    \end{minipage}
\end{figure}

Figures \ref{fig_56} to \ref{fig_58} depict the original $\bm{Q}(\beta_{unseen} = 28.5)$, which is generated with CNM, the \gls{cnmc} predicted $\bm{\tilde{Q}}(\beta_{unseen} = 28.5)$ and their deviation $| \bm{Q}(\beta_{unseen} = 28.5) -  \bm{\tilde{Q}}(\beta_{unseen} = 28.5) |$, respectively.
In the graphs, the probabilities to move from centroid $c_p$ to $c_j$ are indicated. 
Contrasting figure \ref{fig_56} and \ref{fig_57} exhibits barely noticeable differences.
For highlighting present deviations, the direct comparison between the \gls{cnm} and \gls{cnmc} predicted $\bm Q$ tensors is given in figure \ref{fig_58}.
It can be observed that the highest value is $max( \bm{Q}(\beta_{unseen} = 28.5) -  \bm{\tilde{Q}}(\beta_{unseen} = 28.5) |) \approx 0.0697 \approx 0.07$. 
Note that all predicted $\bm Q$ and $\bm T$ tensors are obtained with \gls{rf} as the regression model.
\newline

\begin{figure}[!h]
    %\vspace{0.5cm}
    \begin{subfigure}[h]{0.5\textwidth}
        \centering
        \caption{Original $\bm{Q}(\beta_{unseen} = 28.5)$}
        \includegraphics[width =\textwidth]
        {2_Figures/3_Task/2_Mod_CPE/12_lb_0_Q_Orig_28.5.pdf}
        \label{fig_56}    
    \end{subfigure}
    \hfill
    \begin{subfigure}{0.5\textwidth}
        \centering
        \caption{\gls{cnmc} predicted $\bm{\tilde{Q}}(\beta_{unseen} = 28.5)$ }
        \includegraphics[width =\textwidth]
        {2_Figures/3_Task/2_Mod_CPE/13_lb_2_Q_Aprox_28.5.pdf}
        \label{fig_57}    
    \end{subfigure}

    \smallskip
    \centering
    \begin{subfigure}{0.7\textwidth}
        \caption{Deviation $| \bm{Q}(\beta_{unseen}) -  \bm{\tilde{Q}}(\beta_{unseen}) |$ }
        \includegraphics[width =\textwidth]
        {2_Figures/3_Task/2_Mod_CPE/14_lb_4_Delta_Q_28.5.pdf}
        \label{fig_58}       
    \end{subfigure}
    \vspace{-0.3cm}
    \caption{\emph{SLS}, \gls{svd}, original $\bm{Q}(\beta_{unseen} = 28.5)$ , \gls{cnmc} predicted $\bm{\tilde{Q}}(\beta_{unseen} = 28.5)$ and  deviation $| \bm{Q}(\beta_{unseen} = 28.5) -  \bm{\tilde{Q}}(\beta_{unseen} = 28.5) |$ for $L=1$}
    \label{fig_58_Full}       
\end{figure}

The same procedure shall now be performed with NMF. 
The results are depicted in figures \ref{fig_59} and \ref{fig_60}.
Note that the original \gls{cnm} $\bm{Q}(\beta_{unseen} = 28.5)$ does not change, thus figure \ref{fig_56} can be reused.  
By exploiting figure \ref{fig_61}, it can be observed that the highest deviation for the \gls{nmf} version is $max( \bm{Q}(\beta_{unseen} = 28.5) -  \bm{\tilde{Q}}(\beta_{unseen} = 28.5) |) \approx 0.0699  \approx 0.07$.
The maximal error of \gls{nmf} $(\approx 0.0699)$ is slightly higher than that of \gls{svd}  $(\approx 0.0697)$. 
Nevertheless, both methods have a very similar maximal error and seeing visually other significant differences is hard.
\newline

\begin{figure}[!h]
    %\vspace{0.5cm}
    \begin{subfigure}[h]{0.5\textwidth}
        \centering
        \caption{\gls{cnmc} predicted $\bm{\tilde{Q}}(\beta_{unseen} = 28.5)$ }
        \includegraphics[width =\textwidth]
        {2_Figures/3_Task/2_Mod_CPE/15_lb_2_Q_Aprox_28.5.pdf}
        \label{fig_59}    
    \end{subfigure}
    \hfill
    \begin{subfigure}{0.5\textwidth}
        \caption{Deviation $| \bm{Q}(\beta_{unseen} ) -  \bm{\tilde{Q}}(\beta_{unseen} ) |$}
        \includegraphics[width =\textwidth]
        {2_Figures/3_Task/2_Mod_CPE/16_lb_4_Delta_Q_28.5.pdf}
        \label{fig_60}    
    \end{subfigure}
    \vspace{-0.3cm}
    \caption{\emph{SLS}, \gls{nmf}, \gls{cnmc} predicted $\bm{\tilde{Q}}(\beta_{unseen} = 28.5)$ and deviation $| \bm{Q}(\beta_{unseen} = 28.5) -  \bm{\tilde{Q}}(\beta_{unseen} = 28.5) |$ for $L=1$}
\end{figure}

In order to have a quantifiable error value, the Mean absolute error (MAE) following equation \eqref{eq_23} is leveraged.
The MAE errors for \gls{svd} and \gls{nmf} are  $MAE_{SVD} = 0.002 580 628$ and $MAE_{NMF} = 0.002 490 048$, respectively.
\gls{nmf} is slightly better than \gls{svd} with $ MAE_{SVD}  -  MAE_{NMF} \approx 1 \mathrm{e}{-4}$, which can be considered to be negligibly small. 
Furthermore, it must be stated that \gls{svd} was only allowed to use $r_{SVD} = 4$ modes, due to the $99 \%$ energy demand, whereas \gls{nmf} used $r_{NMF} = 9$ modes.
Given that \gls{svd} is stable in computational time, i.e., it is not assumed that for low $L$, the computational cost scales up to hours, \gls{svd} is the clear winner for this single comparison. \newline


For the sake of completeness, the procedure shall be conducted once  as well for the $\bm T$ tensor.
For this purpose figures \ref{fig_61} to \ref{fig_65} shall be considered.
It can be inspected that the maximal errors for \gls{svd} and \gls{nmf} are $max( \bm{T}(\beta_{unseen} = 28.5) -  \bm{\tilde{T}}(\beta_{unseen} = 28.5) |) \approx 0.126 $ and
 $max( \bm{T}(\beta_{unseen} = 28.5) -  \bm{\tilde{T}}(\beta_{unseen} = 28.5) | ) \approx 0.115 $, respectively. 
The MAE errors are, $MAE_{SVD} = 0.002 275 379 $ and $MAE_{NMF} = 0.001 635 510$. 
\gls{nmf} is again slightly better than \gls{svd} with $ MAE_{SVD}  -  MAE_{NMF} \approx 6 \mathrm{e}{-4}$, which is a deviation of $\approx 0.06 \%$ and might also be considered as negligibly small. \newline


%------------------------------------- SVD T -----------------------------------
\begin{figure}[!h]
    \begin{subfigure}{0.5 \textwidth}
        \centering
        \caption{Original \gls{cnm} $\bm{T}(\beta_{unseen} = 28.5)$ }
        \includegraphics[width =\textwidth]
        {2_Figures/3_Task/2_Mod_CPE/17_lb_1_T_Orig_28.5.pdf}
        \label{fig_61}    
    \end{subfigure}
    \hfill
    \begin{subfigure}{.5 \textwidth}
        \centering
        \caption{\gls{cnmc} predicted $\bm{\tilde{T}}(\beta_{unseen} = 28.5)$}
        \includegraphics[width =\textwidth]
        {2_Figures/3_Task/2_Mod_CPE/18_lb_3_T_Aprox_28.5.pdf}
        \label{fig_62}    
    \end{subfigure}
    
    \smallskip
    \centering
    \begin{subfigure}{0.7\textwidth}
        \caption{Deviation $| \bm{T}(\beta_{unseen}) -  \bm{\tilde{T}}(\beta_{unseen}) |$}
        \includegraphics[width =\textwidth]
        {2_Figures/3_Task/2_Mod_CPE/19_lb_5_Delta_T_28.5.pdf}
        \label{fig_63}    
    \end{subfigure}
    \vspace{-0.3cm}
    \caption{\emph{SLS}, \gls{svd}, original $\bm{T}(\beta_{unseen} = 28.5)$, predicted $\bm{\tilde{T}}(\beta_{unseen} = 28.5)$ and deviation $| \bm{T}(\beta_{unseen} = 28.5) -  \bm{\tilde{T}}(\beta_{unseen} = 28.5) |$ for $L=1$}
\end{figure}


%------------------------------------- NMF T -----------------------------------
\begin{figure}[!h]
    %\vspace{0.5cm}
    \begin{subfigure}[h]{0.5\textwidth}
        \centering
        \caption{\gls{cnmc} predicted $\bm{\tilde{T}}(\beta_{unseen} = 28.5)$}
        \includegraphics[width =\textwidth]
        {2_Figures/3_Task/2_Mod_CPE/20_lb_3_T_Aprox_28.5.pdf}
        \label{fig_64}    
    \end{subfigure}
    \hfill
    \begin{subfigure}{0.5\textwidth}
        \centering
        \caption{Deviation $| \bm{T}(\beta_{unseen}) -  \bm{\tilde{T}}(\beta_{unseen}) |$}
        \includegraphics[width =\textwidth]
        {2_Figures/3_Task/2_Mod_CPE/21_lb_5_Delta_T_28.5.pdf}
        \label{fig_65}    
    \end{subfigure}
    \vspace{-0.3cm}
    \caption{\emph{SLS}, \gls{nmf}, \gls{cnmc} predicted $\bm{\tilde{T}}(\beta_{unseen} = 28.5)$ and deviation $| \bm{T}(\beta_{unseen} = 28.5) -  \bm{\tilde{T}}(\beta_{unseen} = 28.5) |$ for $L=1$}
\end{figure}

Additional MAE errors for different $L$ and $\beta{unseen}= 28.5,\, \beta{unseen}= 32.5$ are collected in table \ref{tab_7_NMF_SVD_QT}.
First, it can be outlined that regardless of the chosen method, \gls{svd} or \gls{nmf}, all encountered MAE errors are very small. 
Consequently, it can be recorded that \gls{cnmc} convinces with an overall well approximation of the $\bm Q / \bm T$ tensors.
Second, comparing \gls{svd} and \gls{nmf} through their respective MAE errors, it can be inspected that the deviation of both is mostly in the order of $ \mathcal{O} \approx 1 \mathrm{e}{-2}$. 
It is a difference in $\approx 0.1 \%$ and can again be considered to be insignificantly small.\newline 

Despite this, \gls{nmf} required the additional change given in equation \eqref{eq_33}, which did not apply to \gls{svd}. 
The transition time entries at the indexes where the probability is positive should be positive as well. Yet, this is not always the case when \gls{nmf} is executed. To correct that, these probability entries are manually set to zero.
This rule was also actively applied to the results presented above. 
Still, the outcome is very satisfactory, because the modeling errors are found to be small. 
\newline

\begin{table}[!h]
    \centering
    \begin{tabular}{c c| c c|  c c  }
        \textbf{$L$} &$\beta_{unseen}$
         & $\boldsymbol{MAE}_{\gls{svd}, \bm Q}$  
         &$\boldsymbol{MAE}_{\gls{nmf}, \bm Q}$ 
         & $\boldsymbol{MAE}_{\gls{svd}, \bm T}$  
         &$\boldsymbol{MAE}_{\gls{nmf}, \bm T}$ \\
        \hline \\
        [-0.8em]
         $1$  & $28.5$ 
         & $0.002580628 $ & $0.002490048$ 
         & $0.002275379 $ & $0.001635510$\\
         
         $1$  & $32.5$ 
         & $0.003544923$ & $0.003650155$ 
         & $0.011152145$ & $0.010690052$\\
         
         $2$  & $28.5$ 
         & $0.001823848$ & $0.001776276$ 
         & $0.000409955$ & $0.000371242$\\
         
         $2$  & $32.5$ 
         & $0.006381635$ & $0.006053059$ 
         & $0.002417142$ & $0.002368680$\\
         
         $3$  & $28.5$ 
         & $0.000369228$ & $0.000356817$ 
         & $0.000067680$ & $0.000062964$\\

         $3$  & $32.5$ 
         & $0.001462458$ & $0.001432738$ 
         & $0.000346298$ & $0.000343520$\\
         

         $4$  & $28.5$ 
         & $0.000055002$ & $0.000052682$ 
         & $0.000009420$ & $0.000008790$\\

         $4$  & $32.5$ 
         & $0.000215147$ & $0.000212329$ 
         & $0.000044509$ & $0.000044225$
    
    \end{tabular}
    \caption{\emph{SLS}, Mean absolute error for different $L$ and two $\beta_{unseen}$}
    \label{tab_7_NMF_SVD_QT}
\end{table}

\begin{equation}
    \begin{aligned}
        TGZ := \bm T ( \bm Q  > 0) \leq 0 \\
        \bm Q ( TGZ) := 0
    \end{aligned}
    \label{eq_33}
\end{equation}

In summary, both methods \gls{nmf} and \gls{svd} provide a good approximation of the $\bm Q / \bm T$ tensors.
The deviation between the prediction quality of both is negligibly small.
However, since \gls{svd} is much faster than \gls{nmf} and does not require an additional parameter study, the recommended decomposition method is \gls{svd}.
Furthermore, it shall be highlighted that \gls{svd} used only $r = 4$ modes for the $\bm Q$ case, whereas for \gls{nmf} $r=9$ were used. 
Finally, as a side remark, all the displayed figures and the MAE errors are generated and calculated with \gls{cnm}'s default implemented methods. 
\FloatBarrier
\section{Transition property regression models}
\label{sec_3_4_SVD_Regression}
In this section, the results of the 3 different regression methods, \glsfirst{rf}, AdaBoost and Gaussian Process (GP) are compared.
All the 3 regressors are implemented in \gls{cnmc} and can be selected via \emph{settings.py}.
The utilized model configuration is \emph{SLS} and the decomposition method is \gls{svd}.\newline 


First, it shall be noted that \gls{cnmc} also offers the possibility to apply \emph{pySindy}. 
However, \emph{pySindy} has struggled to represent the training data in the first place, thus it cannot be employed for predicting $\beta_{unseen}$.
The latter does not mean that \emph{pySindy} is not applicable for the construction of a surrogate model for the decomposed $\bm Q / \bm T$ modes, but rather that the selected candidate library was not powerful enough. 
Nevertheless, only results for the 3 initially mentioned regressors will be discussed.\newline 

In figures \ref{fig_66} to \ref{fig_71} the true (dashed) and the approximation (solid) of the first 4 $\bm Q / \bm T$ modes are shown for the methods RF, AdaBoost and GP, respectively.
To begin with, it can be noted that the mode behavior over different model parameter values $mod(\beta)$ is discontinuous, i.e., it exhibits spikes or sudden changes.
In figures \ref{fig_66} and \ref{fig_67} it can be observed that \gls{rf} reflects the actual behavior of $mod(\beta)$ quite well.
However, it encounters difficulties in capturing some spikes. 
AdaBoost on the other hand proves in figures \ref{fig_68} and \ref{fig_69} to represent the spikes better. 
Overall, AdaBoost outperforms \gls{rf} in mirroring training data. \newline

\begin{figure}[!h]
    %\vspace{0.5cm}
    \begin{subfigure}[h]{0.5 \textwidth}
        \centering
        \caption{$\bm Q$}
        \includegraphics[width =\textwidth]
        {2_Figures/3_Task/3_SVD_QT/0_model_Decomp_Regr_RF_More_Q.pdf}
        \label{fig_66}    
    \end{subfigure}
    \hfill
    \begin{subfigure}{0.5 \textwidth}
        \centering
        \caption{$\bm T$}
        \includegraphics[width =\textwidth]
        {2_Figures/3_Task/3_SVD_QT/1_model_Decomp_Regr_RF_More_T.pdf}
        \label{fig_67}    
    \end{subfigure}
    \vspace{-0.3cm}
    \caption{\emph{SLS}, \gls{svd}, $\bm Q / \bm T$ modes approximation with \gls{rf} for $L=1$}
\end{figure}

\begin{figure}[!h]
    %\vspace{0.5cm}
    \begin{subfigure}[h]{0.5 \textwidth}
        \centering
        \caption{$\bm Q$}
        \includegraphics[width =\textwidth]
        {2_Figures/3_Task/3_SVD_QT/2_model_Decomp_Regr_ABoost_More_Q.pdf}
        \label{fig_68}    
    \end{subfigure}
    \hfill
    \begin{subfigure}{0.5 \textwidth}
        \centering
        \caption{$\bm T$}
        \includegraphics[width =\textwidth]
        {2_Figures/3_Task/3_SVD_QT/3_model_Decomp_Regr_ABoost_More_T.pdf}
        \label{fig_69}    
    \end{subfigure}
    \vspace{-0.3cm}
    \caption{\emph{SLS}, \gls{svd}, $\bm Q / \bm T$ mode approximation with AdaBoost for $L=1$}
\end{figure}

\begin{figure}[!h]
    %\vspace{0.5cm}
    \begin{subfigure}[h]{0.5 \textwidth}
        \centering
        \caption{$\bm Q$}
        \includegraphics[width =\textwidth]
        {2_Figures/3_Task/3_SVD_QT/4_model_Decomp_Regr_GP_More_Q.pdf}
        \label{fig_70}    
    \end{subfigure}
    \hfill
    \begin{subfigure}{0.5 \textwidth}
        \centering
        \caption{$\bm T$}
        \includegraphics[width =\textwidth]
        {2_Figures/3_Task/3_SVD_QT/5_model_Decomp_Regr_GP_More_T.pdf}
        \label{fig_71}    
    \end{subfigure}
    \vspace{-0.3cm}
    \caption{\emph{SLS}, \gls{svd}, $\bm Q / \bm T$ mode approximation with GP for $L=1$}
\end{figure}

Gaussian Process (GP) is a very powerful method for regression. 
Often this is also true when reproducing $mod(\beta)$. 
However, there are cases where the performance is insufficient, as shown in figures \ref{fig_70} and \ref{fig_71}. 
Applying GP results in absolutely incorrect predicted tensors $\bm \tilde{Q}(\beta_{unseen}),\, \bm \tilde{T}(\beta_{unseen})$, where too many tensors entries are wrongly forced to zero. 
Therefore, $\bm \tilde{Q}(\beta_{unseen}),\, \bm \tilde{T}(\beta_{unseen})$ will eventually lead to an unacceptably high deviation from the original trajectory.
Consequently, the GP regression is not applicable for the decomposed $\bm Q / \bm T$ modes without further modification.\newline 

The two remaining regressors are \glsfirst{rf} and AdaBoost.
Although AdaBoost is better at capturing the true modal behavior $mod(\beta)$, there is no guarantee that it will always be equally better at predicting the modal behavior for unseen model parameter values $mod(\beta_{unseen})$.
In table \ref{tab_8_RF_ABoost} the MAE errors for different $L$ and $\beta_{unseen} = [\, 28.5,\, 32.5\,]$ are provided. 
Since the table exhibits much information, the results can also be read qualitatively through the graphs \ref{fig_72_QT_28} and \ref{fig_72_QT_32} for $\beta_{unseen} = 28.5$ and $\beta_{unseen} = 32.5$, respectively.
For the visual inspection, it is important to observe the order of the vertical axis scaling.
It can be noted that the MAE errors themselves and the deviation between the \gls{rf} and AdaBoost MAE errors are very low.
Thus, it can be stated that \gls{rf} as well ad AdaBoost are both well-suited regressors.\newline 


\begin{table}[!h]
    \centering
    \begin{tabular}{c c| c c|  c c  }
        \textbf{$L$} &$\beta_{unseen}$
         & $\boldsymbol{MAE}_{RF, \bm Q}$  
         &$\boldsymbol{MAE}_{AdaBoost, \bm Q}$ 
         & $\boldsymbol{MAE}_{RF, \bm T}$  
         &$\boldsymbol{MAE}_{AdaBoost, \bm T}$ \\
        \hline \\
        [-0.8em]
         $1$  & $28.5$ 
         & $0.002580628 $ & $0.002351781$ 
         & $0.002275379 $ & $0.002814208$\\
         
         $1$  & $32.5$ 
         & $0.003544923$ & $0.004133114$ 
         & $0.011152145$ & $0.013054876$\\
         
         $2$  & $28.5$ 
         & $0.001823848$ & $0.001871858$ 
         & $0.000409955$ & $0.000503748$\\
         
         $2$  & $32.5$ 
         & $0.006381635$ & $0.007952153$ 
         & $0.002417142$ & $0.002660403$\\
         
         $3$  & $28.5$ 
         & $0.000369228$ & $0.000386292$ 
         & $0.000067680$ & $0.000082808$\\

         $3$  & $32.5$ 
         & $0.001462458$ & $0.001613434$ 
         & $0.000346298$ & $0.000360097$\\
         

         $4$  & $28.5$ 
         & $0.000055002$ & $0.000059688$ 
         & $0.000009420$ & $0.000011500$\\

         $4$  & $32.5$ 
         & $0.000215147$ & $0.000230404$ 
         & $0.000044509$ & $0.000046467$\\

         $5$  & $28.5$ 
         & $0.000007276$ & $0.000007712$ 
         & $0.000001312$ & $0.000001600$\\

         $5$  & $32.5$ 
         & $0.000028663$ & $0.000030371$ 
         & $0.000005306$ & $0.000005623$\\

         $6$  & $28.5$ 
         & $0.000000993$ & $0.000052682$ 
         & $0.000000171$ & $0.000000206$\\

         $6$  & $32.5$ 
         & $0.000003513$ & $0.000003740$ 
         & $0.000000629$ & $0.000000668$\\

         $7$  & $28.5$ 
         & $0.000000136$ & $0.000000149$ 
         & $0.000000023$ & $0.000000031$ \\

         $7$  & $32.5$ 
         & $0.000000422$ & $0.000000454$ 
         & $0.000000078$ & $0.000000082$

    
    \end{tabular}
    \caption{\emph{SLS}, Mean absolute error for comparing \gls{rf} and AdaBoost different $L$ and two $\beta_{unseen}$}
    \label{tab_8_RF_ABoost}
\end{table}

\begin{figure}[!h]
    %\vspace{0.5cm}
    \begin{subfigure}[h]{0.5 \textwidth}
        \centering
        \caption{$\bm Q$}
        \includegraphics[width =\textwidth]
        {2_Figures/3_Task/3_SVD_QT/6_Q_28_5.pdf}
        \label{fig_72_Q_28}    
    \end{subfigure}
    \hfill
    \begin{subfigure}{0.5 \textwidth}
        \centering
        \caption{$\bm T$}
        \includegraphics[width =\textwidth]
        {2_Figures/3_Task/3_SVD_QT/7_T_28_5.pdf}
        \label{fig_72_T_28}    
    \end{subfigure}
    \vspace{-0.3cm}
    \caption{\emph{SLS}, Mean absolute error for comparing \gls{rf} and AdaBoost different $L$ and $\beta_{unseen} = 28.5$}
    \label{fig_72_QT_28}    
\end{figure}

\begin{figure}[!h]
    %\vspace{0.5cm}
    \begin{subfigure}[h]{0.5 \textwidth}
        \centering
        \caption{$\bm Q$}
        \includegraphics[width =\textwidth]
        {2_Figures/3_Task/3_SVD_QT/8_Q_32_5.pdf}
        \label{fig_72_Q_32}    
    \end{subfigure}
    \hfill
    \begin{subfigure}{0.5 \textwidth}
        \centering
        \caption{$\bm T$}
        \includegraphics[width =\textwidth]
        {2_Figures/3_Task/3_SVD_QT/9_T_32_5.pdf}
        \label{fig_72_T_32}    
    \end{subfigure}
    \vspace{-0.3cm}
    \caption{\emph{SLS}, Mean absolute error for comparing \gls{rf} and AdaBoost different $L$ and $\beta_{unseen} = 32.5$}
    \label{fig_72_QT_32}    
\end{figure}


In summary, the following can be said, \gls{rf} and AdaBoost are both performing well in regression. Furthermore, no clear winner between the two regressors can be detected. 
The third option GP is dismissed as it sometimes has unacceptably low regression performance. 
Finally, there is the possibility to use \emph{pySindy}, however, for that, an appropriate candidate library must be defined.
\FloatBarrier
\section{CNMc predictions}
\label{sec_3_5_Pred}
In this section, some representative outputs for the \gls{cnmc} predicted trajectories shall be discussed.
For that, first, the quality measurement abilities implemented in \gls{cnmc} are elaborated.
Next, the model \emph{SLS} is analyzed and explained in detail in the subsection \ref{subsec_3_5_1_SLS}.
Finally, the outcome for other models shall be presented briefly in subsection \ref{subsec_3_5_2_Models}.\newline

There are several methods implemented in \gls{cnmc} to assess the quality of the predicted trajectories.
The first one is the autocorrelation, which will be calculated for all $\vec{\beta}_{unseen}$ and all provided $\vec{L}$, for the true, \gls{cnm} and \gls{cnmc} predicted trajectories.
As usual, the output is plotted and saved as HTML files for a feature-rich visual inspection.
For qualitative assessment, the MAE errors are calculated for all $\vec{\beta}_{unseen}$ and $\vec{L}$ for two sets.
The first set consists of the MAE errors between the true and the \gls{cnm} predicted trajectories.
The second set contains the MAE errors between the true and the \gls{cnmc} predicted trajectories.
Both sets are plotted as MAE errors over $L$ and stored as HTML files.
Furthermore, the one $L$ value which exhibits the least MAE error is printed in the terminal and can be found in the log file as well. \newline

The second technique is the \gls{cpd}, which will also be computed for all the 3 trajectories, i.e., true, \gls{cnm} and \gls{cnmc} predicted trajectories.
The \gls{cpd} depicts the probability of being at one centroid $c_i$.
For each $\vec{\beta}_{unseen}$ and all $L$ the \gls{cpd} is plotted and saved.
The third method displays all the 3 trajectories in the state space.
Moreover, the trajectories are plotted as 2-dimensional graphs, i.e., each axis as a subplot over the time $t$.
The final method calculates the MAE errors of the $\bm Q / \bm T$ tensors for all $L$.\newline 

The reason why more than one quality measurement method is integrated into \gls{cnmc} is that \gls{cnmc} should be able to be applied to, among other dynamical systems, chaotic systems.
The motion of the Lorenz system \eqref{eq_6_Lorenz} is not as complex as of the, e.g., the \emph{Four Wing} \eqref{eq_10_4_Wing}.
Nevertheless, the Lorenz system already contains quasi-random elements, i.e., the switching from one ear to the other cannot be captured exactly with a surrogate mode. However, the characteristic of the Lorenz system and other chaotic dynamical systems as well can be replicated.
In order to prove the latter, more than one method to measure the prediction quality is required.
\subsection{Assessment of SLS}
\label{subsec_3_5_1_SLS}
In this subsection, the prediction capability for the \emph{SLS} will be analyzed in detail. All the presented output is generated with \gls{svd} as the decomposition method and \gls{rf} as the $\bm Q / \bm T$ regressor.\newline

The final objective of \gls{cnmc} is to capture the characteristics of the original trajectory.
However, it is important to outline that \gls{cnmc} is trained with the \gls{cnm} predicted trajectories. 
Thus, the outcome of \gls{cnmc} highly depends on the ability of \gls{cnm} to represent the original data. 
Consequently, \gls{cnmc} can only be as effective as \gls{cnm} is in the first place,  with the approximation of the true data.
Figures \ref{fig_72} and \ref{fig_73} show the true, \gls{cnm} and \gls{cnmc} predicted trajectories and a focused view on the \gls{cnm} and \gls{cnmc} trajectories, respectively.
The output was generated for $\beta_{unseen} = 28.5$ and $L =1$.
First, it can be observed that \gls{cnm} is not able to capture the full radius of the Lorenz attractor.
This is caused by the low chosen number of centroids $K=10$.
Furthermore, as mentioned at the beginning of this chapter, the goal is not to replicate the true data one-to-one, but rather to catch the significant behavior of any dynamic system.
With the low number of centroids $K$, \gls{cnm} extracts the characteristics of the Lorenz system well.
Second, the other aim for \gls{cnmc} is to match the \gls{cnm} data as closely as possible.
Both figures \ref{fig_72} and \ref{fig_73} prove that \gls{cnmc} has fulfilled its task very well. \newline

\begin{figure}[!h]
    \begin{subfigure}{0.5\textwidth}
        \centering
        \caption{True, \gls{cnm} and \gls{cnmc} predicted trajectories}
        \includegraphics[width =\textwidth]
        {2_Figures/3_Task/4_SLS/0_lb_28.5_All.pdf}
        \label{fig_72}
    \end{subfigure}
    \hfill
    \begin{subfigure}{0.5\textwidth}
        \centering
        \caption{\gls{cnm} and \gls{cnmc} predicted trajectories}
        \includegraphics[width =\textwidth]
        {2_Figures/3_Task/4_SLS/1_lb_28.5.pdf}
        \label{fig_73}
    \end{subfigure}
    \vspace{-0.3cm}
    \caption{\emph{SLS}, $\beta_{unseen}=28.5,\, L=1$, true, \gls{cnm} and \gls{cnmc} predicted trajectories} 
\end{figure}


A close-up of the movement of the different axes is shown in the picture \ref{fig_74}.
Here, as well, the same can be observed as described above. Namely, the predicted \gls{cnmc} trajectory is not a one-to-one reproduction of the original trajectory.
However, the characteristics, i.e., the magnitude of the motion in all 3 directions (x, y, z) and the shape of the oscillations, are very similar to the original trajectory.
Note that even though the true and predicted trajectories are utilized to assess, whether the characteristical behavior could be extracted, a single evaluation based on the trajectories is not sufficient and often not advised or even possible.
In complex systems, trajectories can change rapidly while dynamical features persist \cite{Fernex2021a}. 
In \gls{cnmc} the predicted trajectories are obtained through the \gls{cnm} propagation, which itself is based on a probabilistic model, i.e. the $\bm Q$ tensor. 
Thus, matching full trajectories becomes even more unrealistic. 
The latter two statements highlight yet again that more than one method of measuring quality is needed. 
To further support the generated outcome the autocorrelation and \gls{cpd} in figure \ref{fig_75} and \ref{fig_76}, respectively, shall be considered.
It can be inspected that the \gls{cnm} and \gls{cnmc} autocorrelations are matching the true autocorrelation in the shape favorably well.
Nonetheless, the degree of reflecting the magnitude fully decreases quite fast.
Considering the \gls{cpd}, it can be recorded that the true \gls{cpd} could overall be reproduced satisfactorily.\newline 

\begin{figure}[!h]
    \centering
    \includegraphics[width =0.75\textwidth]
    {2_Figures/3_Task/4_SLS/2_lb_28.5_3V_All.pdf}
    \caption{\emph{SLS}, $\beta_{unseen}=28.5, \, L=1$, true, \gls{cnm} and \gls{cnmc} predicted trajectories as 2d graphs } 
    \label{fig_74}
\end{figure}


\begin{figure}[!h]
    \begin{subfigure}{0.5\textwidth}
        \centering
        \caption{autocorrelation} 
        \includegraphics[width =\textwidth]
        {2_Figures/3_Task/4_SLS/3_lb_3_all_28.5.pdf}
        \label{fig_75}
    \end{subfigure}
    \hfill
    \begin{subfigure}{0.5\textwidth}
        \centering
        \caption{\gls{cpd}} 
        \includegraphics[width =\textwidth]
        {2_Figures/3_Task/4_SLS/4_lb_28.5.pdf}
        \label{fig_76}
    \end{subfigure}
    \vspace{-0.3cm}
    \caption{\emph{SLS}, $\beta_{unseen}= 28.5, \, L =1$, autocorrelation  and \gls{cpd} for true, \gls{cnm} and \gls{cnmc} predicted trajectories} 
\end{figure}


To illustrate the influence of $L$, figure \ref{fig_77} shall be viewed.
It depicts the MAE error for the true and \gls{cnmc} predicted trajectories for $\beta_{unseen}= [\, 28.5,\, 32.5 \, ]$ with $L$ up to 7.
It can be observed that the choice of $L$ has an impact on the prediction quality measured by autocorrelation.
For $\beta_{unseen}=28.5$ and $\beta_{unseen}=32.5$, the optimal $L$ values are $L = 2$ and $L = 7$, respectively. To emphasize it even more that with the choice of $L$ the prediction quality can be regulated, figure \ref{fig_78} shall be considered.
It displays the 3 autocorrelations for $L = 7$. 
Matching the shape of the true autocorrelation was already established with $L =1$ as shown in figure \ref{fig_75}. In addition to that, $L=7$ improves by matching the true magnitude.
Finally, it shall be mentioned that similar results have been accomplished with other $K$ tested values, where the highest value was $K =50$.

\begin{figure}[!h]
    \begin{minipage}{0.47\textwidth}
        \centering
        \includegraphics[width =\textwidth]
        {2_Figures/3_Task/4_SLS/5_lb_1_Orig_CNMc.pdf}
        \caption{\emph{SLS}, MAE error for true and \gls{cnmc} predicted autocorrelations for $\beta_{unseen}= [\, 28.5,$ $32.5 \, ]$ and different values of $L$} 
        \label{fig_77}
    \end{minipage}
        \hfill
        \begin{minipage}{0.47\textwidth}
            \centering
            \includegraphics[width =\textwidth]
            {2_Figures/3_Task/4_SLS/6_lb_3_all_32.5.pdf}
            \caption{\emph{SLS}, $\beta_{unseen}=32.5, \, L=7$, \gls{cnm} and \gls{cnmc} predicted autocorrelation } 
            \label{fig_78}
        \end{minipage}
    \end{figure}
\FloatBarrier

\subsection{Results of further dynamical systems}
\label{subsec_3_5_2_Models}
In this subsection, the \gls{cnmc} prediction results for other models will be displayed. 
The chosen dynamical systems with their configurations are the following.
% ==============================================================================
\begin{enumerate}
    \item  \emph{FW50}, based on the \emph{Four Wing} set of equations \eqref{eq_10_4_Wing} with $K=50, \, \vec{\beta }_{tr} = [\, \beta_0 = 8 ; \, \beta_{end} = 11 \,], \, n_{\beta, tr} = 13$.

    \item  \emph{Rössler15}, based on the \emph{Rössler} set of equations \eqref{eq_7_Ross} with $K=15, \, \vec{\beta }_{tr} = [\, \beta_0 = 6 ; \, \beta_{end} = 13 \,], \, n_{\beta, tr} = 15$.

    \item  \emph{TS15}, based on the \emph{Two Scroll} set of equations \eqref{eq_9_2_Scroll} with $K=15, \, \vec{\beta }_{tr} = [\, \beta_0 = 5 ; \, \beta_{end} = 12 \,], \, n_{\beta, tr} = 15$.    
\end{enumerate}
All the presented outputs were generated with \gls{svd} as the decomposition method and \gls{rf} as the $\bm Q / \bm T$ regressor.
Furthermore, the B-spline interpolation in the propagation step of \gls{cnm} was replaced with linear interpolation. 
The B-spline interpolation was originally utilized for smoothing the motion between two centroids. 
However, it was discovered for a high number of $K$, the B-spline interpolation is not able to reproduce the motion between two centroids accurately, but rather would impose unacceptable high deviations or oscillations into the predictions. 
This finding is also mentioned in \cite{Max2021} and addressed as one of \emph{ first CNMc's} limitations.  
Two illustrative examples of the unacceptable high deviations caused by the B-spline interpolation are given in figures \ref{fig_82_Traject} and \ref{fig_82_Autocorr}. 
The results are obtained for \emph{LS20} for $\beta = 31.75$ and $\beta = 51.75$ with $L=3$. 
In figures \ref{fig_82_Traj_B} and \ref{fig_83_Traj_B} it can be inspected that the B-spline interpolation has a highly undesired impact on the predicted trajectories.
In Contrast to that, in figures, \ref{fig_82_Traj_L} and \ref{fig_83_Traj_L}, where linear interpolation is utilized, no outliers are added to the predictions.
The impact of the embedded outliers, caused by the B-spline interpolation, on the autocorrelation is depicted in figures \ref{fig_82_Auto_B} and \ref{fig_83_Auto_B}.
The order of the deviation between the true and the \gls{cnmc} predicted autocorrelation can be grasped by inspecting the vertical axis scale.
Comparing it with the linear interpolated autocorrelations, shown in figures \ref{fig_82_Auto_L} and \ref{fig_83_Auto_L}, it can be recorded that the deviation between the true and predicted autocorrelations is significantly lower than in the B-spline interpolation case.
\newline 

Nevertheless, it is important to highlight that the B-spline interpolation is only a tool for smoothing the motion between two centroids. 
The quality of the modeled $\bm Q / \bm T$ cannot be assessed directly by comparing the trajectories and the autocorrelations.
To stress that the \gls{cpd} in figure \ref{fig_82_CPD_B} and \ref{fig_83_CPD_B} shall be considered.
It can be observed that \gls{cpd} does not represent the findings of the autocorrelations, i.e., the true and predicted behavior agree acceptably overall. 
This is because the type of interpolation has no influence on the modeling of the probability tensor $\bm Q$.
Thus, the outcome with the B-spline interpolation should not be regarded as an instrument that enables making assumptions about the entire prediction quality of \gls{cnmc}.  The presented points underline again the fact that more than one method should be considered to evaluate the prediction quality of \gls{cnmc}.
\newline


\begin{figure}[!h]
    \begin{subfigure}{0.5\textwidth}
        \centering
        \caption{Trajectories, B-spline, $\beta_{unseen} = 31.75$ }
        \includegraphics[width =\textwidth]
        {2_Figures/3_Task/5_Models/18_lb_31.75_All.pdf}
        \label{fig_82_Traj_B}
    \end{subfigure}
    \hfill
    \begin{subfigure}{0.5\textwidth}
        \centering
        \caption{Trajectories, B-spline, $\beta_{unseen} = 51.75$}
        \includegraphics[width =\textwidth]
        {2_Figures/3_Task/5_Models/19_lb_51.75_All.pdf}
        \label{fig_83_Traj_B}
    \end{subfigure}
    
    % ------------- Linear ----------------------
    \smallskip
    \begin{subfigure}{0.5\textwidth}
        \centering
        \caption{Trajectories, linear, $\beta_{unseen} = 31.75$ }
        \includegraphics[width =\textwidth]
        {2_Figures/3_Task/5_Models/24_lb_31.75_All.pdf}
        \label{fig_82_Traj_L}
    \end{subfigure}
    \hfill
    \begin{subfigure}{0.5\textwidth}
        \centering
        \caption{Trajectories, linear, $\beta_{unseen} = 51.75$}
        \includegraphics[width =\textwidth]
        {2_Figures/3_Task/5_Models/25_lb_51.75_All.pdf}
        \label{fig_83_Traj_L}
    \end{subfigure}
    \vspace{-0.3cm}
    \caption{Illustrative undesired oscillations cased by the B-spline interpolation and its impact on the predicted trajectory contrasted with linear interpolation, \emph{LS20}, $\beta = 31.75$ and $\beta =51.75$, $L=3$}
    \label{fig_82_Traject}
\end{figure}

%----------------------------------- AUTOCOR -----------------------------------

\begin{figure}[!h]
    \begin{subfigure}{0.5\textwidth}
        \centering
        \caption{Autocorrelations, B-spline, $\beta = 31.75$ }
        \includegraphics[width =\textwidth]
        {2_Figures/3_Task/5_Models/20_lb_3_all_31.75.pdf}
        \label{fig_82_Auto_B}
    \end{subfigure}
    \hfill
    \begin{subfigure}{0.5\textwidth}
        \centering
        \caption{Autocorrelations, B-spline, $\beta_{unseen} = 51.75$}
        \includegraphics[width =\textwidth]
        {2_Figures/3_Task/5_Models/21_lb_3_all_51.75.pdf}
        \label{fig_83_Auto_B}
    \end{subfigure}
    
    \smallskip
    % ------------- LINEAR ----------------------
    \begin{subfigure}{0.5\textwidth}
        \centering
        \caption{Autocorrelations, linear, $\beta = 31.75$ }
        \includegraphics[width =\textwidth]
        {2_Figures/3_Task/5_Models/26_lb_3_all_31.75.pdf}
        \label{fig_82_Auto_L}
    \end{subfigure}
    \hfill
    \begin{subfigure}{0.5\textwidth}
        \centering
        \caption{Autocorrelations, linear, $\beta_{unseen} = 51.75$}
        \includegraphics[width =\textwidth]
        {2_Figures/3_Task/5_Models/27_lb_3_all_51.75.pdf}
        \label{fig_83_Auto_L}
    \end{subfigure}
    \vspace{-0.3cm}
    \caption{Illustrative undesired oscillations cased by the B-spline interpolation and its impact on the predicted autocorrelations contrasted with linear interpolation, \emph{LS20}, $\beta = 31.75$ and $\beta =51.75$, $L=3$}
    \label{fig_82_Autocorr}
\end{figure}
    
\begin{figure}[!h]
    % ------------- CPD ----------------------
    \begin{subfigure}{0.5\textwidth}
        \centering
        \caption{\gls{cpd}, $\beta = 31.75$ }
        \includegraphics[width =\textwidth]
        {2_Figures/3_Task/5_Models/22_lb_31.75.pdf}
        \label{fig_82_CPD_B}
    \end{subfigure}
    \hfill
    \begin{subfigure}{0.5\textwidth}
        \centering
        \caption{\gls{cpd}, $\beta_{unseen} = 51.75$}
        \includegraphics[width =\textwidth]
        {2_Figures/3_Task/5_Models/23_lb_51.75.pdf}
        \label{fig_83_CPD_B}
    \end{subfigure}
    \vspace{-0.3cm}
    \caption{Illustrative the B-spline interpolation and its impact on the \glspl{cpd}, \emph{LS20}, $\beta = 31.75$ and $\beta =51.75$, $L=3$}
\end{figure}

\FloatBarrier
The results generated with the above mentioned linear interpolation for  \emph{FW50}, \emph{Rössler15} and \emph{TS15} are depicted in figures \ref{fig_79} to \ref{fig_81}, respectively. 
Each of them consists of an illustrative trajectory, 3D and 2D trajectories, the autocorrelations, the \gls{cpd} and the MAE error between the true and \gls{cnmc} predicted trajectories for a range of $\vec{L}$ and some $\vec{\beta}_{unseen}$.
The illustrative trajectory includes arrows, which provide additional information.
First, the direction of the motion, then the size of the arrows represents the velocity of the system. Furthermore, the change in the size of the arrows is equivalent to a change in the velocity, i.e., the acceleration.
Systems like the \emph{TS15} exhibit a fast change in the size of the arrows, i.e., the acceleration is nonlinear. 
The more complex the behavior of the acceleration is, the more complex the overall system becomes.
The latter statement serves to emphasize that \gls{cnmc} can be applied not only to rather simple systems such as the Lorenz attractor \cite{lorenz1963deterministic}, but also to more complex systems such as the \emph{FW50} and \emph{TS15}.\newline 

All in all, the provided results for the 3 systems are very similar to those already explained in the previous subsection \ref{subsec_3_5_1_SLS}.
Thus, the results presented are for demonstration purposes and will not be discussed further.
However, the 3 systems also have been calculated with different values for $K$. 
For \emph{FW50}, the range of $\vec{K}= [\, 15, \, 30, \, 50 \, ]$ was explored with the finding that the influence of $K$ remained quite small.
For \emph{Rössler15} and \emph{TS15}, the ranges were chosen as $\vec{K}= [\, 15, \, 30, \, 100\,]$ and $\vec{K}= [\, 15, \, 75 \,]$, respectively.
The influence of $K$ was found to be insignificant also for the latter two systems.
% ==============================================================================
% ======================= FW50 =================================================
% ==============================================================================
\begin{figure}[!h]
    \begin{subfigure}{0.5\textwidth}
        \centering
        \caption{Illustrative trajectory $\beta = 9$ }
        \includegraphics[width =\textwidth]
        {2_Figures/3_Task/5_Models/0_lb_9.000.pdf}
    \end{subfigure}
    \hfill
    \begin{subfigure}{0.5\textwidth}
        \centering
        \caption{Trajectories, $\beta_{unseen} = 8.1$}
        \includegraphics[width =\textwidth]
        {2_Figures/3_Task/5_Models/1_lb_8.1_All.pdf}
    \end{subfigure}

    \smallskip
    \begin{subfigure}{0.5\textwidth}
        \centering
        \caption{2D-trajectories, $\beta_{unseen} = 8.1$}
        \includegraphics[width =\textwidth]
        {2_Figures/3_Task/5_Models/2_lb_8.1_3V_All.pdf}
    \end{subfigure}
    \hfill
    \begin{subfigure}{0.5\textwidth}
        \centering
        \caption{Autocorrelations, $\beta_{unseen} = 8.1$}
        \includegraphics[width =\textwidth]
        {2_Figures/3_Task/5_Models/3_lb_3_all_8.1.pdf}
    \end{subfigure}
    
    
    \smallskip
    \begin{subfigure}{0.5\textwidth}
        \centering
        \caption{\gls{cpd}, $\beta_{unseen} = 8.1$}
        \includegraphics[width =\textwidth]
        {2_Figures/3_Task/5_Models/4_lb_8.1.pdf}
    \end{subfigure}
    \hfill
    \begin{subfigure}{0.5\textwidth}
        \centering
        \caption{Autocorrelations $MAE(L,\, \beta_{unseen})$}
        \includegraphics[width =\textwidth]
        {2_Figures/3_Task/5_Models/5_lb_1_Orig_CNMc.pdf}
    \end{subfigure}
    \vspace{-0.3cm}
    \caption{Results for \emph{FW50}, $\beta_{unseen} = 8.1, \, L= 2$}
    \label{fig_79}
\end{figure}
% ==============================================================================
% ======================= FW50 =================================================
% ==============================================================================

% ==============================================================================
% ======================= Rossler 15 ===========================================
% ==============================================================================
\begin{figure}[!h]
    \begin{subfigure}{0.5\textwidth}
        \centering
        \caption{Illustrative trajectory $\beta = 7.5$ }
        \includegraphics[width =\textwidth]
        {2_Figures/3_Task/5_Models/6_lb_7.500.pdf}
    \end{subfigure}
    \hfill
    \begin{subfigure}{0.5\textwidth}
        \centering
        \caption{Trajectories, $\beta_{unseen} = 9.6$}
        \includegraphics[width =\textwidth]
        {2_Figures/3_Task/5_Models/7_lb_9.6_All.pdf}
    \end{subfigure}

    \smallskip
    \begin{subfigure}{0.5\textwidth}
        \centering
        \caption{2D-trajectories, $\beta_{unseen} = 9.6$}
        \includegraphics[width =\textwidth]
        {2_Figures/3_Task/5_Models/8_lb_9.6_3V_All.pdf}
    \end{subfigure}
    \hfill
    \begin{subfigure}{0.5\textwidth}
        \centering
        \caption{Autocorrelations, $\beta_{unseen} = 9.6$}
        \includegraphics[width =\textwidth]
        {2_Figures/3_Task/5_Models/9_lb_3_all_9.6.pdf}
    \end{subfigure}
    
    
    \smallskip
    \begin{subfigure}{0.5\textwidth}
        \centering
        \caption{\gls{cpd}, $\beta_{unseen} = 9.6$}
        \includegraphics[width =\textwidth]
        {2_Figures/3_Task/5_Models/10_lb_9.6.pdf}
    \end{subfigure}
    \hfill
    \begin{subfigure}{0.5\textwidth}
        \centering
        \caption{Autocorrelations $MAE(L,\, \beta_{unseen})$}
        \includegraphics[width =\textwidth]
        {2_Figures/3_Task/5_Models/11_lb_1_Orig_CNMc.pdf}
    \end{subfigure}
    \vspace{-0.3cm}
    \caption{Results for \emph{Rössler15}, $\beta_{unseen} = 9.6,\, L =1$}
    \label{fig_80}
\end{figure}
% ==============================================================================
% ======================= Rossler 15 ===========================================
% ==============================================================================


% ==============================================================================
% ======================= TS 15 ===========================================
% ==============================================================================
\begin{figure}[!h]
    \begin{subfigure}{0.5\textwidth}
        \centering
        \caption{Illustrative trajectory $\beta = 11$ }
        \includegraphics[width =\textwidth]
        {2_Figures/3_Task/5_Models/12_lb_11.000.pdf}
    \end{subfigure}
    \hfill
    \begin{subfigure}{0.5\textwidth}
        \centering
        \caption{Trajectories, $\beta_{unseen} = 5.1$}
        \includegraphics[width =\textwidth]
        {2_Figures/3_Task/5_Models/13_lb_5.1_All.pdf}
    \end{subfigure}

    \smallskip
    \begin{subfigure}{0.5\textwidth}
        \centering
        \caption{2D-trajectories, $\beta_{unseen} = 5.1$}
        \includegraphics[width =\textwidth]
        {2_Figures/3_Task/5_Models/14_lb_5.1_3V_All.pdf}
    \end{subfigure}
    \hfill
    \begin{subfigure}{0.5\textwidth}
        \centering
        \caption{Autocorrelations, $\beta_{unseen} = 5.1$}
        \includegraphics[width =\textwidth]
        {2_Figures/3_Task/5_Models/15_lb_3_all_5.1.pdf}
    \end{subfigure}
    
    
    \smallskip
    \begin{subfigure}{0.5\textwidth}
        \centering
        \caption{\gls{cpd}, $\beta_{unseen} = 5.1$}
        \includegraphics[width =\textwidth]
        {2_Figures/3_Task/5_Models/16_lb_5.1.pdf}
    \end{subfigure}
    \hfill
    \begin{subfigure}{0.5\textwidth}
        \centering
        \caption{Autocorrelations $MAE(L,\, \beta_{unseen})$}
        \includegraphics[width =\textwidth]
        {2_Figures/3_Task/5_Models/17_lb_1_Orig_CNMc.pdf}
    \end{subfigure}
    \vspace{-0.3cm}
    \caption{Results for \emph{TS15}, $\beta_{unseen} = 5.1,\, L =2$}
    \label{fig_81}
\end{figure}
% ==============================================================================
% ======================= TS 15 ================================================
% ==============================================================================



% % % % ---------------- Task 4 ------------------------------
\chapter{Conclusion and outlook}
A tool to capture and predict the behavior of nonlinear complex and chaotic dynamical systems within a range of some model parameter values $\vec{\beta}$ is presented.
The tool is called \glsfirst{cnmc}.
It could be shown that \gls{cnmc} is able to capture and make predictions for the well-known Lorenz system \cite{lorenz1963deterministic}.
With having removed one of the major limitations in the first attempt of \gls{cnmc} \cite{Max2021}, the introduced version of \gls{cnmc} is not limited to any dimension anymore. 
Furthermore, the restriction of the dynamical system to exhibit a circular trajectory is removed. 
Since these two limitations could be removed, the presented \gls{cnmc} can be applied to any general dynamical system.
To outline this fact, 10 different dynamical systems are implemented by default in \gls{cnmc}.
Some of these dynamical systems were used to evaluate \gls{cnmc} performance.
It could be observed that \gls{cnmc} is not only able to deal with the Lorenz system but also with more complicated systems.
The objective is to represent the characteristic behavior of general dynamical systems that could be fulfilled on all tested systems.\newline 

The third limitation which could be removed is the unacceptably high computational time with \glsfirst{nmf}. 
It could be highlighted that \glsfirst{svd} returns the decomposition within seconds, instead of hours, without adding any inaccuracies.
Moreover, \gls{svd} does not require a parameter study. 
Executing \gls{nmf} once is already computational more expensive than \gls{svd}, but with a parameter study, \gls{nmf} becomes even more unsatisfactory in the application.
By having removed these 3 major limitations, \gls{cnmc} can be applied to any dynamical system within a reasonable computational time on a regular laptop.
Nevertheless, \gls{cnmc} contains algorithms, which highly benefit from computational power. Thus, faster outputs are achieved with clusters.
Also, with having replaced the B-spline interpolation through linear interpolation, the predicted trajectories can be visually depicted appropriately without the 
Another important introduced advancement is that the B-spline interpolation was replaced by linear interpolation. This allows to avoid unreasonably high interpolation errors (oscillations) of the trajectory and enables an appropriate visualization.
\newline


\gls{cnmc} Is written from scratch in a modular way such that implementing it into existing code, replacing employed algorithms with others is straightforward or used as a black-box function.
All important parameters can be adjusted via one file (\emph{settings.py}).
Helpful post-processing features are part of \gls{cnmc} and can also be controlled with \emph{settings.py}.
Overall \gls{cnmc} includes a high number of features, e.g., a log file, storing results at desired steps, saving plots as HTML files which allow extracting further information about the outcome, the ability to execute multiple models consequentially, and activating and disabling each step of \gls{cnmc}.
All displayed outputs in this thesis were generated with \gls{cnmc}. 
Finally, one limitation which remains shall be mentioned.
The used \gls{svd} code receives sparse matrices, however, it returns a dense matrix. The consequence is that with high model orders $L$, quickly multiple hundreds of gigabytes of RAM are required. 
The maximal $L$ which could be achieved on the laptop of the author, which has 16 GB RAM, is $L=7$.\newline

As an outlook, a new \gls{svd} algorithm should be searched for or written from scratch. 
The demand for the new \gls{svd} solver is that it must receive sparse matrices and also returns the solution in form of sparse matrices. 
With that $L$ could be increased, i.e., $L>7$. 
In this thesis, it could be shown that \gls{cnmc} can handle chaotic systems well. Thus, the next step could be, replacing the current data generation step, where differential equations are solved, with actual \gls{cfd} data as input.
Hence, the objective would be to apply \gls{cnmc} to real \gls{cfd} data to predict flow fields.
\chapter{Zusammenfassung auf Deutsch}
Die Arbeit wurde an der Technischen Universität Braunschweig geschrieben.
Da diese Arbeit auf eine Fremdsprache geschrieben wurde, soll der Anforderung der TU-Braunschweig, dass eine Zusammenfassung auf Deutsch, welche etwa 1 DIN A4-Seite beträgt, nachgekommen werden.
Zunächst wird kurz die Motivation dieser Master-Arbeit erklärt. Im Anschluss sollen die Ergebnisse im Kurzen erörtert werden.\newline 

In dieser Master-Arbeit war es Ziel, eine bereits bestehende Methode, das sog. \glsfirst{cnmc}, zu verbessern. Die Vorversion ist in \cite{Max2021} beschrieben. Hier konnte gezeigt werden, dass \gls{cnmc} für das Lorenz System, \cite{lorenz1963deterministic} vielversprechende Approximationen zulässt.
Das Lorenz System ist recht bekannt unter den chaotischen Systemen. Ein chaotisches System ist ein dynamisches System, was selbst durch Differenzialgleichungen beschrieben wird. 
Sinn von \gls{cnmc} ist daher, das Approximieren bzw. Vorhersagen von Trajektorien (zeitliche Lösung der Differenzialgleichung) von  dynamischen Systemen. 
\gls{cnmc} wurde innerhalb der ersten Version speziell für das Lorenz System entwickelt, sodass es nicht für allgemeingültige dynamische System verwendet werden konnte.
Die Limitierungen verlangten unter anderem, dass die Trajektorie kreisförmig seien müsse. Zudem, musste ein 3-dimensionales Problem vorliegen. Weiters kam hinzu, dass ein wichtiger Schritt in dem \gls{cnmc} Arbeitsablauf (Moden-Findung) mehrere Stunden in Anspruch nahm und somit die Anwendung von \gls{cnmc} unattraktiver machte. 
Aufgrund dessen, dass es Schwierigkeiten beim Ausführen der ersten \gls{cnmc}-Version gab, wurde \gls{cnmc} von neu programmiert.\newline


Zunächst wurde der Code nun in der Form geschrieben, dass der Nutzer nach Belieben neue dynamische Systeme einfach hinzufügen kann. Standardmäßig kommt \gls{cnmc} bereits mit 10 verschiedenen dynamischen Systemen. Danach wurden zwei wichtige Limitierungen entfernt. Die Erste, \gls{cnmc} kann inzwischen mit jedem Verhalten der Trajektorie umgehen. In anderen Worten, die Trajektorie des dynamischen Systems muss nicht kreisförmig sein. Zweitens ist \gls{cnmc} nicht mehr durch die Anzahl der Dimension restriktiert. Vereinfacht ausgedrückt, ob \gls{cnmc} auf eine 3d oder eine andere beliege dimensionale Differenzialgleichung angewendet werden soll, spielt keine Rolle mehr.
Für den Schritt, in welchem die Moden einer Daten-Matrix gefunden werden, stehen aktuell zwei verschiedene Möglichkeiten zu Verfügung, \glsfirst{nmf} und \glsfirst{svd}. \gls{nmf} wurde bereits in der ersten Version von \gls{cnmc} verwendet. 
Doch wurde es dahingehend weiter verbessert, dass jetzt das Finden des wichtigen Parameters, der Anzahl der verwendeten Moden, automatisiert durchgeführt wird.
Somit kann \gls{nmf} automatisiert auf unterschiedliche dynamische System angewendet werden.
\gls{svd} ist die zweite Methode und wurde implementiert, um die hohe Rechenzeit des \gls{nmf} zu verhindern.
Es konnte gezeigt werden, dass \gls{svd} tatsächlich, um ein vielfaches schneller als  \gls{nmf} ist. 
Die Rechenzeit von \gls{svd} bewegt sich im Bereich von Sekunden, wohingegen \gls{nmf} mehrere Stunden in Anspruch nehmen kann.
Auch wurde auch gezeigt, dass beide Methoden qualitativ gleichwertige Ergebnisse liefern.\newline


Eine weitere wichtige Änderung, welche in der aktuellen \gls{cnmc} Version implementiert ist die, dass eine sog. B-Spline Interpolation durch eine lineare Interpolation ersetzt wurde. Als Folge können unangebracht hohe Interpolationsfehler (Oszillationen) der Trajektorie umgangen werden. Durch letztere Änderung können die Ergebnisse nun auch Graph dargestellt werden, ohne dass durch die B-Spline Interpolation eingebrachte Ausreißer eine visuelle Auswertung unmöglich machen.\newline 


Mit dieser Arbeit konnte gezeigt werden, dass \gls{cnmc} nicht nur für das Lorenz System, sondern für allgemeingültige dynamische Systeme verwendet werden kann. Hierfür wurden beispielsweise die Ergebnisse für drei andere dynamische Systeme gezeigt. Die aktuelle \gls{cnmc} Version wurde in einer modularen Art geschrieben, welche es erlaubt, einzelne Algorithmen leicht durch andere zu ersetzen. 
Jeder einzelne Haupt-Schritt in \gls{cnmc} kann aktiviert oder deaktiviert werden. Dadurch können bereits vorhanden Ergebnisse eingeladen werden, anstatt diese jedes Mal neu zu berechnen. Das Resultat ist eine hohe Ersparnis an Rechenzeit. \gls{cnmc} kommt mit vielen Features, über eine einzige Datei lässt sich der gesamte Ablauf von \gls{cnmc} steuern. Wodurch bestimmt werden kann,  welche Parameter in den einzelnen Schritten verwendet werden, wo Ergebnisse abgespeichert und geladen werden sollen, sowie auch wo und ob die Ergebnisse visuell abgespeichert werden sollen. 
Die Resultate werden für die visuelle Inspektion als HTML-Dateien zur Verfügung gestellt. Damit ist es möglich weitere Informationen zu erhalten, wie beispielsweise, das Ablesen von Werten an bestimmten Stellen und anderen nützlichen Funktionen, wie etwa das Rotieren, Zoomen und Ausblenden einzelner Graphen. 
Das Ziel war es, dem Nutzer einen Post-Processor mitzugeben, sodass er auch ohne weitere kostenpflichtige Software visuelle Auswertungen vornehmen kann. Doch \gls{cnmc} hat auch eine log-Datei integriert, in welcher alle Ausgaben, wie unter anderem Ergebnisse einzelner Qualitätsmesstechniken (Metriken bzw. Normen) nachgelesen werden können.\newline


Zusammenfassend lässt sich sagen, mit dieser Master-Thesis befindet sich \gls{cnmc} in einem Zustand, in welchem es für allgemeingültige dynamische Systeme angewendet werden kann. Das Implementieren von weiteren Systemen wurde vereinfacht und wichtige Limitierungen, wie Anzahl der Dimensional und unzulässig hohe Rechenzeit konnten beseitigt werden. Zudem ist das Tool gut dokumentiert, und bietet diverse Features an, worunter beispielsweise die Post-Processing Möglichkeiten inbegriffen sind.







% % % ---------------- Appendix ------------------------------
\appendix
\chapter{Further implemented dynamical systems}
\label{ch_Ap_Dyna}
\begin{enumerate}
    \item \textbf{Chen} \cite{Chen1999}: 
    \begin{equation}
        \label{eq_8_Chen}
        \begin{aligned}
            \dot x &= a\, (y - x) \\
            \dot y &= x \,(\beta - a) - xz + \beta y \\
            \dot z &= x y -b z
        \end{aligned}
    \end{equation}

    \item \textbf{Lu} \cite{Lu2002}:
    \begin{equation}
        \label{eq_9_Lu}
        \begin{aligned}
            \dot x &= a \, (y -x) \\
            \dot y &= \beta y -x z  \\
            \dot z &= x y - b z
        \end{aligned}
    \end{equation}

    \item \textbf{Van der Pol} \cite{VanderPol}:
    \begin{equation}
        \label{eq_14_VDP}
        \begin{aligned}
            \dot x &= y \\
            \dot y &= y \beta\,(1-x^2) -x
        \end{aligned}
    \end{equation}

\end{enumerate}

\chapter{Some basics about chaotic systems}
\label{ch_Ap_Chaotic}
Since 
Chaotic systems are the height 
of intricacy when considering dynamical systems. 
The reason why the term intricacy was chosen
instead of complexity is that chaotic systems can be, but are not necessarily 
complex. For the relation between complex and 
chaotic the reader is referred to \cite{Rickles2007}. 
The mentioned intricacy of chaotic systems shall be explained by 
reviewing two reasons. First, 
chaotic systems are sensitive to their initial conditions. 
To understand this, imagine we want to solve an \gls{ode}. In order to solve any
differential 
equation, the initial condition or starting state must be known. Meaning, that the 
solution to the \gls{ode} at the very first initial step, from where the 
remaining interval is solved, must be identified beforehand. 
One might believe, a starting point, which is not guessed unreasonably off, 
should suffice to infer the system's future dynamics.\newline

This is 
an educated attempt, however, it is not true for systems that exhibit
sensitivity to initial conditions. These systems amplify any 
perturbation or deviation exponentially 
as time increases. From this it can be concluded
that even in case the initial value would be accurate to, e.g., 10 decimal places,
still after some time, the outcome can not be trusted anymore. 
Visually 
this can be comprehended by thinking of initial conditions
as locations in space. Let us picture two points with two initial conditions
that are selected to be next to each other. Only by zooming in multiple times, 
a small spatial deviation should be perceivable. 
As the time changes, the points will leave the location defined through the initial condition. \newline 


With 
chaotic systems in mind, both initially neighboring 
points will diverge exponentially fast from each other.
As a consequence of the initial condition not being
known with infinite precision, the initial microscopic
errors become macroscopic with increasing time. Microscopic mistakes 
might be considered to be imperceptible and thus have no impact 
on the outcome, which would be worth to be mentioned.
Macroscopic mistakes on the other hand are visible. Depending on 
accuracy demands solutions might be or might not be accepted.
However, as time continues further, the results eventually 
will become completely unusable and diverge from the actual output on a macroscopic scale.\newline 


The second reason, why chaotic systems are very difficult 
to cope with, is the lack of a clear definition. It can be 
argued that even visually, it is not always possible to
unambiguously identify a chaotic system. The idea 
is that at some time step, a chaotic system appears to 
be evolving randomly over time. The question then arises,
how is someone supposed to distinguish between something which 
is indeed evolving randomly and something which only appears 
to be random. The follow-up question most likely is going to be, 
what is the difference between chaos and randomness, or 
even if there is a difference. \newline 

Maybe randomness itself is only 
a lack of knowledge, e.g., the movement of gas particles 
can be considered to be chaotic or random. If the 
velocity and spatial position of each molecule are 
trackable, the concept of temperature is made 
redundant. Gibbs only invented the concept of temperature 
in order to be able to make some qualitative statements 
about a system \cite{Argyris2017}.
A system that can not be described microscopically.
Here the question arises if the movement of the molecules 
would be random, how is it possible that every time 
some amount of heat is introduced into a system, the temperature
changes in one direction. If a random microscale system 
always tends to go in one direction within a macroscale view,  
a clear definition of randomness is required. \newline

Laplace once said if the initial condition
(space and velocity) of each atom would be known,  
the entire future 
could be calculated. In other words, if a system is 
build on equations, which is a deterministic way 
to describe an event, the outcome should just 
depend on the values of the variables. 
Thus, the future, for as long as it is desired could be predicted 
or computed exactly. To briefly summarize this conversion, 
Albert Einstein once remarked that God would not play dice. Nils 
Bohr replied that it 
would be presumptuous of us human beings to prescribe to the Almighty 
how he is to take his decisions. A more in-depth introduction to 
this subject is provided by \cite{Argyris2017}. 
Nevertheless, by doing literature research, one way to 
visually distinguish between
randomness and chaos was found \cite{Boeing2016}. 
Yet, in \cite{Boeing2016} the method was only 
deployed on a logistic map. Hence, further research 
is required here. \newline

As explained, a clear definition of chaos does not exist. 
However, some parts of definitions do occur regularly, e.g., 
the already mentioned \glsfirst{sdic}. Other definition parts are the following: Chaotic 
motion is \textbf{aperiodic} and based on a \textbf{deterministic} system.
An aperiodic system is not repeating any 
previous \textbf{trajectory} and a deterministic system is 
described by governing equations. A trajectory is the evolution 
of a dynamical system over time. For instance, a dynamical system 
consisting of 3 variables is denoted as a 3-dimensional dynamical system.
Each of the variables has its own representation axis.
Assuming these 
3 variables capture space, motion in the x-,y- and z-direction 
is possible. For each point in a defined time range, there is one set of x, y and z values, which fully describes the output of the dynamical system or the position at a chosen time point. 
Simply put, the trajectory is the movement 
or change of the variables of the differential equation over time. Usually, the 
trajectory is displayed in the phase space, i.e., the axis represents the state or values of the variables of a dynamical system. An example can be observed in section \ref{subsec_1_1_3_first_CNMc}. \newline


One misconception which is often believed \cite{Taylor2010} 
and found, e.g., in
Wikipedia \cite{Wiki_Chaos} is that
strange attractors would only appear as a consequence of 
chaos. Yet, Grebogi et al. \cite{Grebogi1984} proved
otherwise. According to 
\cite{Boeing2016,Taylor2010} strange attractors exhibit 
self-similarity. This can be understood visually by imaging any shape 
of a trajectory. Now by zooming in or out, the exact same shape 
is found again. The amount of zooming in or out and consequently 
changing the view scale, will not change the perceived 
shape of the trajectory. Self-similarity happens to be
one of the fundamental properties of a geometry 
in order to be called a fractal \cite{Taylor2010}. 
In case one believes,
strange attractors would always be chaotic and knows that by definition strange attractors phase 
space is self-similar, then 
something further misleading is concluded.
Namely, if a geometry is turned out not only
to be self-similar but also to be a fractal, this 
would demand interpreting every fractal to be
chaotic. \newline 

To refute this, consider the Gophy 
attractor \cite{Grebogi1984}. 
It exhibits the described self-similarity,  
moreover, it is a fractal, and it is also a 
strange attractor. However, the Gophy 
attractor is not chaotic. The reason is found, when 
calculating the Lyapunov exponent, which is negative
\cite{Taylor2010}. Latter tells us that two neighboring 
trajectories are not separating exponentially fast 
from each other. Thus, it does not obey the 
sensitive dependence 
of initial conditions requirement and is 
regarded to be non-chaotic. The key messages are 
that a chaotic attractor surely is a strange 
attractor and a strange attractor is not necessarily 
chaotic. A strange attractor refers to a fractal
geometry in which chaotic behavior may
or may not exist \cite{Taylor2010}. 
Having acquired the knowledge that strange attractors 
can occur in chaotic systems and form a fractal, 
one might infer another question. If a chaotic 
strange attractor always generates a geometry, which 
stays constant when scaled, can chaos be 
regarded to be random?\newline 


This question will not be discussed in detail here, but for the sake of completeness, the 3 known types of nonstrange attractors 
shall be mentioned. These are
the fixed point attractor, the limit cycle attractor, and the
torus attractor \cite{Taylor2010}. 
A fixed point attractor is one point in the phase space, which attracts or pulls nearby trajectories to itself. 
Inside the fix-point attractor, there is no motion, meaning 
the derivative of the differential equation is zero. 
In simpler words,
once the trajectory runs into a fix-point, the trajectory ends there. 
This is because no change over time can be found here. 
A limit cycle can be expressed as an endlessly repeating loop, e.g. in the shape of a circle. 
The trajectory can start at
any given initial condition, still, it can go through a place in the phase space, from where the trajectory is continued as an infinitely
repeating loop. 
For a visualization of the latter and the tours, as well more 
detail the reader is referred to \cite{Argyris2017, Kutz2022, Strogatz2019, Taylor2010}.

% =====================================================================
% ========================= Bib =======================================
% =====================================================================
% % % define the bib file
\bibliography{3_Bibtex/jabref.bib}


\end{document}